\begin{abstract}
 
Cyber Physical Systems are rather prolific these days, and due to the complexity of these systems, system designers have to account for requirements that go beyond the tracking/stability of traditional control and the purely spatial reach-avoid requirements common in trajectory planning. Metric Temporal Logic (MTL) allows us to express these classical requirements, as well as more complex reactive and spatio-temporal ones in an easy to represent form. Unfortunately, despite the ease of representation, synthesis of control algorithms to satisfy these requirements either do not scale well to real world problems of a larger scale (MILP based techniques), or provide little to no performance guarantees (stochastic heuristics). By using a smooth approximation of the robustness degree of MTL specifications, we leverage powerful off the shelf optimization algorithms to offer performance guarantees in terms of convergence to local minima. In this work, we aim to build a tool that leverages this method, and uses parallelization to provide control for large scale problems with complex MTL specifications. Through two case studies, one on autonomous centralized ATC for the Chicago tracon area, and another on centralized ATC for autonomous medical drones in Montreal we show the ability to control systems with MTL specifications on a scale previously unrealizable. The general purpose nature of our approach leaves the door open for application to other problems as well where the requirements can be encoded as MTL specifications.
 
 
% Cyber-Physical Systems must withstand a wide range of errors, from bugs in their software to attacks on their physical sensors.
% Given a formal specification of their desired behavior in Metric Temporal Logic (MTL), the robust semantics of the specification provides a notion of \textit{system robustness} that can be calculated directly on the output behavior of the system, without explicit reference to the various sources or models of the errors.
% The robustness of the MTL specification has been used both to verify the system offline (via robustness minimization) and to control the system online (to maximize its robustness over some horizon).
% Unfortunately, the robustness objective function is difficult to work with: it is recursively defined, non-convex and non-differentiable.
% In this paper, we propose smooth approximations of the robustness. 
% Such approximations are differentiable, thus enabling us to use powerful off-the-shelf gradient descent algorithms for optimizing it.
% By using them we can also offer guarantees on the performance of the optimization in terms of convergence to minima.
% We show that the approximation error is bounded to any desired level, and that the approximation can be tuned to the specification.
% We demonstrate the use of the smooth robustness to control two quad-rotors in an autonomous air traffic control scenario, and for temperature control of a building for comfort.
\end{abstract}
