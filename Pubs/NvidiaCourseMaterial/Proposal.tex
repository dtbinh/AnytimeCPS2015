\documentclass{article}[14pt]

\usepackage{amsmath}
\usepackage{amssymb}
\usepackage{graphicx}
\usepackage{float}
\usepackage{array}
\usepackage{tikz}
\usepackage{latexsym}
\usepackage{xspace}
\usepackage{algorithm2e}
\setlength{\textheight}{9in}
\setlength{\textwidth}{6.5in}
\setlength{\columnsep}{0.3125in}
\setlength{\topmargin}{-0in}
\setlength{\headheight}{-0in}
\setlength{\headsep}{0in}
\setlength{\parindent}{1pc}
\setlength{\oddsidemargin}{0in}

%\parindent=0pt

\title{Course Material Development Proposal}

\begin{document}
\maketitle
\section{University Services Overview}
\subsection{Overview}

The course focuses on the state-of-the-art algorithms for autonomous navigation and their implementation on low power computation platforms with energy budgets. An emphasis will be on the computation requirements of the algorithms and how the use of parallel computing, especially how GPU implementations can improve computation performance and the closed-loop performance of the system. Of particular interest is the Nvidia Jetson, where computation tasks can be deployed on the CPU (low throughput, low power) or the GPU (high throughput high power). In order to understand the implementation details and benefits/trade-offs offfered by deploying tasks on the GPU, a part of the course material will be on \emph{<Joe?>} and will scale to platforms other than the Jetson. The course will cover topics ranging from processing the data from various sensors on an autonomous car (IMU, Camera, Lidar) in the Robot Operating System (ROS), localization, mapping, path planning and Control of the car for tracking planned trajectories.  The course will have regular projects based on the algorithms covered and implementation on a $1/10^{th}$ scale autonomous racing platform with a host of sensors and the Nvidia Jetson computation platform. The final project will be an autonomous car navigation with focus on not only perfomance of the car but also on energy consumption by the computation platform.
\subsection{Schedule}

\begin{center}
\begin{tabular}{ | l | l |}
\hline
Start of UNIVERSITY Services & Effective Date (TBD) \\
\hline
.. & .. \\
\hline
\end{tabular}
\end{center}

\pagebreak

\section{NVIDIA Deliverables and NVIDIA Delivery Schedule}
\subsection{Description}
\begin{center}
\begin{tabular}{ | l | l |}
\hline
Nvidia Deliverable & Description \\
\hline
.. & .. \\
\hline
\end{tabular}
\end{center}

\subsection{Delivery Date}

\begin{center}
\begin{tabular}{ | l | l |}
\hline
Nvidia Deliverable & Delivery Date \\
\hline
.. .. & .. \\
\hline
\end{tabular}
\end{center}

\pagebreak

\section{UNIVERSITY Deliverables and UNIVERSITY Delivery Schedule}

\subsection{Description}

\begin{center}
\begin{tabular}{ | l | l | l |}
\hline
Delivery Code & UNIVERSITY deliverable & Description \\
\hline
..  & .. & ..\\
\hline
\end{tabular}
\end{center}

\begin{center}
\begin{tabular}{ | l | l | l |}
\hline
Deliverables: & .. & .. \\
\hline
.. .. & .. & .. \\
\hline
\end{tabular}
\end{center}

\pagebreak

The outline for which the UNIVERSITY Deliverables are to be based on is the following:

\begin{center}
\begin{tabular}{ | l || l |}
\hline
Modules & .. \\
\hline
1 & Course introduction, Prior results from Autonomous Driving (DARPA Grand Challenge, the Google Car) , the autonomous racing platform \\
\hline
2 & ROS Introduction and Advanced topics, introduction to the Jetson \\
\hline
3 & GPU programming \emph{<Joe?>} \\
\hline
.. & ..\\
\hline
.. & Common Sensors in Autonomous Cars and their individual benefits/limitations, Reading and Visualizing sensor data in ROS \\
\hline
..  & State Estimation and Sensor fusion. Kalman filter basics, EKF and UKF,  Particle Filters \\
\hline
.. & Introduction to feedback control. PID and State-feedback based methods. \\
\hline
.. & Mapping and Localization \\
\hline
.. & Mission and trajectory planning \\
\hline
.. & Implementation considerations on low computation power platforms \\
\hline 
.. & CPU-GPU task allocation, consideration of energy budgets \\
\hline
\end{tabular}
\end{center}

Indicates Evaluation Kit modules
Indicates slide decks to be developed

\subsection{Delivery Dates}
\begin{center}
\begin{tabular}{ | l | l |}
\hline
UNIVERSITY Deliverables & Delivery Date \\
\hline
.. & ..\\
\hline
.. & ..\\
\hline
\end{tabular}
\end{center}

\subsection{Detailed Deliverable Descriptions and Requirements}

\end{document}
