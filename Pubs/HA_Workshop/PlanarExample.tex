\subsection{Planar example}

\begin{equation}
x_{k+1} = x_k + u_k
\end{equation}

Where $x_k,u_k \in \mathbb{R}^2$ and $ u_k \in [-0.52,0.52]^2 \forall k$. The sampling rate is $10$Hz.The specification to be satisfied is
\begin{equation}
\formula = \always_{[0,T]} \neg (x_k \in [-1,1]^2) \land \eventually_{[0,T]} (x_k \in [2,2.5]^2)
\end{equation}

The evaluation purposes, we vary the formula time-horizon $T$ from $2$ to $5$ seconds, resulting in control horizons of $20$ to $50$ steps at $10$Hz. For comparison to BluSTL, we run BluSTL in \textit{boolean}, i.e. satisfaction of $\formula$ mode and, \textit{robust} mode which attempts to maximize robustness with its corresponding MILP formulation. For both modes of BluSTL, the control cost (norm of $u$) is set to zero.

\begin{table}[htb]
\small
\begin{center}
\caption{{\small Timing for satisfaction of $\formula$ for SR-SQP and BlueSTL over 100 runs with random initial states.}}
\vspace{-10pt}
\label{tbl:opt_performance}
\begin{tabular} {|c|c|c|c|c|}
	\hline
	\textbf{T} & Mean exec. time BS (s) & Std. exec. time BS (s) &  Mean Exec. time SR-SQP (s) & Std. exec. time SR-SQP (s)\\ \hline
	2 & 0.9592 & 0.8229 & \textbf{0.3140} & 0.1346 \\ \hline
	3 & 1.3696 & 1.7165 & \textbf{0.3297} & 0.2558\\ \hline
	4 & 3.8643 & 5.0953  & \textbf{0.6077} & 0.2958\\ \hline
	5 & 4.3606 & 12.9720  & \textbf{0.7394} & 0.2982\\ \hline
\end{tabular}	
\end{center}
\vspace{-20pt}
\end{table}

When in satisfaction mode, BluSTL results in trajectories with zero ($-10^{-11}$) robustness (as seen in), while our method (which stops optimizing on getting a robustness greater than $\epsilon$) results in trajectories with an average robustness of $0.10$. 

\begin{table}[htb]
\small
\begin{center}
\caption{{\small Timing for robustness maximization of $\formula$ for SR-SQP and BlueSTL over 100 runs with random initial states.}}
\vspace{-10pt}
\label{tbl:opt_performance}
\begin{tabular} {|c|c|c|c|c|}
	\hline
	\textbf{T} & Mean exec. time BS (s) & Std. exec. time BS (s) &  Mean Exec. time SR-SQP (s) & Std. exec. time SR-SQP (s)\\ \hline
	2 & NA & NA & \textbf{0.3140} & 0.1346 \\ \hline
	3 & NA & NA & \textbf{0.3297} & 0.2558\\ \hline
	4 & NA & NA  & \textbf{0.6077} & 0.2958\\ \hline
	5 & NA & NA  & \textbf{0.7394} & 0.2982\\ \hline
\end{tabular}	
\end{center}
\vspace{-20pt}
\end{table}
