\begin{abstract}

Many modern systems, from swarms of unmanned aerial vehicles (UAVs) to battery management systems, have complex spatio-temporal requirements that go beyond the traditional notions of stability and tracking. In general, provably safe control of a large scale system like Air Traffic Control or a swarm of UAVs is a hard problem to specify formally and even harder to solve in real-time. While the complex underlying specifications of such systems can be mathematically captured in the framework of temporal logic, specifically Metric Temporal Logic (MTL), the problem of scalable control of dynamical systems given such specifications is not yet solved. State of the art approaches for control of systems with MTL specifications \cite{Raman14_MPCSTL}, \cite{Saha_acc16} rely on solving Mixed Integer Programming (MIP) problems as the underlying optimization for control, and we show that they lack the scalability to solve problems of a scale that involves multiple dynamical agents with complex specifications. 

We introduce smooth (infinitely differentiable) approximations to the robustness function of MTL formulas \cite{PantAM17_SmoothOpTechRpt}, which can be made arbitrarily close to the true robustness value. This allows us to use powerful off the shelf gradient descent optimizers to maximize the smooth robustness function. Formally, if the system satisfies the formula with robustness $r$, then any disturbance of size less than $r$ cannot cause it to violate the formula. This implies that if the optimization can find a robustness value $r>0$, then the underlying MTL specification is robustly satisfied. In addition to using this optimization for the control of dynamical systems with MTL specifications, we show how the structure of the robustness function itself, along with that of the specifications for multi-agent systems can be leveraged to massively parallelize the robustness computation. This allows us to evaluate multiple runs of the robustness function in and optimize over it in real-time using gradient based optimization solvers. We show how our proposed technique works on a simulated Air Traffic Control problem \cite{MaxCAJ} for the Chicago O'Hare (ORD) airport terminal region. Simulations show our method can operate at scales impossible for existing tools for control with MTL specifications, as well as the overall performance of the systemc compared to historical data of manual ATC operation.

\end{abstract}
