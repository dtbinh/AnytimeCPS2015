\subsection{The need for temporal logic}
\label{sec:morari}
The above requirements go beyond traditional control objectives like stability, tracking, quadratic cost optimization and reach-while-avoid for which we have well-developed theory.
While these requirements can be directly encoded from natural language into a Mixed Integer Program (MIP), such direct encoding can easily have thousands of variables, as will be seen in the experiments. 
Thus it is error-prone and checking that it corresponds to the designer's intent is near impossible.
On the other hand, such control requirements are easily and succinctly expressed in Metric Temporal Logic (MTL)~\cite{Koymans90,Ouaknine08_RecentResultsMTL}.
MTL is a formal language for expressing reactive requirements with constraints on their time of occurence and sequencing, such as those of the ATC.
The advantage of first expressing them as MTL formulas is that such formulas are more succinct and legible, and less error-prone, than the corresponding MIP.
In this sense, MTL bridges the gap between the ease of use of natural language and the rigor of mathematical formulation.
%
For example, ATC rule (A) can be formalized with the following MTL formula ($\always$ means `Always', $q$ is an aircraft and $q_z$ is its altitude).
\begin{equation*}
\label{eq:rule1mtl}
\always( q \in Zone1 \implies q_z \leq \text{Ceiling1} \land q_z \geq \text{Floor1})
\end{equation*}
Rule (B) can be formalized as follows.
\begin{flalign*}
\label{eq:rule3mtl}
\always(Busy \implies\eventually_{[t_1,t_2]} (&q \in \text{Holding-6} \, \lor \,q \in \text{Holding-7}) 
\nonumber \\
&\until_{[0,\text{MaxHolding}]} \neg Busy)
\end{flalign*}

This says that Always ($\always$), if airport is Busy, then sometime within $t_1$ to $t_2$ seconds ($\eventually_{[t_1,t_2]} $), the plane goes into one of two Holding areas.
It stays there $\until$ntil the airport is not ($\neg$) busy, which must happen before duration MaxHolding elapses.
