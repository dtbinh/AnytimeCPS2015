% This is "sig-alternate.tex" V2.1 April 2013
% This file should be compiled with V2.5 of "sig-alternate.cls" May 2012
%
% This example file demonstrates the use of the 'sig-alternate.cls'
% V2.5 LaTeX2e document class file. It is for those submitting
% articles to ACM Conference Proceedings WHO DO NOT WISH TO
% STRICTLY ADHERE TO THE SIGS (PUBS-BOARD-ENDORSED) STYLE.
% The 'sig-alternate.cls' file will produce a similar-looking,
% albeit, 'tighter' paper resulting in, invariably, fewer pages.
%
% ----------------------------------------------------------------------------------------------------------------
% This .tex file (and associated .cls V2.5) produces:
%       1) The Permission Statement
%       2) The Conference (location) Info information
%       3) The Copyright Line with ACM data
%       4) NO page numbers
%
% as against the acm_proc_article-sp.cls file which
% DOES NOT produce 1) thru' 3) above.
%
% Using 'sig-alternate.cls' you have control, however, from within
% the source .tex file, over both the CopyrightYear
% (defaulted to 200X) and the ACM Copyright Data
% (defaulted to X-XXXXX-XX-X/XX/XX).
% e.g.
% \CopyrightYear{2007} will cause 2007 to appear in the copyright line.
% \crdata{0-12345-67-8/90/12} will cause 0-12345-67-8/90/12 to appear in the copyright line.
%
% ---------------------------------------------------------------------------------------------------------------
% This .tex source is an example which *does* use
% the .bib file (from which the .bbl file % is produced).
% REMEMBER HOWEVER: After having produced the .bbl file,
% and prior to final submission, you *NEED* to 'insert'
% your .bbl file into your source .tex file so as to provide
% ONE 'self-contained' source file.
%
% ================= IF YOU HAVE QUESTIONS =======================
% Questions regarding the SIGS styles, SIGS policies and
% procedures, Conferences etc. should be sent to
% Adrienne Griscti (griscti@acm.org)
%
% Technical questions _only_ to
% Gerald Murray (murray@hq.acm.org)
% ===============================================================
%
% For tracking purposes - this is V2.0 - May 2012

\documentclass{sig-alternate-05-2015}
\usepackage{amsmath} % assumes amsmath package installed
\usepackage{amssymb}  % assumes amsmath package installed
\usepackage{algorithm}
\usepackage{algpseudocode}
\usepackage{algorithmicx}
\usepackage{graphicx}
\usepackage{fainekos-macros}
\usepackage{csquotes}

\newcommand{\sd}{\rho}
\newcommand{\bs}[2]{\ll\!#1,#2\!\gg}
\newcommand{\rd}[2]{\ll\!#1,#2\!\gg_\Re}
\newcommand{\rs}[2]{\ensuremath{\llbracket #1,#2 \rrbracket}}
\newcommand{\crs}[2]{\rs{#1}{#2}_{\sd_\tau}}
\newcommand{\cl}[1]{\overline{#1}}
\newcommand{\Ltf}{\Lc_t(\formula)}
\newcommand{\Ltnp}{\Lc_t(\lnot p)}
\newcommand{\Ltp}{\Lc_t(p)}
\newcommand{\Lrf}{\Lc_{\Re_\Fc}}
\newcommand{\psd}{psd}
\newcommand{\dird}{\overrightarrow{d}}


% To-dos...
%\usepackage{xargs}                      % Use more than one optional 
%\usepackage[pdftex,dvipsnames]{xcolor}  % Coloured text etc. 
%\usepackage[colorinlistoftodos]{todonotes}
%setlength{\marginparwidth}{4cm}

\setboolean{HIGHLIGHTREDCHANGES}{FALSE}
% Control spacing...
\setlength{\textfloatsep}{0pt}
\linespread{0.97}

\thickmuskip=0mu

\begin{document}

\CopyrightYear{2017} \setcopyright{acmcopyright}
\conferenceinfo{ICCPS '17,}{April 18--21, 2017, Pittsburgh, Pennsylvania, USA.}
\isbn{978-1-4503-3955-1/16/04}\acmPrice{\$15.00}
\doi{http://dx.doi.org/10.1145/2883817.2883841}
%Authors, replace the red X's with your assigned DOI string.

\title{Control and falsification using smooth robust semantics of temporal logic}

\author{
	% 1st. author
	\alignauthor
	Three little piggies\\
	\affaddr{Department of Electrical and Systems Engineering}\\
	\affaddr{University of Pennsylvania, Philadelphia, PA, USA}\\
	\email{\{piggies1,2,3\}@seas.upenn.edu}%
}


\maketitle
\begin{abstract}
 
 
 
 
 Cyber-Physical Systems must withstand a wide range of errors, from bugs in their software to attacks on their physical sensors.
 Given a formal specification of their desired behavior in Metric Temporal Logic (MTL), the robust semantics of the specification provides a notion of \textit{system robustness} that can be calculated directly on the output behavior of the system, without explicit reference to the various sources or models of the errors.
 The robustness of the MTL specification has been used both to verify the system offline (via robustness minimization) and to control the system online (to maximize its robustness over some horizon).
 Unfortunately, the robustness objective function is difficult to work with: it is recursively defined, non-convex and non-differentiable.
 In this paper, we propose smooth approximations of the robustness. 
 Such approximations are differentiable, thus enabling us to use powerful off-the-shelf gradient descent algorithms for optimizing it.
 By using them we can also offer guarantees on the performance of the optimization in terms of convergence to minima.
 We show that the approximation error is bounded to any desired level, and that the approximation can be tuned to the specification.
 We demonstrate the use of the smooth robustness to control two quad-rotors in an autonomous air traffic control scenario, and for temperature control of a building for comfort.
\end{abstract}

\section{Motivation}
\label{sec:motivation}

Autonomous vehicles promise significant benefits to society, from reduced accident rates to greater mobility for the elderly. 
The biggest challenge in the design of autonomous vehicles comes from the uncertainty of the environment in which they will operate. 
Their control algorithms must be able to cope with driving events that occur on widely ranging time scales. 
For example, relaxed rural driving can accommodate planning actions every few seconds, while imminent collision avoidance requires planning and actuation on the order of a few milliseconds.
Thus `real-time' performance will imply different things depending on the context. 

A second, related, challenge is that the perception algorithms on-board these vehicles (like object detection based on video feed) must handle a very large amount of data, leading to increased power consumption. 
This is especially true for autonomous vehicles (AVs) since they carry multiple sensors (cameras, LIDAR, radars, ultrasound radars, etc) whose data must be processed in real-time to avoid accidents.

As a result of the variability in the environment, the control and perception systems are over-engineered to operate as if the worst-case conditions always hold. 
This results in unnecessarily high power consumption from the computation platform. 
For example, an autonomous research vehicle has two 4KW power inverters dedicated to computation.
Compare this to the average power drawn by the drivetrain of a small electric vehicle, which is 2.4KW, when modeled  in ADVISOR \cite{nreladvisor} going through the Urban Dynamometer Drive Cycle.
(The peak power draw in the same test is 30 KW).
It is also worth noting that over the past few decades, the power consumption of processors has increased by more than double, while battery energy density has only improved by about a quarter \cite{Lahiri}. 

In the current work, we explore the idea of trading-off computation time and power for quality of output, and how this trade-off affects \emph{control performance}.
We start from the observation that the best quality output from the perception algorithm is not always required for the system to achieve the desired control performance.
For example, the control objective might be to follow the center of a driving lane, and control performance is measured by the deviation from that center.
At slow speeds, poor quality of position estimate may be tolerated since it won't lead to excessive deviations from the center.
Therefore we focus on \emph{anytime perception algorithms} that have a pre-defined set of interruption times. 
In general, the earlier the algorithm is interrupted, the worse is the quality of its output. 
On the other hand, that quality may be sufficient for the control algorithm to achieve its goal.

In \cite{RTSS15} we proposed a way in which a standard perception algorithm can be turned into an anytime algorithm via off-line profiling, and thus can offer a time/power/quality trade-off.
We also designed a model predictive controller than can make use of the trade-off offered by the anytime perception algorithm.
To achieve the time/power/quality trade-off, we produced multiple versions of the perception algorithm.
Broadly speaking, a version that ran for longer produced a higher quality output. 

In this work, we turn our attention to achieving the time/power/control performance trade-off \emph{at a fixed level of output quality}, using \emph{platform-level} optimizations.
Even if the output quality of the perception algorithm is fixed, the \emph{speed} at which the output is computed affects control performance: in general, the longer the computation delay, the worse the control performance.
Specifically, we work with an autonomous car $1/10^{th}$ the size of a regular car (Fig. \ref{fig:traxxas}).
\begin{figure}[t]
	\centering
	\includegraphics[scale=0.08]{traxxas.jpg}
	\caption{Traxxas autonomous car with camera.}
		\label{fig:traxxas}
\end{figure}  
It is equipped with a front facing monocular camera and runs the Vanishing Point perception algorithm \cite{VP1}. 
The on-board computation platform has both a CPU and GPU. 
More details of the experimental setup are in Section \ref{sec:experimentalSetup}.
The platform-level optimization divides the Vanishing Point algorithm into components, and decides whether to run each component on the CPU or GPU, and at what frequency.
The assignment to CPU or GPU is not static: every time the algorithm is executed, a different assignment may result, at a different frequency.
The assignment is dictated by the controller: based on off-line profiling of each processor's performance on each component, the controller decides what quality is currently acceptable while minimizing power consumption. 
The profiling is described in more detail in Section \ref{sec:profiling}.
In the present paper, we present initial results of partitioning the Vanishing Point algorithm and running different components on CPU and GPU, and at different frequencies.
We also provide the related power numbers, demonstrating that a meaningful difference in runtime and power consumption exists depending on the assignment and frequency.
In future work, we design a controller to make use of this and other optimizations, and relate the current results to the control algorithm.


\input{robustsemantics}
%\input{qualitativeSmoothVsRaw}
%\input{smoothVsSA}
%\section{Case study}
\label{sec:case study}
\todo[inline]{introduce two numerical examples. Finite horizon control in a single shot, can be applied in a receding/shrinking horizon approach wherever applicable.}


\subsection{HVAC Control of a building for comfort}
%caseBldg

In order to evaluate the control performance of our method on a system with more meaningful dynamics than the point-mass system, we test it on temperature control of a 4-state model of a single zone room in a building (\cite{HAMLAB}). Such a model is commonly used in literature for evaluation of predictive control algorithms \cite{AchinACC}. The control problem we solve is similar to the example used in \cite{Raman14_MPCSTL}, where the objective is to bring the room temperature to a comfortable range when the building is occupied (given predictions on the building occupancy). The specification is gives as:
\begin{equation}
\label{eq:BldgSpec}
\Psi = \always_I(\text{RoomTemp} \in \text{Comfort})
\end{equation}
Here, $I$ is the interval where the room is occupied, and $\text{Comfort}$ is the range of temperatures (in Celsius) deemed comfortable ($[22,28]$). For the control horizon, we consider a 24 hour period, in which the building is occupied from time steps $10$ to $19$ (i.e. $I=[10,19]$), corresponding to a 10-hour workday. The single-zone model, discretized at a sampling rate of 1 hour (which is common in building temperature control) is of the form:

\begin{equation}
x_{k+1} = Ax_{k}+Bu_k+B_dd_k
\end{equation}
Here, $A$, $B$ and $B_d$ matrices are from the HAMLAB model. $x \in \mathbb{R}^4$ is the state of the model, the $4^{th}$ element of which is the room temperature, the others are auxiliary temperatures corresponding to other physical properties of the zone. The input to the system, $u \in \mathbb{R}^1$ is the heating/cooling energy to the system. $b_d \in \mathbb{R}^3$ are disturbances (due to occupancy, outside temperature, solar radiation), which we assume perfect predictions of. Data for the disturbances is obtained for $1^{st}$ April 2000, which is our 24 hours of interest, from the ??? data set. The control problem we solve is of the form in \eqref{eq:general_ctrl}, with $\gamma$ and $\delta$ both set to zero, i.e. robustness is only in the objective, to be maximized. This allows for a fair comparison between our method (SQP on smooth robustness), gradient descent (SQP) on robustness, and Simulated Annealing (SA) for robustness maximization. With respect to the general control problem of \eqref{eq:general_ctrl}, the limits on the states are $X=[0,50]^4$ and on the inputs $U=[-1000,2000]$.

To initialize the optimzation for all three methods, we generate an initial trajectory, starting from $x_0=[21 \, 21 \, 21 \, 21]'$, which does not satisfy $\formula$. The trajectories after optimization from the three methods are shown in Fig.\ref{fig:ZoneTemp}. Our method and SA both result in trajectories that satisfy $\formula$, with a robustness of $2.9994$ and $2.8862$ respectively. On the other hand, SQP for gradient descent on the robustness function results in a trajectory that does not satisfy $\formula$ ($\rob_{\formula} = -0.1492$), and terminates on a local maxima. This is possibly due to the lack of existence of a gradient along certain directions. 

In the particular problem, with the given comfort range, the maximum robustness achievable in the unconstrained case would $3$, achieved by setting the room temperature at $25$C for the interval $I$. Our method results in a robustness which is just $0.02\%$ less than the unconstrained optimal value, while SA gets to a value $3.8\%$ less than the (unconstrained) optima. Note, for this particular problem, since we assume perfect knowledge of occupancy and disturbances, the problem of satisfying the formula, or indeed of even maximizing robustness, can be solved simply with a quadratic program with linear constraints. We use this example to illustrate the applicability of our method, as well as its performance, while adding a word of caution against the naive use of gradient descent for robustness maximization, even while the gradient of robustness exists \textit{almost everywhere}. In the following example, we take a specification which cannot be trivially turned into a quadratic-program without adding tighter constraints than the specification asks for (or binary variables).


\begin{figure}[t]
\centering
\includegraphics[width=0.49\textwidth]{figures/ZoneTemp_scissored}
\caption{Zone temperatures. The green rectangle shows the comfortable temperature limit of 22-28 C, applicable during time steps 10-19 (when the building is occupied).}
\label{fig:ZoneTemp}
\end{figure}

%1. Single zone building model from ...

%2. Specification for comfort when occupied.

%3. 24 hour look ahead, given disturbances and occupancy. Initial guess (with negative robustness) via solving an LP.

%4. Can be applied in a receding horizon manner. For the given setting, could very well be solved using a linear program asking for temperature between 22-28C for time steps 10 to 19, but we use our method to illustrate how robustness based control can be used to satisfy a specification. The next example (Autonomous ATC) shows control with a specification cannot be trivially translated to a linear program with Polyhedral constraints.

%5. Figure shows room temperature for the 3 methods (other states and disturbances/control in a single figure if necessary)

%6. Table shows robustness of obtained trajectory via the 3 methods. Note, Optimal solution would be temperature of 25C  (robustness of 3) for the occupancy period (if dynamics/constraints would allow it).



\subsection{Autonomous Air Traffic Controller for quad-rotors}
%CaseQuad
\todo[inline]{Start by motivating the example as something more than a dumb system. E.g. "Air traffic control offers many opportunities for automation to allow a more efficient and safer landing patterns. The constraints of air traffic control are complex and contain many safety rules. In this example we express these rules in MTL and demonstrate how the smoothed robustness is used to generate control strategies for safely and robustly landing two quadrotors. The proposed approach outperforms Simulated Annealing and gradient descent on the non-smooth robustness."
	
	Then you give the details below}
1. We take a quad-rotor model with linearized dynamics around hover, similar to those used in \cite{}. The case study involves centralized control of two quad-rotors, with operational objectives given as an MTL specification, and a constrained air-space.

1.b. Give model, constraints, specification. Shrinking horizon (fixing history) approach applicable (cite Vasu paper)

2. With the given specification, standard control approaches involving polyhedral constraints are hard to apply because of the temporal aspect of the eventually operator involved. While the two (if-then) altitude rules in the specification can be coded as polyhedral constraints on the set, it would result in a non-convex constraint set for positions. Similarly, the minimum distance between two quad-rotors can also be moved to the constraints but would result in another non-convex constraint if we choose to do so. In our formulation, the non-convexity remains in the cost-function while the constraints are linear.

3. For simulation purposes, we use obtain 3 initial trajectories (via solving different linear programs) from the given initial state to the terminal (landing) set. These three trajectories, each of which has negative robustness (i.e. does not satisfy the given specification), serve as three different initial solutions to A) Our approach B) SQP using the actual robustness function as the cost, C) Simulated Annealing with the actual robustness function as the cost. This multi-start approach can be used in practice when there is a fast initial trajectory generator available.

3.b. parameters for simulation annealing and citation for it.

4. Fig shows the initial trajectories for both the quadrotors in the given air-space. Fig. shows the three trajectories obtained after applying our control method, with the three initial trajectories as starting points for the optimization, respectively. 

5. Table shows for the three trajectories, initial robustness, robustness for trajectory obtained via the three methods (and the approximate robustness when applicable). In addition, we also tested out simulated annealing with the smooth robustness function in the cost (with the first initial trajectory), resulting in a trajectory with a final cost of blah (approximate robustness of blah).



\begin{figure}[t]
\centering
\includegraphics[width=0.49\textwidth]{figures/QuadInitTrajs_scissored}
\caption{The airspace with the corresponding sets, and initial trajectories for the two quad-rotors. Note, all 3 initial trajectories violate the specification. Here, $p0_{i}^j$ refers to the positions of the $i^{th}$ initial trajectory for the $j^{th}$ quadrotor.}
\label{fig:quad_init}
\end{figure}

\begin{figure}[t]
\centering
\includegraphics[width=0.49\textwidth]{figures/QuadTrajs_scissored}
\caption{ Trajectories obtained via SQP on smooth robustness, with three different initial trajectories acting as initial solutions for the SQP. Note, all 3 trajectories satisfy $\Psi$. Here, $p_{i}^j$ refers to the positions of the $i^{th}$ initial trajectory for the $j^{th}$ quadrotor.}
\label{fig:quad_init}
\end{figure}


\subsection{Discussion}
With a simple 2-state numerical example, and two case studies on systems with more complex dynamics, we show the applicability and consistently good performance of our method. As seen in the two case studies, our method outperforms both simulated annealing and gradient descent (via SQP) on the robustness function. Also, while in the first case study, gradient descent does not find a trajectory that satisfies the specification, in the second case study, simulated annealing cannot find trajectories that satisfy the specification from two of the three initial trajectories (used as initial solutions for optimization). Our method on the other hand, successfully finds a trajectory that satisfies the specification, while resulting in the best robustness value achieved across all examples considered. 
Since all implementations were done in MATLAB, the focus was not real-time applicability of the proposed method. For the first case study, where the dynamics are slow, our method should still be applicable ($~20s$) of execution time (compared to $~5 \text{ mins}$ for SA), for the second case study a MATLAB implementation is infeasible due to the very fast dynamics and sampling times involved. We have shown previously in ??? that control of a real quadrotor with the dynamics in \eqref{eq:quad_dyn} is possible on a low computation power platform while solving multiple quadratic programs at even higher sampling rates ($20Hz$). With this in mind, we expect a C/C++ implementation of SQP (and the smooth robustness function) should allow us to implement our method on such a system. Ongoing work focuses on a general interpreter for formulae to generate corresponding smooth robustness functions, as well as their derivatives.

\addtolength{\textheight}{-12cm}   % This command serves to balance the column lengths
% on the last page of the document manually. It shortens
% the textheight of the last page by a suitable amount.
% This command does not take effect until the next page
% so it should come on the page before the last. Make
% sure that you do not shorten the textheight too much.
\section*{ACKNOWLEDGMENT}

The authors would like to thank 
\bibliographystyle{plain}
\bibliography{fainekos_bibrefs,hscc16,hscc2016,iccps2017}
\end{document}