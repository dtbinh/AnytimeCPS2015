\subsection{Robust Feasibility}

The Robust MPC algorithm, with its constraints and the error set computations for bounding $\tilde{e}_k$ using online reachability, is designed to be robustly feasible to the estimation error, $e_k$, in the estimate of the non-linear system $\hat{x}_k$. This is stated in Theorem \ref{th:robust_feas}, which follows.

\begin{theorem}
\label{th:robust_feas}
If for some time step $k_0$, the Robust MPC optimization, $\mathbb{P}_{k_0}(\hat{z}_{k_0})$ is feasible, then the feedback linearized system, and hence the non-linear system controlled by the Robust MPC of algorithm \textbf{algref} and subject to the disturbances (\textbf{esterror,processnoise}) robutly satisfies the state and input constraints \textbf{eqns} and all subsequent optimizations $\mathbb{P}_{k}(\hat{z}_{k}), \, \forall k>k_0$ are feasible.
\end{theorem}

With this result, the proof of which is in the \textbf{Appendix}, we can state Theorem \ref{th:recursive_feas}.

\begin{theorem}
\label{th:recursive_feas}
If at the initial time step, $\mathbb{P}_0(\hat{z}_0)$ is feasible, then the the feedback linearized, and hence the non-linear system, subject to the disturbances  (\textbf{esterror,processnoise}) satisfies the state and input constraints under the control law of \textbf{algref}, and all subsequent iterations of the algorithm are feasible.
\end{theorem}

\begin{proof}
The Theorem can be proved recursively by applying Theorem \ref{th:robust_feas}. Suppose at time step $k$, the algorithm is feasible and the control input computed by it is $v_k$, by Theorem \ref{th:robust_feas}, $v_k$ is feasible and so $u_k \in U$. Under this control input, the state at time $k+1$, $z_{k+1} \in Z$, and $\mathbb{P}_{k+1}(\hat{x}_{k+1}$ is also feasible, i.e. the algorithm is feasible. Therefore, by induction, the Theorem holds. 
\end{proof}