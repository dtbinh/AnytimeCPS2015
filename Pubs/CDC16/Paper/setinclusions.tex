\section{Set inclusions}
\label{sec:set inclusions}

%Let $\oa{X_{k+1+j|k+1}}$ be the $j+1$-step reach set computed at time $k+1$ as described in ???.
%Recall that $\oa{X_{k+1+j|k+1}}$ is an outer-approximation of the true reach $X_{k+1+j|k+1}$, which is the set that $x_{k+1+j}$ belongs to if it starts at $x_{k|k}$.

Given $x \in X$, define the set $V(x) \defeq \{v \in \Re^a \such u(x) = R^{-1}(x)[b(x)+v] \in U\}$.
We assume that there exist functions $v_i: X \rightarrow \Re$ s.t. for any $x$, $V(x) = \{[v_1,\ldots,v_a]^T \such \ua{v}_i(x) \leq v_i \leq \oa{v_i}(x) \}$.
Because in general $V(x)$ is not a rectangle, we work with inner and outer rectangular approximations of $V(x)$.
Specifically, let $\Xc$ be a subset of $X$.
Define the inner and outer bounding rectangles, respectively
\[\ua{V}(\Xc) \defeq \{v=[v_1,\ldots,v_a]^T \such \max_{x\in \Xc} \ua{v}_i(x)  \leq v_i \leq \min_{x \in \Xc} \oa{v}_i(x) \} \]
\[\oa{V}(\Xc) \defeq \{v=[v_1,\ldots,v_a]^T \such \min_{x\in \Xc} \ua{v}_i(x)  \leq v_i \leq \max_{x \in \Xc} \oa{v}_i(x) \} \]

By construction, we have for any subset $\Xc \subset X$
\begin{equation}
\label{eq:Vbounds}
\ua{V}(\Xc) \leq \cup_{x \in \Xc} V(x) \subset \oa{V}(\Xc)
\end{equation}
If two subsets of $X$ satisfy $\Xc_1 \subset \Xc_2$, then it holds that 
\begin{eqnarray}
\label{eq:V inclusions}
\ua{V}(\Xc_2) \subset \ua{V}(\Xc_1)
\nonumber
\\
\oa{V}(\Xc_1) \subset \oa{V}(\Xc_2)
\end{eqnarray}

We will also need to define containing sets for the state estimation error in $z$ space:
recall that for any $k,j$, 
$\hat{z}_{k+j} = T(\hat{x}_{k+j}) = T(x_{k+j} + e_{k+j}) \approxeq T(x_{k+j}) + M(x_{k+j})e_{k+j} = z_{k+j} + M(x_{k+j})e_{k+j} = z_{k+j} + \te_{{k+j}}$.
Therefore the state estimation error $\te_{k+j}$ lives in 
$\cup_{x\in X_{k+j|k}, e \in E}M(x)e = \cup_{x \in X_{k+j|k}}M(x)E$, 
where $X_{k+j|k}$ is the $j$-step reach set of the nonlinear dynamics computed starting at time $k$.
%
We over-approximate this set by a rectangle as follows: 
if $\te_{k+j}(i)$ is the $i^{th}$ component of vector $\te_{k+j}$, then 
\[\min_{x \in X_{k+j|k}, e \in E}M_i(x)e \leq \te_{k+j}(i) \leq \max_{x \in X_{k+j|k}, e \in E} M_i(x)e\]
where $M_i(x)$ is the $i^{th}$ row of matrix $M(x)$.
Note we can use $\max$ and $\min$ because the sets $X_{k+j|k}$ and $E$ are closed and bounded so the extrema of the continuous function $M_i(x)e$ are achieved on this set.

Since the set $X_{k+j|k}$ is unknown (we only have access to a state estimate, and the exact reachability computation in general is impossible to perform), we over-approximate it by a reachability tool like ??Rtreach, to obtain $\oa{X}_{k+j|k}$.
Then it holds that 
\[\min_{x \in \oa{X}_{k+j|k} e \in E}M_i(x)e \leq \te_{k+j}(i) \leq \max_{x \in \oa{X}_{k+j|k}, e \in E} M_i(x)e\]

In the examples we use one last over-approximation to simplify the optimizations needed to calculate the component-wise bounds, specifically, we use 
\begin{eqnarray}
\label{eq:Etilde}
\sum_{\ell=1}^{n} \min_{x \in \oa{X}_{k+j|k}, e \in E} M_{i\ell}(x)e(\ell)  \leq \te_{k+j}(i) 
\nonumber 
\\
\leq \sum_{\ell=1}^{n} \max_{x \in \oa{X}_{k+j|k}, e \in E} M_{i\ell}(x)e(\ell)
\end{eqnarray}
where $M_{ij}$ is the $(i,j)^{th}$ element of matrix $M$.
Therefore we define the rectangular error set $\tE_{k+j|k}$ to be the set of vectors $e = [e_1,\ldots, e_{\dimZ}]^T$ satisfying \eqref{eq:Etilde}.

While the sets $\tE_{k}$ over-approximate the mapped estimation error, we also need to calculate containing sets for the process noise $\hat{w}$.
Recall that for all $k,j$, 
$\hz_{k+j+1} = z_{k+j+1} + \te_{k+j+1} = Az_{k+j}+Bv_k+w_{k+j} + \te_{k+j+1} =  A(\hz_{k+j} - \te_{k+j}) + Bv_k + w_{k+j} + \te_{k+j+1} = A\hz_{k+j} + Bv_k + w_{k+j} + \te_{k+j+1} - A \te_{k+j} = A\hz_{k+j} + Bv_k + \hw_{k+j+1}$.

Therefore 
\begin{equation}
\label{eq:What}
\hw_{k+j+1} \in \What_{k+j+1|k} \defeq W \oplus \tE_{k+j+1|k} \oplus(-A\tE_{k+j|k})
\end{equation}

We are now ready to state and prove a key lemma regarding the evolution of the state, error and input sets within one MPC optimization problem, and between MPC optimization problems. 
This lemma will be key to proving recursive feasibility of the MPC controller, since it allows us to show that the constraint sets of one problem, at time $k$, are appropriate supersets of the constraint sets of the next problem, at time $k+1$. 

\begin{lemma}
	\label{lem:set inclusions}
	Let $\oa{X}_{k+j|k}$ be the $j$-step outer-approximate reach set computed at time $k$ by a reachability tool as described in ???.
	
	Let $\What_{k+j|k}$ be the set defined in \eqref{eq:What}.
	
	Let $\tE_{k+j|k}$ be the error set computed using \eqref{eq:Etilde} by substituting $E \leftarrow \tE_{k|k}$.
	
	Let $\ua{V}_{k+j|k} = \ua{V}(\oa{X}_{k+j|k})$ and $\oa{V}_{k+j|k} = \oa{V}(\oa{X}_{k+j|k})$ 

Then the following hold:
\begin{enumerate}
	\item $\oa{X}_{k+j+1|k+1} \subseteq \oa{X}_{k+j+1|k}$
	\label{set:X}
	\item $\tE_{k+1+j|k+1} \subseteq \tE_{k+j+1|k}$
	\label{set:tE}
	\item $\What_{k+1+j|k+1} \subseteq {W}_{k+j+1|k}$
	\label{set:What}
	\item $\oa{V}_{k+1+j|k+1} \subseteq \oa{V}_{k+j+1|k}$
	\label{set:oaV}
	\item $\ua{V}_{k+1+j|k+1} \supseteq \ua{V}_{k+j+1|k}$ (note the change in inclusion direction)
	\label{set:uaV}		
\end{enumerate} 
%\begin{enumerate}
%		\item $\oa{X}_{k+j+1|k+1} \subseteq \oa{X}_{k+j+1|k}$
%		\item $\tilde{E}_{k+j+1|k+1} \subseteq \tilde{E}_{k+j+1|k}$
%		\item ${W}_{k+j+1|k+1} \subseteq {W}_{k+j+1|k}$
%		\item $\underline{V}_{k+j+1|k+1} \supseteq \underline{V}_{k+j+1|k}$
%		\item $\bar{V}_{k+j+1|k+1} \subseteq \bar{V}_{k+j+1|k}$
%	\end{enumerate}
\end{lemma} 

\begin{proof}
	
\ref{set:X}) 

\ref{set:tE}) 	By \ref{set:X}) 
 we have that 
 $ \min_{x \in \oa{X}_{k+j+1|k}, e \in E} M_{i\ell}(x)e(\ell) \leq \min_{x \in \oa{X}_{k+1+j|k+1}, e \in E} M_{i\ell}(x)e(\ell) \leq $ and 
 $\max_{x \in \oa{X}_{k+j+1|k}, e \in E} M_{i\ell}(x)e(\ell) \leq \max_{x \in \oa{X}_{k+1+j|k+1}, e \in E} M_{i\ell}(x)e(\ell)$
 which yields the desired result.
 
 \ref{set:What}) This is immediate from the definition \eqref{eq:What} and \ref{set:tE}).
 
 \ref{set:oaV}) and \ref{set:uaV}) These are immediate from \eqref{eq:V inclusions}.
 
	\end{proof}
