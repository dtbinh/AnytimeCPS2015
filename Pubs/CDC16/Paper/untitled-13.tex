\section{Set definitions for the RMPC}

For the RMPC algorithm, we need to define some of the sets which were used in the RMPC formulation.

\subsection{Approximating the bounding sets for the input}
\label{sec:approx input sets}
Given $x \in X$, define the set $V(x) \defeq \{v \in \Re^{\dimV} \such u(x) = R^{-1}(x)[b(x)+v] \in U\}$.
We assume that there exist functions $\ua{v}_i, \oa{v}_i: X \rightarrow \Re$ s.t. for any $x$, $V(x) = \{[v_1,\ldots,v_{\dimV}]^T \such \ua{v}_i(x) \leq v_i \leq \oa{v_i}(x) \}$.
Because in general $V(x)$ is not a rectangle, we work with inner and outer rectangular approximations of $V(x)$.
Specifically, let $\Xc$ be a subset of $X$.
Define the inner and outer bounding rectangles, respectively
\[\ua{V}(\Xc) \defeq \{v=[v_1,\ldots,v_{\dimV}]^T \such \max_{x\in \Xc} \ua{v}_i(x)  \leq v_i \leq \min_{x \in \Xc} \oa{v}_i(x) \} \]
\[\oa{V}(\Xc) \defeq \{v=[v_1,\ldots,v_{\dimV}]^T \such \min_{x\in \Xc} \ua{v}_i(x)  \leq v_i \leq \max_{x \in \Xc} \oa{v}_i(x) \} \]

By construction, we have for any subset $\Xc \subset X$
\begin{equation}
\label{eq:Vbounds}
\ua{V}(\Xc) \leq \cup_{x \in \Xc} V(x) \subset \oa{V}(\Xc)
\end{equation}
If two subsets of $X$ satisfy $\Xc_1 \subset \Xc_2$, then it holds that 
\begin{eqnarray}
\label{eq:V inclusions}
\ua{V}(\Xc_2) \subset \ua{V}(\Xc_1)
\nonumber
\\
\oa{V}(\Xc_1) \subset \oa{V}(\Xc_2)
\end{eqnarray}


\subsection{Approximating the bounding sets for the disturbances}
\label{sec:approx dist}
We will also need to define containing sets for the state estimation error in $z$ space:
Recall, we have $\hat{z}_k = T(\hat{x}_k) = T(x_k+e_k)$. We assume that the magnitude of $e_k$ is small w.r.t the magnitude of $x_k$. With his, we do a Jacobian Linearization as follows

\begin{subequations}
\begin{align}
\hat{z}_k = T(x_k) + \frac{d}{dx_k} T(x_k+e_k)|_{e_k=0}e_k + O(e^2)
\end{align}
\end{subequations}



recall that for any $k,j$, 
$\hat{z}_{k+j} = T(\hat{x}_{k+j}) = T(x_{k+j} + e_{k+j}) \approxeq T(x_{k+j}) + M(x_{k+j})e_{k+j} = z_{k+j} + M(x_{k+j})e_{k+j} = z_{k+j} + \te_{{k+j}}$.
Therefore the state estimation error $\te_{k+j}$ lives in 
$\cup_{x\in X_{k+j|k}, e \in E}M(x)e = \cup_{x \in X_{k+j|k}}M(x)E$, 
where $X_{k+j|k}$ is the $j$-step reach set of the nonlinear dynamics computed starting at time $k$.
%
We over-approximate this set by a rectangle as follows: 
if $\te_{k+j}(i)$ is the $i^{th}$ component of vector $\te_{k+j}$, then 
\[\min_{x \in X_{k+j|k}, e \in E}M_i(x)e \leq \te_{k+j}(i) \leq \max_{x \in X_{k+j|k}, e \in E} M_i(x)e\]
where $M_i(x)$ is the $i^{th}$ row of matrix $M(x)$.
Note we can use $\max$ and $\min$ because the sets $X_{k+j|k}$ and $E$ are closed and bounded so the extrema of the continuous function $M_i(x)e$ are achieved on this set.

Since the set $X_{k+j|k}$ is unknown (we only have access to a state estimate, and the exact reachability computation in general is impossible to perform), we over-approximate it by a reachability tool like ??Rtreach, to obtain $\oa{X}_{k+j|k}$.
Then it holds that 
\[\min_{x \in \oa{X}_{k+j|k} e \in E}M_i(x)e \leq \te_{k+j}(i) \leq \max_{x \in \oa{X}_{k+j|k}, e \in E} M_i(x)e\]

In the examples we use one last over-approximation to simplify the optimizations needed to calculate the component-wise bounds, specifically, we use 
\begin{eqnarray}
\label{eq:Etilde}
\sum_{\ell=1}^{n} \min_{x \in \oa{X}_{k+j|k}, e \in E} M_{i\ell}(x)e(\ell)  \leq \te_{k+j}(i) 
\nonumber 
\\
\leq \sum_{\ell=1}^{n} \max_{x \in \oa{X}_{k+j|k}, e \in E} M_{i\ell}(x)e(\ell)
\end{eqnarray}
where $M_{i\ell}$ is the $(i,\ell)^{th}$ element of matrix $M$.
Therefore we define the rectangular error set $\tE_{k+j|k}$ to be the set of vectors $e = [e_1,\ldots, e_{\dimZ}]^T$ satisfying \eqref{eq:Etilde}.

While the sets $\tE_{k}$ over-approximate the mapped estimation error, we also need to calculate containing sets for the process noise $\hat{w}$.
Recall that for all $k,j$, 
$\hz_{k+j+1} = z_{k+j+1} + \te_{k+j+1} = Az_{k+j}+Bv_k+w_{k+j} + \te_{k+j+1} =  A(\hz_{k+j} - \te_{k+j}) + Bv_k + w_{k+j} + \te_{k+j+1} = A\hz_{k+j} + Bv_k + w_{k+j} + \te_{k+j+1} - A \te_{k+j} = A\hz_{k+j} + Bv_k + \hw_{k+j+1}$.

Therefore 
\begin{equation}
\label{eq:What}
\hw_{k+j+1} \in \What_{k+j+1|k} \defeq W \oplus \tE_{k+j+1|k} \oplus(-A\tE_{k+j|k})
\end{equation}