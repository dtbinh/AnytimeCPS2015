\subsection{State and Input Constraints for the Robust MPC}
\label{sec:Constraints}
The state and input constraints for the RMPC are defined as follows:

\textit{The state constraints $\nomZset{k+j}{k}$:}
The tightened state constraints are functions of the error sets $\tE_{k+j|k}$ and disturbance sets $What_{k+j|k}$, and defined $\forall\,j=0,\dotsc,N$
{\small{
\begin{equation} 
\label{eq:Set_constraints}
\nomZset{k+j}{k} = Z \ominus_{i=0}^{j-1}(L_i \What_{k+(j-i)|k})\ominus (-\tE_{k+j|k})
\end{equation}
}}
(Recall $Z$ is a subset of $T(X)$, $\What_{k+j|k}$ and $\tE_{k+j|k}$ are formally defined in Sec. \ref{sec:set definitions}).
The state transition matrix $L_j$, $\forall j=0,\dotsc,N$   is defined as $L_0 = \mathbb{I}, L_{j+1} = (A+BK)L_j $.
The intuition behind this construction was given at the start of this section.

\textit{The input constraints $V_{k+j|k}$:}
$\forall j=0,...,N-1$
\begin{equation} 
\label{eq:input_constraints}
V_{k+j|k} = \ua{V}_{k+j|k} \ominus_{i=0}^{j-1}KL_i\What_{k+(j-i)|k} 
\end{equation}
where $\ua{V}_{k+j|k} $ is an inner-approximation of the set of admissible inputs $v$ at prediction step $j+k|k$, as defined in Sec. \ref{sec:approx input sets}.
The intuition behind this construction is similar to that of $\nomZset{k+j}{j}$: given the inner approximation $\ua{V}_{k|k} $, it is further shrunk at each prediction step $j$ by propagating forward the noise $\hw_k$ through the dynamics, and shaped according to the stabilizing feedback law $K$, following \cite{RichardsH05_RMPC}.

\textit{The terminal constraint $P_f$:}
This constrains the extended state $p_k = [\nz_{k}, v_{k-1}]^T$, and is given by 
\begin{equation}
\label{eq:P_f_def}
P_f = C_p\ominus \left\lbrack \begin{matrix} (A+BK)^N \\ K(A+BK)^{N-1}\end{matrix} \right \rbrack \What_{max}
\end{equation}
where $\What_{max} \subset \Re^{\dimZ}$ is a bounding set on the worst-case disturbance (we show how it's computed in Sec. \ref{sec:approx dist}),
and $C_p \subset \Re^{\dimZ}\times \Re^{\dimV}$ is an invariant set of the nominal dynamics subject to the stabilizing controller $\nz \mapsto K\nz$, naturally extended to the extended state $p$:
that is, there exists a feedback control law $p \mapsto \widehat{K}p$, such that $\forall p\in C_p$
\begin{eqnarray}
\label{eq:C_def}
\widehat{A} p + \widehat{B} \widehat{K} p + \widehat{L}_N  [\hw^T, 0^T]^T \, \in C_p,\, \forall \hw \in \What_{max} 
\end{eqnarray}
with $\widehat{A} = \begin{bmatrix} A & 0_{n \times m} \\ 0_{m \times n} & 0_{m \times m}   \end{bmatrix} $,
$\widehat{B} = \begin{bmatrix}  B \\ \mathbb{I}_{m \times m} \end{bmatrix}$, 
$\widehat{K} = \begin{bmatrix}  K & 0_{m \times m}  \end{bmatrix}$,
$\widehat{L} _N = (\widehat{A} + \widehat{B} \widehat{K})^N$.
It is important to note the following:
\begin{itemize}
	\item The set $P_f$ can be computed offline since it depends on $\What_{max}$, $\tE_{max}$ and the global inner approximation for the constraints on $v$, $V_{inner-global}$, all of which can be computed offline.
	\item If $P_f$ is non-empty, then all intermediate sets that appear in \eqref{eq:nom mpc} are also non-empty, since $P_f$ shrinks the state and input sets by the maximum disturbances $\What_{max}$ and $\tE_{max}$.	
	Thus we can tell, before running the system, whether RMPC might be faced with empty constraint sets (and thus infeasible optimizations).
	\item Note that all constraints are linear.x
\end{itemize}


