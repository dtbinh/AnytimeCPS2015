\subsection{State and Input Constraints for the Robust MPC}
\label{sec:Constraints}
The state and input constraints for the Robust MPC are as defined as follows:

\textit{The state constraints $\nomZset{k+j}{k}$:}
The tightened state constraints are functions of the error sets $E_{k+j|k}$ and disturbance sets $W_{k+j|k}$, and defined $\forall\,j=0,\dotsc,N$
{\small{
\begin{equation} 
\label{eq:Set_constraints}
\nomZset{k+j}{k} = Z \ominus_{i=0}^{j-1}(L_iW_{k+(j-i)|k})\ominus (-\tE_{k+j|k})
\end{equation}
}}
(Recall $Z$ is a subset of $T(X)$).
The state transition matrix $L_j$, $\forall j=0,\dotsc,N$   is defined as $L_0 = \mathbb{I}, L_{j+1} = (A+BK)L_j $.
The intuition behind this construction was given at the start of this section.

\textit{The input constraints $V_{k+j|k}$:}
$\forall j=0,...,N-1$
\begin{equation} 
\label{eq:input_constraints}
V_{k+j|k} = \ua{V}_{k+j|k} \ominus_{i=0}^{j-1}KL_i\What_{k+(j-i)|k} 
\end{equation}
where $\ua{V}_{k+j|k} $ is an inner-approximation of the set of admissible inputs $v$ at prediction step $j+k|k$, as defined in Sec. \ref{sec:approx input sets}.
The intuition behind this construction is similar to that of $\nomZset{k+j}{j}$: given the inner approximation $\ua{V}_{k|k} $, this is further shrunk at each prediction step $j$ by propagating forward the noise $\hw_k$ through the dynamics, and shaped according to the stabilizing feedback law $K$, following \cite{RichardsH05_RMPC}.

\textit{The terminal constraint $P_f$:}
This constrains the extended state $p_k = [\nz_{k}, v_{k-1}]^T$, and is given by 
\begin{equation}
\label{eq:P_f_def}
P_f = C_p\ominus \left\lbrack \begin{matrix} (A+BK)^N \\ K(A+BK)^{N-1}\end{matrix} \right \rbrack \What_{max}
\end{equation}
where $C_p \subset \Re^{\dimZ}\times \Re^{\dimV}$ is a control invariant set, inside the set $P_N(\tilde{E}_{max},\What_{max})$ which is computed by concatenating Eq. \ref{eq:Set_constraints} and Eq. \ref{eq:input_constraints} with the worst case disturbance bounds replacing the time varying bounds. $C_p$ is computed as  under the stabilizing controller $\nz \mapsto K\nz$, 
and $\What_{max} \subset \Re^{\dimZ}$ is a bounding set on the worst-case disturbance (we show how it's computed in Sec. \ref{sec:implementing set computations}).

By definition of the robust control invariant set $C_p$, there exists a feedback control law $v_p=\hat{K}p$, such that $\forall p\in C_p$
\begin{subequations}
\begin{align}
\label{eq:C_def}
\hat{A}p &+ \hat{B}Kp+L_N \hat{F}w \, \in C_p,\, \forall w \in W_{max} \\
p \in &P_N(\tilde{E}_{max}, \What_{max})
\end{align}
\end{subequations}

It is important to note the following:
\begin{itemize}
	\item The set $P_f$ can be computed offline since it depends on $W_{max}$, $\tE_{max}$ and the global inner approximation for the constraints on $v$, $V_{inner-global}$, all of which can be computed offline (covered in Sec. \ref{sec:implementing set computations}).
	\item If $P_f$ is non-empty, then all intermediate sets that appear in \eqref{eq:nom mpc} are also non-empty, since $P_f$ shrinks the state and input sets by the maximum disturbances $W_{max}$ and $\tE_{max}$.	
	Thus we can tell, before running the system, whether RMPC might be faced with empty constraint sets (and thus infeasible optimizations).
\end{itemize}
%
%
%\begin{subequations}
%\begin{align}
%p_k &=[\bar{z}_k \, v_{k-1}]' \label{eq:joint_state} \\
%\hat{A} &= \begin{bmatrix} A & 0_{nxm} \\ 0_{mxn} & 0_{mxm}   \end{bmatrix} \label{eq:A_hat} \\
%\hat{B} &= \begin{bmatrix}  B \\ \mathbb{I}_{mxm} \end{bmatrix} \label{eq:B_hat} \\
%\hat{K} &= \begin{bmatrix}  K & 0_{mxm}  \end{bmatrix} \label{eq:K_hat} \\
%\hat{L} _i&= (\hat{A}+\hat{B}\hat{K})^i\,\forall i = 0,\dotsc \label{eq:L_hat} \\
%\hat{F} &= \begin{bmatrix} \mathbb{I} \\ 0_{mxn} \end{bmatrix} \\
%\What_{max} &= W \oplus \tilde{E}_{max} \oplus (-A\tilde{E}_{max}) \label{eq:W_max} 
%\end{align}
%\end{subequations}
%
%$\hat{A}$, $\hat{B}$, define the dynamics of the new state $p_{k+1} = \hat{A}p_{k} + \hat{B}v_k$. $\hat{L}$ is defined as $L$ is in Eq.(\ref{eq:L_def}). Also, similar to $K$, which is a stabilizing state feedback matrix for the system defined by $A$ and $B$, $\hat{K}$ acts as a stabilizing matrix for the system given by $\hat{A}$ and $\hat{B}$.
%
%With these definitions, $P_f$ is defined as:
%
%Here, $C_p$ is a robust control invariant set for the worst case disturbances given by $\What_{max}$. Note. $C_p$ is the invariant set inside $P_N(\tilde{E}_{max},\tilde{E}_{max})$, where $\hat{Z}_N(\tilde{E}_{max},\tilde{E}_{max})$ is computed recursively, similar to Eq. \ref{eq:Set_constraints} but with all error sets replaced by $\tilde{E}_{max}$ and correspondingly all disturbance sets by $\What_{max}$, as follows $\forall j=0,\dotsc,N$
%
%
%\begin{subequations}
%\begin{align}
%%{P}_0 &=P \ominus(-\hat{F}\tilde{E}_{max}) \\
%%\hat{Z}_j(\tilde{E}_{max},\tilde{E}_{max}) &= \hat{Z}_j(\tilde{E}_{max},\tilde{E}_{max})\ominus \hat{L}_{j}\What_{max} \\
%P_{j} &= P \ominus_{i=0}^{j-1}(\hat{L}_i\hat{F}\What_{max})\ominus (-\hat{F}\tilde{E}_{max})
%\end{align}
%\end{subequations}
%
%Where $P=Z\times V_{inner-global}$. Note here, these constraints are similar to the form of the state constraints and the input constraints outlined earlier in the section. They are actually the state and input constraints concatenated for the corresponding time steps, with the matrices $\hat{A}$, $\hat{B}$, $\hat{K}$, $\hat{L}_i$ and $\hat{F}$ constructed as shown above such that this fact holds and the definition of the terminal constraint is consistent with the other constraints for steps $j=0,\dotsc,N$. $V_{inner-global}$ and $\tilde{E}_{max}$ are described in Sec. ??
% The robust control invariant set inside $P_N(\tilde{E}_{max},\tilde{E}_{max})$, $C_p$ is such that there exists a feedback control law $v=\hat{K}p$, such that $\forall p\in C_p$
%
%
%\begin{subequations}
%\begin{align}
%\label{eq:C_def}
%\hat{A}p &+ \hat{B}Kp+L_N \hat{F}w \, \in C_p,\, \forall w \in W_{max} \\
%p \in &P_N(\tilde{E}_{max}, \tilde{E}_{max})
%\end{align}
%\end{subequations}

