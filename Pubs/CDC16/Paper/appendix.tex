\section*{Appendix}

\subsection{Constraints of successive MPC problems}
\label{sec:inclusions statement}
We are now ready to state and prove a key lemma regarding the evolution of the state, error and input sets between MPC optimization problems. 
This lemma will be key to proving recursive feasibility of the MPC controller, since it allows us to show that the constraint sets of one problem, at time $k$, are appropriate supersets of the constraint sets of the next problem, at time $k+1$. 

\begin{lemma}
	\label{lem:set inclusions}
	Let $\oa{X}_{k+j|k}$ be the $j$-step outer-approximate reach set computed at time $k$ by a reachability tool as described in Sec. \ref{sec:x reach}.
	
	Let $\What_{k+j|k}$ be the set defined in \eqref{eq:What}.
	
	Let $\tE_{k+j|k}$ be the error set computed using \eqref{eq:Etilde} by substituting $E \leftarrow \tE_{k|k}$.
	
	Let $\ua{V}_{k+j|k} = \ua{V}(\oa{X}_{k+j|k})$ and $\oa{V}_{k+j|k} = \oa{V}(\oa{X}_{k+j|k})$ 

Then the following hold for all $k \geq 0, ,j \geq 1$:
\begin{enumerate}
	\item $\oa{X}_{k+1+j|k+1} \subseteq \oa{X}_{k+j+1|k}$
	\label{set:X}
	\item $\tE_{k+1+j|k+1} \subseteq \tE_{k+j+1|k}$
	\label{set:tE}
	\item $\What_{k+1+j|k+1} \subseteq \What_{k+j+1|k}$
	\label{set:What}
	\item $\oa{V}_{k+1+j|k+1} \subseteq \oa{V}_{k+j+1|k}$
	\label{set:oaV}
	\item $\ua{V}_{k+1+j|k+1} \supseteq \ua{V}_{k+j+1|k}$ (note the change in inclusion direction)
	\label{set:uaV}		
\end{enumerate} 
%\begin{enumerate}
%		\item $\oa{X}_{k+j+1|k+1} \subseteq \oa{X}_{k+j+1|k}$
%		\item $\tilde{E}_{k+j+1|k+1} \subseteq \tilde{E}_{k+j+1|k}$
%		\item ${W}_{k+j+1|k+1} \subseteq {W}_{k+j+1|k}$
%		\item $\underline{V}_{k+j+1|k+1} \supseteq \underline{V}_{k+j+1|k}$
%		\item $\bar{V}_{k+j+1|k+1} \subseteq \bar{V}_{k+j+1|k}$
%	\end{enumerate}
\end{lemma} 

\begin{proof}
	
\ref{set:X}) 
Fix an arbitrary $k$. We prove this by induction on $j \geq 1$.

\underline{Base case: $j=1$}. By construction, $\hx_{k+1} \in \RT{\Xset{k}{k}} \oplus E$.
Therefore at time $k+1$, when setting up the problem $\mathbb{P}_{k+1}(\hat{z}_{k+1})$, the algorithm will first compute
$\Xset{k+1}{k+1} = \hx_{k+1} \oplus (-E)  \subset \RT{\Xset{k}{k}} \oplus E \oplus (-E) = \oaXset{k+1}{k}$.
Also 
$\oaXset{k+2}{k+1} = \RT{\Xset{k+1}{k+1}} \oplus E \oplus(-E) \subset  \RT{\oaXset{k+1}{k}} \oplus E \oplus(-E) = \oaXset{k+2}{k}$.

\underline{Induction step: $j > 1$}.
By definition, $\oaXset{k+1+j}{k+1} = \RT{\oaXset{k+1+j-1}{k+1}} \oplus E \oplus (-E) \subset  \RT{\oaXset{k+j}{k}} \oplus E \oplus (-E)$ (by the induction hypothesis). This last set equals $\oaXset{k+j+1}{k}$ by definition.

\ref{set:tE}) 	By \ref{set:X}) 
 we have that 
 $ \min_{x \in \oa{X}_{k+j+1|k}, e \in E} M_{i\ell}(x)e(\ell) \leq \min_{x \in \oa{X}_{k+1+j|k+1}, e \in E} M_{i\ell}(x)e(\ell)$ and that 
 $\max_{x \in \oa{X}_{k+j+1|k}, e \in E} M_{i\ell}(x)e(\ell) \leq \max_{x \in \oa{X}_{k+1+j|k+1}, e \in E} M_{i\ell}(x)e(\ell)$
 which yields the desired result.
 
 \ref{set:What}) This is immediate from the definition \eqref{eq:What} and \ref{set:tE}).
 
 \ref{set:oaV}) and \ref{set:uaV}) These are immediate from \eqref{eq:V inclusions}.
 
	\end{proof}


\subsection{Proof of Theorem \ref{th:robust_feas}}
\label{sec:proof of thm 1}
We will prove the Theorem by recursion by showing that if at time step $k$, the problem $\mathbb{P}_{k}(\hat{z}_k)$ is feasible and the feasible control input $v_k = v^{*}_{k|k}$ is applied, then $v_k$ is admissible (meets the system constraints) and at time $k+1$, $z_{k+1}$ is inside $Z$ and also $\mathbb{P}_{k+1}(\hat{z}_{k+1})$ is feasible for all disturbances. By recursion then, if we have feasibility at step $k=k_0$, we have robust constraint satisfaction and feasibility at time step $k_0+1$ and so on for all $k>k_0$. 

To begin, let $\mathbb{P}_{k}(\hat{z}_k)$ be feasible, then it has a feasible solution $(\lbrace z^{*}_{k+j|k}\rbrace_{j=0}^{N+1}, \, \lbrace v^{*}_{k+j|k}\rbrace_{j=0}^{N} )$ that satisfies all the constraints of the Robust MPC. Now let's construct a feasible candidate solution for $\mathbb{P}_{k+1}(\hat{z}_{k+1})$ at the next time step by shifting the above solution one-step forward. Consider the candidate solution:

\begin{subequations}
\begin{align}
\label{eq:candidate}
\bar{z}_{k+j+1|k+1} &= \bar{z}^{*}_{k+j+1|k} + L_j\hat{w}_{k+1}, \, \forall j =0,\dotsc,N \\
\bar{z}_{k+N+2|k+1}&= A\bar{z}_{k+N+1|k+1} + B\bar{v}_{k+N+1|k+1} \\
\bar{v}_{k+j+1|k+1}&=v^{*}_{k+j+1|k} + KL_j\hat{w}_{k+1}, \, \forall j =0,\dotsc,N-1 \\
\bar{v}_{k+N+1|k+1}&=K\bar{z}_{k+N+1|k+1} 
\end{align}
\end{subequations}

First we will show that the input and state constraints are satisfied $v_k$ and $z_{k+1}$, then prove feasibility of the above candidate solution for $\mathbb{P}_{k+1}(\hat{z}_{k+1})$.

\textit{Validity of the applied input and next state:}
The next state is:
\begin{subequations}
\begin{align}
\label{eq:z_next}
z_{k+1} &= Az_k + Bv_k + w_k = A(\hat{z}_k-\tilde{e}_k)+Bv^{*}_{k|k}+w_k \\
 &= A\hat{z}_k+Bv^{*}_{k|k}-\tilde{e}_{k+1}+(w_k+\tilde{e}_{k+1}-A\tilde{e}_k) \\
\text{Since }& z^{*}_{k|k}=\hat{z}_k \text{ by initial condition of } \mathbb{P}_{k}(\hat{z}_k) \nonumber \\
z_{k+1}&= z^{*}_{k+1|k} - \tilde{e}_{k+1} + \hat{w}_{k+1}
\end{align}
\end{subequations}

By definition of $\nomZset{k+1}{k}$ in Eq. \ref{eq:Set_constraints} and by feasibility of the solution at time $k$,
\begin{equation}
z^{*}_{k+1|k} \in\nomZset{k+1}{k} = Z \ominus (-\tilde{E}_{k+1|k}) \ominus L_0\What_{k+1|k}
\end{equation}

By definition, $\tilde{e}_{k+1} \in \tilde{E}_{k+1|k}$ and $\hat{w}_{k+1} \in  \What_{k+1|k}$. Using this fact, definition of Pontraygin difference and Eq. \ref{eq:z_next} (also remember, $L_0 = \mathbb{I}$), we have that:
\begin{equation}
z_{k+1}\in Z \Rightarrow x_{k+1} \in X \text{ by definition of Z}
\end{equation}

By the feasibility of $v^{*}_{k|k}$ for $\mathbb{P}_{k}(\hat{z}_k)$ and by the definition of $\underline{V}_{k|k}$,
\begin{equation}
v_k = v^{*}_{k|k} \in \underline{V}_{k|k} \Rightarrow u_k \in U
\end{equation}

Hence, if $\mathbb{P}_{k}(\hat{z}_k)$ is feasible, then the applied input at time step $k$ and the resulting next state $z_{k+1}$ (and hence $x_{k+1}$) are feasible under all possible disturbances. 
The next part of the proof will focus on showing that the candidate solution of Eq. \ref{eq:candidate} is indeed feasible for $\mathbb{P}_{k+1}(\hat{z}_{k+1})$ by proving that it meets all the constraints:

\textit{Initial Condition:} Note,
\begin{subequations}
\begin{align}
\hat{z}_{k+1} &= z_{k+1} + \tilde{e}_{k+1} = Az_{k+1} + Bv_k +w_k +\tilde{e}_{k+1} \\
&= A(z_k-\tilde{e}_k) + Bv_k + w_k +\tilde{e}_{k+1} \\
&= A\hat{z}_k + Bv_k + (w_k + \tilde{e}_{k+1} -A\tilde{e}_k)  \\
\hat{z}_{k+1}&=A\hat{z}_k + Bv_k + \hat{w}_{k+1}
\end{align}
\end{subequations}
Also by the construction of the candidate solution,
\begin{subequations}
\begin{align}
\bar{z}_{k+1|k+1} &= z^{*}_{k+1|k} + L_0 \hat{w}_{k+1} \\
&= Az^{*}_{k|k} + Bv^{*}_{k|k} + \hat{w}_{k+1}
\end{align}
\end{subequations}

Since, $z^{*}_{k|k}=\hat{z}_k$ and $v^{*}_{k|k} = v_k$, by the two equations above, we have
\begin{equation}
\bar{z}_{k+1|k+1} = \hat{z}_{k+1}
\end{equation}

Hence, the candidate solution does indeed satisfy the initial condition for $\mathbb{P}_{k+1}(\hat{z}_{k+1})$. Next we show that the candidate solution satisfies the nominal dynamics:

\textit{ Nominal Dynamics:} For $0\leq J<N$,we have:
\begin{subequations}
\begin{align}
&\bar{z}_{k+j+2|k+1} = z^{*}_{k+j+2|k} + L_{j+1}\hat{w}_{k+1} \\
&= Az^{*}_{k+j+1}+Bv^{*}_{k+j+1|k} + L_{j+1}\hat{w}_{k+1} \\
& \text{By the construction of the candidate solution} \nonumber \\
& = A(\bar{z}_{k+j+1|k+1}-L_j\hat{w}_{k+1}) + B(\bar{v}_{k+j+1|k+1} - KL_j \hat{w}_{k+1}) \nonumber \\
&\, + L_{j+1}\hat{w}_{k+1} \\
& = A\bar{z}_{k+j+1|k+1} + B\bar{v}_{k+j+1|k+1} -(A+BK)L_j\hat{w}_{k+1} \nonumber \\
&\,+ L_{j+1}\hat{w}_{k+1} \\
&\text{Using the definition of $L_{j+1}$ in Eq. \ref{eq:L_def},} \nonumber \\
& = A\bar{z}_{k+j+1|k+1} + B\bar{v}_{k+j+1|k+1} - L_{j+1}\hat{w}_{k+1} + L_{j+1}\hat{w}_{k+1} \\
& = A\bar{z}_{k+j+1|k+1} + B\bar{v}_{k+j+1|k+1}
\end{align}
\end{subequations}

For $j=N$, by construction $\bar{z}_{k+N+2|k+1} = A\bar{z}_{k+N+1|k+1} + B\bar{v}_{k+N+1|k+1}$. Hence, the candidate solution does indeed satisfy the nominal dynamics.

\textit{State Constraints:} To show feasibility of the candidate solution w.r.t the state constraints, we need to show that $\bar{z}_{(k+1)+j|k+1}\in \nomZset{k+1+j}{k+1}\, \forall j=0,\dotsc,N$. Re-writing Eq.\ref{eq:Set_constraints} for $\mathbb{P}_{k}(\hat{z}_k)$ for $j=0,\dotsc,N-1$, we have:

\begin{subequations}
\label{eq:redef_Zj}
\begin{align}
\nomZset{k+j+1}{k} &= Z \ominus_{i=0}^{j}L_i\What_{k+j+1-i|k} \ominus (-\tilde{E}_{k+j+1|k}) \\
 &= Z \ominus L_j \What_{k+1|k} \ominus_{i=1}^{j}L_i\What_{k+j+1-i|k} \ominus (-\tilde{E}_{k+j+1|k})\\
 &= Z \ominus L_j \What_{k+1|k} \ominus_{i=0}^{j-1}L_i\What_{k+j-i|k} \ominus (-\tilde{E}_{k+j+1|k})
\end{align}
\end{subequations}


Also, let us write the state constraints for all $j=0,\dotsc,N$ for the problem at time $k+1$, i.e. for $\mathbb{P}_{k+1}(\hat{z}_{k+1})$:
\begin{equation}
\label{eq:redef_Zjp1}
\nomZset{(k+1)+j}{k+1} = Z \ominus_{i=0}^{j-1}L_i\What_{k+j-i|k+1} \ominus (-\tilde{E}_{k+1+j|k+1}) \\
\end{equation}

Remember, by construction of the candidate, we have $\bar{z}_{k+j+1|k+1} = z^{*}_{k+j+1|k} + L_j\hat{w}_{k+1}$.
Also by feasibility of the algorithm at time $k$, we have $z^{*}_{k+j+1|k}\in \nomZset{k+j+1}{k}$, and by definition, $L_j\hat{w}_{k+1} \in L_j\What_{k+1|k}$. Therefore, by the definition of Pontraygin difference, and Eq. \ref{eq:redef_Zj}, we have $\forall j=0,\dotsc,N-1$,
\begin{subequations}
\begin{align}
&\bar{z}_{(k+1)+j|k+1} \in Z \ominus_{i=0}^{j-1}L_i\What_{k+j-i|k} \ominus (-\tilde{E}_{k+j+1|k}) \\
&\text{Using points 2) and 3) from Lemma \ref{lem:one} } \nonumber \\ 
&Z \ominus_{i=0}^{j-1}L_i\What_{k+j-i|k} \ominus (-\tilde{E}_{k+j+1|k}) \subseteq Z \ominus_{i=0}^{j-1}L_i\What_{k+j-i|k+1}  \nonumber \\
& \ominus (-\tilde{E}_{k+j+1|k+1}) \\
&\text{ And using Eq. \ref{eq:redef_Zjp1}, this imples } \nonumber \\
&\bar{z}_{(k+1)+j|k+1} \in \nomZset{k+1+j}{k+1}
\end{align} 
\end{subequations}

Therefore, we have $\bar{z}_{k+1+j|k+1} \in \nomZset{k+1+j}{k+1},\,\forall j=0,\dotsc,N-1$. 

Now for $j=N$, $\bar{z}_{k+N+1|k+1} = z^{*}_{k+N+1|k} + L_N\hat{w}_{k+1}$. From the terminal constraint we have $[z^{*}_{k+N+1|k}\, v^{*}_{k+N|k}] \in P_f = C_p \ominus \hat{L}_N\hat{F}\What_{max}$. Since $w_{k+1} \in \What_{max}$, and by the construction of the candidate solution (and using the definition of Pontraygin difference, we have), 

\begin{equation}
\label{eq:CandidateInC}
[\bar{z}_{k+N+1|k+1}\, \bar{v}_{k+N|k+1}]' \in C
\end{equation}

Remember, by definition of the invariant set, $C_p \in P_N(\tilde{E}_{max},\tilde{E}_{max})$, and since by definition of $\tilde{E}_{max}$ and Eq. \ref{eq:Set_constraints}, we have $P_N(\tilde{E}_{max},\tilde{E}_{max}) \subseteq \nomZset{k+1+N}{k+1} \times V_{k+1+N|k+1}$, or $C_p \in  \nomZset{k+1+N}{k+1} \times {V}_{k+1+N|k+1}$. This implies that $\bar{z}_{k+N+1|k+1} \in \nomZset{k+1+N}{k+1}$ and additionally, $v_{k+N|k+1} \in {V}_{k+1+N|k+1}$.
Therefore, the set constraints are met by candidate solution $\forall j=0,\dotsc,N$. 

\textit{Input Constraints:} For the inputs, we show that the candidate solution, $\bar{v}_{k+j+1|k+1}, j=0,\ldots,N-2$, satisfies the input constraints for $\mathbb{P}_{k+1}(\hat{z}_{k+1}) $ by using a similar argument as that used for the state constraints. 
Let us re-write the input constraints for $\mathbb{P}_{k}(\hat{z}_{k})$ for $j=0,\dotsc,N-2$,

\begin{subequations}
\label{eq:V_redef}
\begin{align}
V_{k+j+1|k}&=\underline{V}_{k+j+1|k} \ominus_{i=0}^{j} KL_i\What_{k+j+1-i|k} \\
&=\underline{V}_{k+j+1|k} \ominus KL_jW_{k+1|k} \ominus_{i=1}^{j} KL_i\What_{k+j+1-i|k} \\
&=\underline{V}_{k+j+1|k} \ominus KL_jW_{k+1|k} \ominus_{i=0}^{j-1} KL_i\What_{k+j-i|k}
\end{align}
\end{subequations}

Let us also re-write the input constraints for $\mathbb{P}_{k+1}(\hat{z}_{k+1})$ for $j=0,\dotsc,N-1$,
\begin{equation}
\label{eq:Vjkp1}
V_{k+1+j|k+1}=\underline{V}_{k+j+1|k+1} \ominus_{i=0}^{j-1} KL_i\What_{k+j-i|k+1}
\end{equation}

By construction of the candidate, we have $\bar{v}_{k+1+j|k+1}=v^{*}_{k+j+1|k}+KL_j\hat{w}_{k+1}$. Also by feasibility of the algorithm at time $k$, we have $v^{*}_{k+j+1|k} \in V_{k+j+1|k}$, and by definition, $L_j\hat{w}_{k+1} \in L_j\What_{k+1|k}$. Therefore by definition of the Pontraygin difference and Eq. \ref{eq:V_redef}, we have $\forall j=1,\dotsc,N-1$,
\begin{subequations}
\begin{align}
&\bar{v}_{(k+1)+j|k+1} \in \underline{V}_{k+j+1|k} \ominus_{i=0}^{j-1}L_i\What_{k+j-1|k} \\
&\text{Using points 3) and 4) from Lemma \ref{lem:one}} \nonumber \\
&\underline{V}_{k+j+1|k} \ominus_{i=0}^{j-1}L_i\What_{k+j-1|k} \subseteq \underline{V}_{k+j+1|k+1} \ominus_{i=0}^{j-1}L_i\What_{k+j-1|k+1} \\
&\text{And using Eq. \ref{eq:Vjkp1}, this implies} \nonumber \\
&\bar{v}_{(k+1)+j|k+1} \in V_{k+1+j|k+1}
\end{align}
\end{subequations}

Note,  for $j=N-1$, we have already shown in the proof for the state constraints that by definition of the invariant set $C$, $v_{k+N|k+1} \in {V}_{k+1+N-1|k+1}$ by respecting an even tighter constraint.
For the last input for $j=N$, we have $\bar{v}_{k+1+N|k+1}=K\bar{z}_{k+N+1|k}$, we show that it is inside the (joint) terminal constraint $P_f$, and hence is feasible.

\textit{Terminal Constraints:} Finally, we need to show that $[\bar{z}_{k+N+2} \, \bar{v}_{k+N+1}]' \in P_f$. This can be shown using the construction of the terminal set and the candidate solution. From Equation \ref{eq:candidate}, we have:
\begin{subequations}
\begin{align}
\bar{z}_{k+N+2|k+1}&=A\bar{z}_{k+N+1|k+1} + B\bar{v}_{k+N+1|k} \\
\bar{v}_{k+N+1|k+1}&=K\bar{z}_{k+N+1|k+1}
\end{align}
\end{subequations}

Concatenate these two into $p_{k+N+2|k+1} = [\bar{z}_{k+N+2|k+1}\, \bar{v}_{k+N+1|k+1}]'$. Also $p_{k+N+1|k+1} = [\bar{z}_{k+N+1} \,\bar{v}_{k+N}]'$ was in $C_p$ as shown previously (Eq. \ref{eq:CandidateInC}). Therefore, by definition of the invariant set $C_p$ (Equation \ref{eq:C_def}), we have that $p_{k+N+2|k+1} + \hat{L}_N \hat{F} w_{k+1|k}\in C_p$ for all $w_{k+1|k}\in \What_{k+1|k} \subseteq \What_{max}$. Therefore by definition of Pontraygin difference, $p_{k+N+2|k+1} \in C_p \ominus \hat{L}_N\hat{F}\What_{max} = P_f$. Therefore the terminal constraint is also met.

With this, we have the proof for Theorem 1 as we have shown that feasible solution at time step $k$ for $\mathbb{P}_{k}(\hat{z}_{k}) $ implies that the applied input $v_k$ is feasible, the next state $z_{k+1} \in Z$ and the problem $\mathbb{P}_{k+1}(\hat{z}_{k+1}) $ is feasible at time $k+1$, and hence  $\mathbb{P}_{k+2}(\hat{z}_{k+2}) $ is feasible for time step $k+2$ and so on. $\blacksquare$






