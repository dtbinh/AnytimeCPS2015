\section{Problem Formulation} 
\label{sec:formulation}
A common method for control of nonlinear systems is Feedback linearization \cite{khalil}. 
Briefly, in feedback linearization, one applies the feedback law $u(x,v) = R(x)^{-1}(-b(x)+v)$ to \eqref{eq:generic NLMPC}, so that the resulting dynamics, expressed in terms of the transformed state $z = T(x)$, are linear time-invariant:
\begin{equation}
\label{eq:LTI_fb_lin}
S_{fl}: \dot{z} = A_cz + B_cv
\end{equation}
By using the remaining control authority in $v$ to control $S_{fl}$, we can effectively control the non-linear system for, say, stability or reference tracking.
$T$ is a diffeomorphism \cite{khalil} over a domain $D \subset X$.
%Without loss of generality, we assume that the system  has a relative degree $\rho=\dimX$ for SISO systems, and similarly a vector relative degree $\lbrace \rho_i \rbrace_{i=1}^{\dimU}\, \text{s.t.} \sum_i \rho_i = \dimU$ for MIMO systems (assuming number of outputs is the same as the number of inputs, $\dimU$).
The original and transformed states, $x$ and $z$, have the same dimension, as do $u$ and $v$, i.e. $\dimX = \dimZ$ and $\dimU = \dimV$.
Because we are controlling the feedback linearized system, we must find constraint sets $Z$ and $V$ for the state $z$ and input $v$, respectively, such that $(z,v) \in Z\times V \implies (\iT(z), u(\iT(z), v)) \in X \times U$.
This is done in Sec. \ref{sec:transforming x to z}.
%Without loss of generality we assume that the system \eqref{eq:generic NLMPC} has no zero dynamics and all states are controllable \cite{khalil}. 
We assume that the system \eqref{eq:generic NLMPC} has no zero dynamics \cite{khalil} and all states are controllable. In case there are zero dynamics, then our approach is applicable to the controllable subset of the states as long as the span of the rows of $G(x)$ is involutive \cite{khalil}.



For feedback linearizing \eqref{eq:generic NLMPC} and for controlling \eqref{eq:generic NLMPC}, only a periodic state estimate $\hx$ of $x$ is available.
This estimate is available periodically every $\tau$ time units, so we may write $\hx_{k} \defeq \hx(k\tau) = x_k + e_k$, where $x_k$ and $e_k$ are sampled state and error respectively.
We assume that $e_k$ is in a bounded set $E$ for all $k$.
This implies that the feedback linearized system can be represented in discrete-time: $z_{k+1} = Az_k + B z_k$.

The corresponding $z$-space estimate $\hz_{k}$ is given by $\hz_{k} = T(\hx_{k})$.
In general the $z$-space error $\tilde{e}_k \defeq T(\hx_k) - T(x_k)$ is bounded for every $k$ but does not necessarily lie in $E$.
Let $\tE_{k}$ be the set containing $\tilde{e}_k$: in Sec. \ref{sec:approx dist} we show how to compute it.

Because the linearizing control operates on the state estimate and not $x_k$, we add a process noise term to the linearized, discrete-time dynamics. 
Our system model is therefore
\begin{equation}
\label{eq:discrete linear dyn}
z_{k+1} = Az_k + Bv_k + w_k
\end{equation}
where the noise term $w_k$ lies in the bounded set $W$ for all $k$.
An estimate of $W$ can be obtained using the techniques of this paper.
The control problem \eqref{eq:generic NLMPC} is therefore replaced by:
\begin{eqnarray}
\label{eq:discrete linear problem}
\min_{\textbf{z},\textbf{v}} &\;& \sum_{k=0}^{\infty}z_k^TQz_k + v_k^TRv_k \\
\text{s.t. } z_{k+1} &=& Az_k + Bv_k + w_k\nonumber \\
z_k&\in& Z,  v_k \in V,w_k \in W  \nonumber
\end{eqnarray}
(In Thm. \ref{thm:stability}, we show that minimizing this cost function implies stability of the system).

It is easy to derive the dynamics of the state estimate $\hz_{k}$: 
\begin{eqnarray}
\label{eq:dynamics_estimate}
\hat{z}_{k+1} &=& z_{k+1} + \tilde{e}_{k+1} \\
&=&Az_k + Bv_k + w_k + \tilde{e}_{k+1}  \nonumber\\
&=&A\hz_k + Bv_k + (w_k+ \tilde{e}_{k+1} -A \tilde{e}_{k}) \nonumber\\
&=&A\hz_k + Bv_k + \hw_{k+1} \nonumber
\end{eqnarray}
where $\hw_{k+1} = w_k+ \tilde{e}_{k+1} -A \tilde{e}_{k}$, and lies in the set $\What_{k+1} \defeq W\oplus\tE_{k+1}\oplus(-A\tE_k)$. 

\begin{exmp}
	Consider the 2D system 
	\begin{equation}
	\label{eq:toy_dynamics}
	\dot{x}_1 = \sin(x_2) , \dot{x}_2 =-x_1^2 + u 
	\end{equation}
	The safe set for $x$ is given as $X = \lbrace |x_1| \leq \pi /2, |x_2| \leq \pi/3 \rbrace$, and the input set is $U = [-2.75, 2.75]$.
	For the measurement $y = h(x) = x_1$, the system can be feedback linearized on the domain $D = \lbrace x | \cos(x_2) \neq 0 \rbrace $, where it has a relative degree of $\rho=2$. 
	The corresponding linearizing feedback input is $u = -\tan(x_2) + (\cos(x_2))v$.
	The feedback linearized system is $\dot{z}_1 = z_2\, ,\dot{z}_2 = v$, where $T$ is given by $z=T((x_1,x_2)) = (x_1, \sin(x_2))$.	
	We can analytically compute the safe set in $z$-space as $Z = T(X) =  \lbrace |z_1| \leq \pi /2, |z_2| \leq 0.8660\rbrace$.
	\exmend
\end{exmp}

For a more complicated $T$, it is not possible to obtain analytical expressions for $Z$. 
The computation of $Z$ in this more general case is addressed in Sec. \ref{sec:transforming x to z}.

\textbf{Notation}.
%Given a subset $\Xc$ of $X$ and a horizon $T \geq 0$, the \emph{reachable set of system $S$ from $\Xc$ in time $T$ subject to inputs in $U$} is the set of states that $S$ visits at time $T$ if it starts somewhere in $\Xc$ and is subjected to inputs from $U$.
%Formally, $\RT{\Xc} \defeq \{y \in X \such \exists x_0 \in \Xc: x \text{ is a trajectory of the system with inputs in }U, x(0)=x_0, x(T)=y\}$.
Given two subsets $A,B$ of $\Re^n$, their \textit{Minkowski sum} is $A\oplus B \defeq \{a+b \such a\in A, b\in B\}$.
Their \emph{Pontryagin difference} is $A\ominus B = \{c \in \Re^n \such c+b \in A\; \forall b \in B\}$.
Given integers $n \leq m$, $[n:m] \defeq \{n,n+1,\ldots,m\}$.

\textbf{Assumption}. 
The approach we use applies when $X, U, E$ and $W$ are arbitrary convex polytopes (i.e. bounded intersections of half-spaces).
For the sake of simplicity, in this paper we assume they are all hyper-rectangles that contain the origin, i.e. sets of the form $[\underline{a}_1, \overline{a}_1] \times \ldots \times  [\underline{a}_n, \overline{a}_n]$, $\underline{a}_i \leq 0 \leq \overline{a}_i\,\forall i$.