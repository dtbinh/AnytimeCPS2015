\section{Introduction}
\label{sec:intro}

In this paper we are concerned with the problem of controlling nonlinear dynamical systems $S$ of the form $\dot{x} = f(x) + G(x)u$ under state and input constraints, and subject to errors in the state estimate.
This problem is formulated as
\begin{eqnarray}
	\label{eq:generic NLMPC}
	\min_{\textbf{x},\textbf{u}} &\;& l(\textbf{x}, \textbf{u}) \\
	\text{s.t. } \dot{x}&=&f(x) + G(x)u \nonumber \\
	x&\in& X, u \in U  \nonumber
\end{eqnarray}
where $l(\textbf{x}, \textbf{u})$ is a cost function whose minimization over the state and input trajectories $\textbf{x}$ and $\textbf{u}$ ensures stability of the system. 
Sets $X \subset \Re^{\dimX}$ and $ U \subset \Re^{\dimU}$ encode constraints on the state (e.g., safety) and the input.
The input $u = u(\hx)$ is a function of a state estimate that in general differs from the true state of the system.

Whereas Model Predictive Control (MPC) is a widely used control technique for solving the above problem for linear systems, its application to nonlinear systems involves the repeated solution of generally non-quadratic, non-convex optimizations.
Various approaches for solving (or approximately solving) the optimizations and their trade-offs are reviewed in \cite{Cannon04_EfficientMPC}.
Another approach is to first \emph{feedback linearize} the system $S$ \cite{khalil}: namely, the applied control $u = u(x,v)$ is designed in such a way that the resulting closed-loop dynamics $S_{fl}$ are now \emph{linear}:
	\[S_{fl}: \dot{z} = Az + Bv\]
The input $v$ to the linearized dynamics can now be computed so as to optimize system performance and ensure stability.
Since some control authority has been sacrificed to linearize the dynamics, the solution to the linear MPC problem is of course sub-optimal relative to the exact solution to the nonlinear MPC problem.
The state $z$ of the linearized system $S_{fl}$ is related to the state $x$ of the nonlinear system via a (system-specific) function $T$: $z=T(x)$.

Previous work on nonlinear MPC with feedback linearization assumed the state $x(t)$ is perfectly known to the controller at any moment in time \cite{SimonLG13_MPC}.
In practice, only a state estimate $\hx(t)$ is available, and $\hx(t) \neq x(t)$.
Thus a controller designed to work optimally when operating on the true state $x$ is in general sub-optimal when operating on $\hx$ (any may even lead to instability).
Robust MPC (RMPC) has been investigated as a way of handling state estimation errors, but only for linear systems \cite{RichardsH05_RMPC}.
\todo[inline]{litt search}
Existing work has also assumed that $T$ is the identity, which simplifies the subsequent stability and performance analysis \cite{SimonLG13_MPC}. 
\todo[inline]{litt search}
A non-identity $T$ is problematic when the state is not perfectly known, since the state estimation error $e = \hx-x$ maps to the linearized dynamics via $T$ in non-trivial ways greatly complicating the analysis.
In particular, the error bounds for the state estimate in $z$-space now depend on the current nonlinear state $x$.

In this paper we develop a RMPC solution to the above control problem in the case of input-affine systems, with state estimation errors and non-identity $T$.
We are motivated by the problem of anytime control \cite{PantAMNDM15_Anytime}, in which the receding horizon controller receives a delayed inaccurate state estimate.

One of the complications introduced by feedback linearization is that the bounds on the input ($u \in U$) may become non-convex state-dependent bounds on the input $v$ to $S_{fl}$: 
$V = \{\ua{v}(x, U) \leq v \leq \oa{v}(x,U)\}$.
Following \cite{SimonLG13_MPC}, we use forward reachability calculations in $X$ to provide inner convex approximations to the input set $V$.
Here too, a non-identity $T$ increases the computational burden since the reachability calculation must happen on the non-linear system (with an identity $T$, it can happen on the linear system).
We propose two ways for handling this: one is by running an inaccurate but fast nonlinear reachability tool over short horizons, and the other is by running reachability on the linear system and inverting the mapping $T$ in a conservative manner.

The paper is organized as follows,??

