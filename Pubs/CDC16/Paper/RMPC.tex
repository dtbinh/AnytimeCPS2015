\section{Robust MPC for the feedback linearized system}
\label{sec:feedback lin rmpc}

Following \cite{RichardsH05_RMPC}, \cite{PantAMNDM15_Anytime}, we formulate a Robust MPC (RMPC) controller of \eqref{eq:discrete linear problem} via \emph{constraint restriction}.
We outline the idea before providing the technical details.
The key idea is to move the effects of estimation error $\tilde{e}_k$ and  process noise $w_k$ (the `disturbances') to the constraints, and work with the nominal  (i.e., disturbance-free) dynamics: $\nz_{k+1} = A\nz_{k}+Bv_k$, $\nz_{0} = \hat{z}_0$.
Because we would be optimizing over disturbance-free states, we must account for the noise in the constraints. 
Specifically, rather than require the next (nominal) state $\nz_{k+1}$ to be in $Z$, we require it to be in the shrunk set $Z \ominus \What_{k+1|k} \ominus \tE_{k+1|k}$: 
by definition of Pontryagin difference, this implies that whatever the actual value of the noise $\hw_{k+1} \in \What_{k+1|k}$ and of the estimation error $\tilde{e}_{k+1} \in \tE_{k+1|k}$, the actual state $z_{k+1}$ will be in $Z$. 
This is repeated over the entire MPC prediction horizon $j=1,\ldots,N$, with further shrinking at every step.
For further steps ($j>1$), the process noise $\hw_{k+j|k}$ is propagated through the dynamics, so the shrinking term $\What$ is shaped by a stabilizing feedback controller $\nz \mapsto K\nz$.
At the final step ($j=N+1$), a terminal constraint is derived using the worst case estimation error set $\tE_{max}$ and a global inner approximation for the input constraints, $V_{inner-global}$. 

Through this successive constraint tightening we ensure robust safety and feasibility of the feedback linearized system (and hence of the non-linear system). 
Since we use just the nominal dynamics, and show that the tightened constraints are linear in the state and inputs, we still solve a Quadratic Program (QP) for the RMPC optimization.
The difficulty of applying RMPC in our setting is that the amounts by which the various sets are shrunk vary with time and are (nonlinear) state-dependent, and involve set computations with the non-convexity preserving mapping $T$.
One of our contributions in this paper is to establish recursive feasibility of RMPC with time-varying constraint sets.

The RMPC optimization $\Pk{k}$ for solving \eqref{eq:discrete linear problem} is:
\begin{subequations} 
\label{eq:nom mpc}
\begin{align}
J^{*}(\nz_{k}) &= \min_{\mathbf{\nz},\mathbf{u}} \sum_{j=0}^{N}\lbrace \nz_{k+j|k}^{T}Q \nz_{k+j|k} + {v}_{k+j|k}^{T}R{v}_{k+j|k}\rbrace \nonumber \\ 
                    &\quad  +  \nz_{k+N+1|k}^T Q_f \nz_{k+N+1|k}  \label{eq:cost} \\
\nz_{k|k}       &= \hat{z}_{k} \label{eq:init_cond}\\
\nz_{k+j+1|k} &=A\nz_{k+j|k} + Bv_{k+j|k} , j=0,\ldots,N\label{eq:nom_dyn} \\
\nz_{k+j|k}     & \in \nomZset{k+j}{k} , \; j=0,\ldots,N \label{eq:states_con}\\
v_{k+j|k}        & \in  V_{k+j|k} , \;j=0,\ldots,N-1 \label{eq:input_con} \\
p_{N+1}               &= \lbrack z_{k+N+1|k} , v_{k+N|k} \rbrack^{T}  \in P_f \label{eq:joint_term} 
	\end{align}
\end{subequations}

Here, $\nz$ is the state of the nominal feedback linearized system.
The cost and constraints of the optimization are explained below:
\begin{itemize}
\item Eq. \eqref{eq:cost} shows a cost quadratic in $\nz$ and $v$, where as usual $Q$ is positive definite and $R$ is positive semi-definite. 
In the terminal cost term, $Q_f$ is the solution of the Lyapunov equation $Q_f-(A+BK)^{T}Q_f(A+BK) = Q+K^{T}RK$.
This choice guarantees that the terminal cost equals the infinite horizon cost under a linear feedback control $\nz \mapsto K\nz$ \cite{CannonK15MPC}.

\item Eq. \eqref{eq:init_cond} initializes the nominal state with the current state estimate.

\item Eq. \eqref{eq:nom_dyn} gives the nominal dynamics of the discretized linearized system.

\item Eq. \eqref{eq:states_con} tightens the admissible set of the nominal state by a sequence of shrinking sets.

\item Eq. \eqref{eq:input_con} constrains $v_{k+j|k}$ such that the corresponding $u(x,v)$ is admissible, and the RMPC is recursively feasible.

\item Eq. \eqref{eq:joint_term} constrains the final input and nominal state to be within a terminal set $P_f$.

\end{itemize}

The details of how these sets' definitions and computations are given in Sec. \ref{sec:set definitions}.

\subsection{State and Input Constraints for the Robust MPC}

The state and input constraints for the Robust MPC are as defined as follows:

\textit{The state constraints $Z_{j|k}$:}
The tightened state constraints are functions of the error sets $E_{k+j|k}$, and defined $\forall\,j=0,\dotsc,N$
\small{
\begin{subequations} \label{eq:Set_constraints}
\begin{align}
Z_{0|k}(\tilde{E}_{k|k}) &=Z\ominus(-\tilde{E}_{k|k}) \\
Z_{j+1|k}(\tilde{E}_{k+j+1|k},\tilde{E}_{k+j|k}) &= Z_{j|k}(\tilde{E}_{k+j+1|k},\tilde{E}_{k+j|k})\ominus L_{j}W_{k+1|k}
\end{align}
\end{subequations}}

Here, the matrix $L_j$, $\forall j=0,\dotsc,n$   is defined as
\begin{subequations} \label{eq:L_def}
\begin{align}
L_0&=\mathbb{I} \\
L_{j+1} &= (A+BK)L_j
\end{align}
\end{subequations}

\textit{The input constraints $V_{j|k}$:}
The input constraints are given, $\forall j=0,...,N-1$, as:

\begin{subequations}
\begin{align}
V_{0|k} = \underline{V}_{k|k} \\
V_{j+1|k} = \underline{V}_{k+j+1|k} \ominus KL_jW_{k+1|k}
\end{align}
\end{subequations}

\textit{The terminal constraint $Z_f$:}
For the terminal constraint, which is a joint constraint on both the input and the nominal state, let us first define a few terms.
\begin{subequations}
\begin{align}
p_k &=[z_k \, v_{k-1}]' \label{eq:joint_state} \\
\hat{A} &= \begin{bmatrix} A & 0_{nxm} \\ 0_{mxn} & 0_{mxm}   \end{bmatrix} \label{eq:A_hat} \\
\hat{B} &= \begin{bmatrix}  B \\ \mathbb{I}_{mxm} \end{bmatrix} \label{eq:B_hat} \\
\hat{K} &= \begin{bmatrix}  K & 0_{mxm}  \end{bmatrix} \label{eq:K_hat} \\
\hat{L} &= \hat{A}+\hat{B}\hat{K} \label{eq:L_hat} \\
\hat{F} &= \begin{bmatrix} \mathbb{I} \\ 0_{mxn} \end{bmatrix} \\
W_{max} &= W \oplus \tilde{E}_{max} \oplus (-A\tilde{E}_{max}) \label{eq:W_max} 
\end{align}
\end{subequations}

Here, $p_k$ is a state formed by concatenating the state at time $k$ and the input at time $k-1$ (Eq.(\ref{eq:joint_state})). $\hat{A}$, $\hat{B}$, define the dynamics of the new state $p_{k+1} = \hat{A}p_{k} + \hat{B}v_k$. $\hat{L}$ is defined as $L$ is in Eq.(\ref{eq:L_def}). Also, similar to $K$, which is a stabilizing state feedback matrix for the system defined by $A$ and $B$, $\hat{K}$ acts as a stabilizing matrix for the system given by $\hat{A}$ and $\hat{B}$.

With these definitions, $Z_f$ is defined as:
\begin{equation}
Z_f = C\ominus \hat{L}_N\hat{F}{W}_{max}
\end{equation}

Here, $C$ is a robust control invariant set for the worst case disturbances given by $W_{max}$. Note. $C$ is the invariant set inside $\hat{Z}_N(\tilde{E}_{max},\tilde{E}_{max})$, where $\hat{Z}_N(\tilde{E}_{max},\tilde{E}_{max})$ is computed recursively, similar to Eq. \ref{eq:Set_constraints} but with all error sets replaced by $\tilde{E}_{max}$ as follows $\forall j=0,\dotsc,N$


\begin{subequations}
\begin{align}
\hat{Z}_0(\tilde{E}_{max}) &=\hat{Z}\ominus(-\tilde{E}_{max}) \\
\hat{Z}_j(\tilde{E}_{max},\tilde{E}_{max}) &= \hat{Z}_j(\tilde{E}_{max},\tilde{E}_{max})\ominus \hat{L}_{j}W_{max}
\end{align}
\end{subequations}

Where $\hat{Z}=Z\times V_{inner-global}$. The robust control invariant set inside $Z_N(\tilde{E}_{max},\tilde{E}_{max})$, $C$ is such that there exists a feedback control law $v=\hat{K}p$, such that $\forall p\in C$


\begin{subequations}
\begin{align}
\hat{A}z &+ \hat{B}Kp+L_N \hat{F}w \, \in C,\, \forall w \in W_{max} \\
p \in &\hat{Z}_N(\tilde{E}_{max}, \tilde{E}_{max})
\end{align}
\end{subequations}

It is worth noting, that this set is computed offline since it depends on the worst case disturbances $W_{max}$, $E_{max}$ and the global inner convex approximation for the constraints on $v$, $V_{inner-global}$ which can be all be computed offline. With this conservative approximation of the shrinking sets, it is obvious that this is indeed invariant for all disturbance sets, $\forall j$, $E_{k+j|k}$ and $W_{k+j|k}$, which both are contained in $E_{max}$, $W_{max}$ respectively. 


\subsection{The Control Algorithm}
\label{sec:the control algo}
We can now describe the algorithm used for solving \eqref{eq:generic NLMPC} by robust receding horizon control of the feedback linearized nominal dynamics.

\begin{algorithm}
	\caption{RMPC via feedback linearization}
\begin{algorithmic}
	\Require System model, $X$, $U$, $E$, $W$ 
	\State	Offline, compute:
	\State \quad Initial safe sets $X_0$ and $Z$ \Comment{Sec. \ref{sec:transforming x to z}}
	\State \quad $\tE_{max}$, $\What_{max}$ \Comment{Sec. \ref{sec:approx dist} }
	\State \quad $C_p$, $P_f$ \Comment{Sec. \ref{sec:Constraints}}
	\State Online: 
	\If{$P_f = \emptyset$}
	\State Quit
	\Else
	\For{$k=1,2,\ldots$}
	\State Compute $\ua{V}_{k+j|k},\,\tE_{k+j|k},\, \What_{k+j|k}$ \Comment{Sec. \ref{sec:approx input sets}}
	\State Compute $\nomZset{k+j}{k},\,V_{k+j|k}$ \Comment{Sec. \ref{sec:Constraints}}
	\State $(v^*_{k|k}, \ldots, v^*_{k+N|k}) = $ Solution of $\Pk{k}$ 
	\State $v_k = v^{*}_{k|k}$
	\State Apply $u_k = R(\hx_{k}^{-1})[b(\hx_{k})+v_k]$ to plant 
	\EndFor	
	\EndIf		
\end{algorithmic}
\label{alg:RMPC}
\end{algorithm}



\subsection{Robust Feasibility and Stability}
We are now ready to state the main result of this paper: namely, that the RMPC of the feedback linearized system \eqref{eq:nom mpc} is feasible at all time steps if it starts out feasible, and that it stabilizes the nonlinear system, for all possible values of the state estimation error and feedback linearization error.

\begin{theorem}[Robust Feasibility]
\label{th:robust_feas}
If at some time step $k_0 \geq 0$, the RMPC optimization $\Pk{k_0}$ is feasible, then all subsequent optimizations  $\Pk{k} k> k_0$ are also feasible.
Moreover, the nonlinear system \eqref{eq:generic NLMPC} controlled by algorithm \ref{alg:RMPC} and subject to the disturbances ($E$, $W$) satisfies its state and input constraints at all times $k \geq k_0$.
\end{theorem}


\begin{theorem}[Stability]
%	\label{thm:stability}
%Given an equilibrium point $x_e \in X_0 \subset \iT(Z)$ of the nonlinear dynamics \eqref{eq:generic NLMPC}, Algorithm \ref{alg:RMPC} stabilizes the nonlinear system to $x_e$.
%\end{theorem}
	\label{thm:stability}
Given an equilibrium point $x_e \in X_0 \subset \iT(Z)$ of the nonlinear dynamics \eqref{eq:generic NLMPC}, Algorithm \ref{alg:RMPC} stabilizes the nonlinear system to an invariant set around $x_e$.
\end{theorem}

\begin{proof}
Let $T$ be the diffeomorphism mapping $x$ to $z$ from feedback linearization.
By a change of variables $z' = z - T(x_e)$, stabilizing the linear dynamics (with state $z'$) to 0 implies stabilizing the nonlinear dynamics to $x_e$.
Recall that $Q$ and $Q_f$ of  \eqref{eq:nom mpc} are positive definite and that $R$ is positive semi-definite,  so that the optimal cost $J^*(\nz_{k})$ is a positive definite function of $\nz_{k}$, and that the terminal weight in \eqref{eq:nom mpc} is equivalent to the infinite horizon cost (by our choice of $Q_f$). 
Finally Thm.  \ref{th:robust_feas} guarantees that the tail of the input sequence computed at $k$ is admissible at time $k+1$. 
Therefore it is a standard result that the optimal cost $J^{*}({\nz}_{k})$ is non-increasing in $k$ and that $0$ is a stable equilibrium for the closed-loop linear system (e.g., see \cite{CannonK15MPC} ). 
Moreover the nominal feedback-linearized system ($\nz$) converges to 0 from anywhere in $Z$. Therefore, the nominal $\bar{x}_{k}$ converges to $x_e$ from anywhere in $X_0 \subset \iT(Z)$. The true state ($x_k$) then converges to the invariant set around $x_e$.
\end{proof}