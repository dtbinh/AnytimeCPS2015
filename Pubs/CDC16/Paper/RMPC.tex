\section{Robust MPC for the feedback linearized system}

Following \cite{RichardsH05_RMPC}, \cite{PantAMNDM15_Anytime}, we formulate a Robust MPC (RMPC) controller of \eqref{eq:discrete linear problem} via \emph{constraint restriction}.
We outline the idea before providing the technical details.
The key idea is to move the effects of estimation error $\tilde{e}_k$ and  process noise $w_k$ (the `disturbances') to the constraints, and work with the nominal  (i.e., disturbance-free) dynamics: $\nz_{k+1} = A\nz_{k}+Bv_k$, $\nz_{0} = z_0$.
Because we would be optimizing over disturbance-free states, we must account for the noise in the constraints. 
Specifically, rather than require the next (nominal) state $\nz_{k+1}$ to be in $Z$, we require it to be in the shrunk set $Z \ominus \What_{k+1|k} \ominus \tE_{k+1|k}$: by definition of Pontryagin difference, this implies that whatever the actual value of the noise $\hw_{k+1} \in \What_{k+1|k}$ and of the estimation error $\tilde{e}_{k+1} \in \tE_{k+1|k}$, the actual state $z_{k+1}$ will be in $Z$. 
This is repeated over the entire MPC prediction horizon $j=1,\ldots,N$, with further shrinking at every step.
For further steps ($j>1$), the process noise $\What_{k+j|k}$ is propagated through the dynamics, so the shrinking term $\What$ is shaped by a stabilizing feedback controller $x \mapsto Kx$.
At the final step ($j=N$), a terminal constraint is derived using the worst case estimation error set $\tE_{max}$ and a convex global inner approximation for the input constraints, $V_{inner-global}$. 
Details of this construction follow below.

Through this successive constraint tightening we ensure robust safety and feasibility of the feedback linearized system (and hence of the non-linear system). 
Since we use just the nominal dynamics, and show that the tightened constraints are linear in the state and inputs, we still solve a Quadratic Program (QP) for the MPC optimization.
The difficulty of applying RMPC in our setting is that the amounts by which the various sets are shrunk varies with time and is (nonlinear) state-dependent, and involves set computations with the non-convexity preserving mapping $T$.
One of our contributions in this paper is to establish recursive feasibility of MPC with time-varying constraint sets.
% This not only allows us to keep a check on the complexity of the MPC optimization, but also allows us to formulate the MPC with a quadratic cost on the input to the feedback linearized system \textbf{[InputPaperRef8]}.

%
%\textbf{Note:} So far we have not found an expression for the bound on $\tilde{e}_k$ given $e_k \in E$. While for the rest of this section we assume we have a polyhedral bound $\tilde{e}_k \in \tilde{E}_k \, \forall k$, we will show how to explicitly compute this bounding set $\tilde{E}_k$ using online reachability in Sec.?? . Since we compute $\tilde{E}_{k+j},\, j\geq 0$ using an online reachability method that depends on the state estimate at time $k$, we represent the set as $\tilde{E}_{k+j|k}$ to explicitly show its dependence on the state estimate at time $k$. Similarly, the computed bound for $\What_{k+1}$, given state estimate at time $k$, is given by $\What_{k+1|k}=W\oplus{E}_{k+1|k}\oplus(-A\tilde{E_{k|k}})$. 


\todo[inline]{mention that if shrunk terminal set is non-empty, then nothing in between is empty. }
The RMPC optimization  $\mathbb{P}_{k_0}(\hat{z}_{k_0})$ for solving \eqref{eq:discrete linear problem} is:
%\begin{eqnarray} 
%	\label{eq:nom mpc}
%	J^{*}(\nz_{k}) &=& \min_{\mathbf{\nz},\mathbf{u}} \sum_{j=0}^{N}\lbrace \nz_{k+j|k}^{T}Q \nz_{k+j|k} + {v}_{k+j|k}^{T}R{v}_{k+j|k}\rbrace \nonumber \\ 
%	&\quad & +  \nz_{k+N+1|k}^T Q_f \nz_{k+N+1|k}  \label{eq:cost} \\
%	\nz_{k|k}       &= &\hat{z}_{k} \label{eq:init_cond}\\
%	\nz_{k+j+1|k} &=&A\nz_{k+j|k} + Bv_{k+j|k} , j=0,\ldots,N\label{eq:nom_dyn} \\
%	\nz_{k+j|k}     & \in& \nomZset{k+j}{k} , \; j=0,\ldots,N \label{eq:states_con}\\
%	v_{k+j|k}        & \in & V_{k+j|k} , \;j=0,\ldots,N-1 \label{eq:input_con} \\
%	p_N               &= &\lbrack z_{k+N+1|k} , v_{k+N|k} \rbrack^{T}  \in P_f \label{eq:joint_term} 
%\end{eqnarray}
\begin{subequations} 
\label{eq:nom mpc}
\begin{align}
J^{*}(\nz_{k}) &= \min_{\mathbf{\nz},\mathbf{u}} \sum_{j=0}^{N}\lbrace \nz_{k+j|k}^{T}Q \nz_{k+j|k} + {v}_{k+j|k}^{T}R{v}_{k+j|k}\rbrace \nonumber \\ 
                    &\quad  +  \nz_{k+N+1|k}^T Q_f \nz_{k+N+1|k}  \label{eq:cost} \\
\nz_{k|k}       &= \hat{z}_{k} \label{eq:init_cond}\\
\nz_{k+j+1|k} &=A\nz_{k+j|k} + Bv_{k+j|k} , j=0,\ldots,N\label{eq:nom_dyn} \\
\nz_{k+j|k}     & \in \nomZset{k+j}{k} , \; j=0,\ldots,N \label{eq:states_con}\\
v_{k+j|k}        & \in  V_{k+j|k} , \;j=0,\ldots,N-1 \label{eq:input_con} \\
p_{N+1}               &= \lbrack z_{k+N+1|k} , v_{k+N|k} \rbrack^{T}  \in P_f \label{eq:joint_term} 
	\end{align}
\end{subequations}

Here, $\nz$ is the state of the nominal linearized system
The cost and constraints of the optimization are explained below:
\begin{itemize}
\item Eq. \eqref{eq:cost} shows a cost quadratic in $\nz$ and $v$, where as usual $Q$ is positive definite and R is positive semi-definite. 
In the terminal cost term, $Q_f$ is the solution of the Lyapunov equation $Q_f-(A+BK)^{T}Q_f(A+BK) = Q+K^{T}RK$.
This choice guarantees that the terminal cost equals the infinite horizon cost under a linear feedback control $\nz \mapsto K\nz$ \cite{CannonK15MPC}.

\item Eq. \eqref{eq:nom_dyn} gives the nominal dynamics of the discretized linearized system.

\item Eq. \eqref{eq:init_cond} initializes the nominal state with the current state estimate.

\item Eq. \eqref{eq:states_con} tightens the admissible set of the nominal state by a sequence of shrinking sets.

\item Eq. \eqref{eq:input_con} constrains $v_{k+j|k}$ such that the corresponding $u(x,v)$ is admissible, and the MPC is recursively feasible.

\item Eq. \eqref{eq:joint_term} constrains the final input and nominal state to be within a terminal set $P_f$.

\end{itemize}

\subsection{State and Input Constraints for the Robust MPC}
\label{sec:Constraints}
The state and input constraints for the RMPC are defined as follows:

\textit{The state constraints $\nomZset{k+j}{k}$:}
The tightened state constraints are functions of the error sets $\tE_{k+j|k}$ and disturbance sets $What_{k+j|k}$, and defined $\forall\,j=0,\dotsc,N$
{\small{
\begin{equation} 
\label{eq:Set_constraints}
\nomZset{k+j}{k} = Z \ominus_{i=0}^{j-1}(L_i \What_{k+(j-i)|k})\ominus (-\tE_{k+j|k})
\end{equation}
}}
(Recall $Z$ is a subset of $T(X)$, $\What_{k+j|k}$ and $\tE_{k+j|k}$ are formally defined in Sec. \ref{sec:set definitions}).
The state transition matrix $L_j$, $\forall j=0,\dotsc,N$   is defined as $L_0 = \mathbb{I}, L_{j+1} = (A+BK)L_j $.
The intuition behind this construction was given at the start of this section.

\textit{The input constraints $V_{k+j|k}$:}
$\forall j=0,...,N-1$
\begin{equation} 
\label{eq:input_constraints}
V_{k+j|k} = \ua{V}_{k+j|k} \ominus_{i=0}^{j-1}KL_i\What_{k+(j-i)|k} 
\end{equation}
where $\ua{V}_{k+j|k} $ is an inner-approximation of the set of admissible inputs $v$ at prediction step $j+k|k$, as defined in Sec. \ref{sec:approx input sets}.
The intuition behind this construction is similar to that of $\nomZset{k+j}{j}$: given the inner approximation $\ua{V}_{k|k} $, it is further shrunk at each prediction step $j$ by propagating forward the noise $\hw_k$ through the dynamics, and shaped according to the stabilizing feedback law $K$, following \cite{RichardsH05_RMPC}.

\textit{The terminal constraint $P_f$:}
This constrains the extended state $p_k = [\nz_{k}, v_{k-1}]^T$, and is given by 
\begin{equation}
\label{eq:P_f_def}
P_f = C_p\ominus \left\lbrack \begin{matrix} (A+BK)^N \\ K(A+BK)^{N-1}\end{matrix} \right \rbrack \What_{max}
\end{equation}
where $\What_{max} \subset \Re^{\dimZ}$ is a bounding set on the worst-case disturbance (we show how it's computed in Sec. \ref{sec:approx dist}),
and $C_p \subset \Re^{\dimZ}\times \Re^{\dimV}$ is an invariant set of the nominal dynamics subject to the stabilizing controller $\nz \mapsto K\nz$, naturally extended to the extended state $p$:
that is, there exists a feedback control law $p \mapsto \widehat{K}p$, such that $\forall p\in C_p$
\begin{eqnarray}
\label{eq:C_def}
\widehat{A} p + \widehat{B} \widehat{K} p + \widehat{L}_N  [\hw^T, 0^T]^T \, \in C_p,\, \forall \hw \in \What_{max} 
\end{eqnarray}
with $\widehat{A} = \begin{bmatrix} A & 0_{n \times m} \\ 0_{m \times n} & 0_{m \times m}   \end{bmatrix} $,
$\widehat{B} = \begin{bmatrix}  B \\ \mathbb{I}_{m \times m} \end{bmatrix}$, 
$\widehat{K} = \begin{bmatrix}  K & 0_{m \times m}  \end{bmatrix}$,
$\widehat{L} _N = (\widehat{A} + \widehat{B} \widehat{K})^N$.
It is important to note the following:
\begin{itemize}
	\item The set $P_f$ can be computed offline since it depends on $\What_{max}$, $\tE_{max}$ and the global inner approximation for the constraints on $v$, $V_{inner-global}$, all of which can be computed offline.
	\item If $P_f$ is non-empty, then all intermediate sets that appear in \eqref{eq:nom mpc} are also non-empty, since $P_f$ shrinks the state and input sets by the maximum disturbances $\What_{max}$ and $\tE_{max}$.	
	Thus we can tell, before running the system, whether RMPC might be faced with empty constraint sets (and thus infeasible optimizations).
	\item Note that all constraints are linear.x
\end{itemize}




\subsection{Robust Feasibility}

The Robust MPC algorithm, with its constraints and the error set computations for bounding $\tilde{e}_k$ using online reachability, is designed to be robustly feasible to the estimation error, $e_k$, in the estimate of the non-linear system $\hat{x}_k$. This is stated in Theorem \ref{th:robust_feas}, which follows.

\begin{theorem}
\label{th:robust_feas}
If for some time step $k_0$, the Robust MPC optimization, $\mathbb{P}_{k_0}(\hat{z}_{k_0})$ is feasible, then the feedback linearized system, and hence the non-linear system controlled by the Robust MPC of algorithm \textbf{algref} and subject to the disturbances (\textbf{esterror,processnoise}) robutly satisfies the state and input constraints \textbf{eqns} and all subsequent optimizations $\mathbb{P}_{k}(\hat{z}_{k}), \, \forall k>k_0$ are feasible.
\end{theorem}

With this result, the proof of which is in the \textbf{Appendix}, we can state Theorem \ref{th:recursive_feas}.

\begin{theorem}
\label{th:recursive_feas}
If at the initial time step, $\mathbb{P}_0(\hat{z}_0)$ is feasible, then the the feedback linearized, and hence the non-linear system, subject to the disturbances  (\textbf{esterror,processnoise}) satisfies the state and input constraints under the control law of \textbf{algref}, and all subsequent iterations of the algorithm are feasible.
\end{theorem}

\begin{proof}
The Theorem can be proved recursively by applying Theorem \ref{th:robust_feas}. Suppose at time step $k$, the algorithm is feasible and the control input computed by it is $v_k$, by Theorem \ref{th:robust_feas}, $v_k$ is feasible and so $u_k \in U$. Under this control input, the state at time $k+1$, $z_{k+1} \in Z$, and $\mathbb{P}_{k+1}(\hat{x}_{k+1})$ is also feasible, i.e. the algorithm is feasible. Therefore, by induction, the Theorem holds. 
\end{proof}


\begin{theorem}[Stability]
%	\label{thm:stability}
%Given an equilibrium point $x_e \in X_0 \subset \iT(Z)$ of the nonlinear dynamics \eqref{eq:generic NLMPC}, Algorithm \ref{alg:RMPC} stabilizes the nonlinear system to $x_e$.
%\end{theorem}
	\label{thm:stability}
Given an equilibrium point $x_e \in X_0 \subset \iT(Z)$ of the nonlinear dynamics \eqref{eq:generic NLMPC}, Algorithm \ref{alg:RMPC} stabilizes the nonlinear system to an invariant set around $x_e$.
\end{theorem}

\begin{proof}
Let $T$ be the diffeomorphism mapping $x$ to $z$ from feedback linearization.
By a change of variables $z' = z - T(x_e)$, stabilizing the linear dynamics (with state $z'$) to 0 implies stabilizing the nonlinear dynamics to $x_e$.
Recall that $Q$ and $Q_f$ of  \eqref{eq:nom mpc} are positive definite and that $R$ is positive semi-definite,  so that the optimal cost $J^*(\nz_{k})$ is a positive definite function of $\nz_{k}$, and that the terminal weight in \eqref{eq:nom mpc} is equivalent to the infinite horizon cost (by our choice of $Q_f$). 
Finally Thm.  \ref{th:robust_feas} guarantees that the tail of the input sequence computed at $k$ is admissible at time $k+1$. 
Therefore it is a standard result that the optimal cost $J^{*}({\nz}_{k})$ is non-increasing in $k$ and that $0$ is a stable equilibrium for the closed-loop linear system (e.g., see \cite{CannonK15MPC} ). 
Moreover the nominal feedback-linearized system ($\nz$) converges to 0 from anywhere in $Z$. Therefore, the nominal $\bar{x}_{k}$ converges to $x_e$ from anywhere in $X_0 \subset \iT(Z)$. The true state ($x_k$) then converges to the invariant set around $x_e$.
\end{proof}