\section{Discussion}

In this paper we develop the first algorithm for robust control of a non-linear system with state and input constraints via feedback linearization and Robust MPC. %which to the best of our knowledge is the first such work involving 
%robust predictive control of a non-linear system using feedback linearization. 
We develop a method to compute linear constraints on the state and inputs of the feedback linearized system such that the non-linear system respects its state and input constraints under bounded state estimation errors. 
%This constraint tightening on the feedback linearized system is done so as to guarantee robust constraint satisfaction and recursive feasibility for the algorithm. It allows results in linear constraints on variables of the Robust MPC optimization, allowing us to robustly control the non-linear system while solving a computationally low cost quadratic program with linear constraints. 
We demonstrate the applicability of our approach on a planar system and a flexible link manipulator example. Results show that the control algorithm stabilizes the systems while ensuring robust constraint satisfaction.% and being feasible at all time instants. 
While we only evaluated our approach for single input systems, the formulation and set computations involved hold as is for multi-input systems as well.


Limitations of the approach mostly have to do with the numerical limitations involved in computing the constraint sets, e.g. in the manipulaor example, the set of states,$X_0$, from which we can control the system, is a strict subset of the set of safe states $X$ of the non-linear system. Similarly in the computation of the control limits and error sets, over-conservatism is potentially a problem.
%Another problematic case can be when rectangular over approximations of the already over-approximated reach sets use to compute online the input and error set bounds result in very conservative bounds on the input and error set, i.e. tending to %$V_{global-%inner}$ and $\tilde{e}_{max}$ respectively. While in our examples, we do not see this happen, it is not difficult to find envision systems and reachable states where this can happen. 


Ongoing work focuses on implementing this approach for evaluation on a $1/10^{th}$ scale autonomous car, running a low power embedded platform. Real-time online reachability \cite{rtreach}, interval arithmetic% for online computation of the set %$\underline{V}_{k+j|k}$, 
and support function based computations for $\bar{Z}_{k|j|k}$ should allow for fast enough computation of the linear constraints for the RMPC optimization. In \cite{PantAMNDM15_Anytime}, we have already shown that CVXGEN $\cite{cvxgen}$ is fast enough to solve Quadratic programs with linear constraints on low-powered embedded platforms at high enough sampling rates to allow for satisfactory control of a real system.


