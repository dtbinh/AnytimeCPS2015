\section{Building Example}
\label{sec:building}
%\begin{table}[t]
%	\begin{center}
%		\caption{{\small Runtimes for \eqref{eq:BldgSpec} for Smooth Operator (SOP) and BluSTL over 100 runs with random initial states. All times are presented as Mean / Std deviation / $95^{th}$ percentile (s)}}
%		\vspace{-5pt}
%		\label{tbl:bldg}
%		\begin{tabular} {|c|c|c|c|}
%			\hline
%			BluSTL in Sat mode &   SOP in Sat mode & BluSTL in Robustness mode & SOP in Robustness mode	 \\ \hline
%			0.0413/0.0457/0.0377 & \textbf{0.0281/0.0272/0.0114} & 0.2530/0.0763/0.2092 & 2.8078/0.6641/2.1094\\ \hline	
%		\end{tabular}	
%	\end{center}
%\end{table}

\begin{table}[t]
	\begin{center}
		\caption{{\small Runtimes for \eqref{eq:BldgSpec} for Smooth Operator (SOP) and BluSTL over 100 runs with random initial states, running in Sat and Robustness (Rob) modes. All times are presented as Mean / Std deviation / $95^{th}$ percentile (s)}.}
		\vspace{-5pt}
		\label{tbl:bldg}
		\begin{tabular} {|c|c|}
			\hline
			\textbf{BluSTL (Sat mode)} &   \textbf{SOP (Sat mode)}
			\\ \hline
			0.0413/0.0457/0.0945 & 0.0281/0.0272/0.0556
			\\ \hline
			 \textbf{BluSTL (Rob. mode)} & \textbf{SOP (Rob. mode)}
			\\ \hline
			 0.2530/0.0763/0.4147 & 2.8078/0.6641/3.7931\\ \hline	
		\end{tabular}	
	\end{center}
	\vspace{-10pt}
\end{table}
Our second example is a Heating, Ventilation and Air Conditioning (HVAC) control of a 4-state model of a single zone in a building. 
Such a model is commonly used in literature for evaluation of predictive control algorithms \cite{Jain2016}. 
The control problem we solve is similar to the example used in \cite{Raman14_MPCSTL}, where the objective is to bring the zone temperature to a comfortable range when the zone is occupied (given predictions on the building occupancy). 
The spec is:
\begin{equation}
\label{eq:BldgSpec}
\Phi= \always_I(\text{ZoneTemp} \in \text{Comfort})
\end{equation}
Here, $I$ is the time interval where the zone is occupied, and $\text{Comfort}$ is the range of temperatures deemed comfortable, namely $[22,28]$. For the control horizon, we consider a 24 hour period (24 steps at 1hr sampling), in which the building is occupied from time steps $10$ to $19$ (i.e. $I=[10,19]$), i.e. a regular workday. 

\textbf{System dynamics.} The single-zone model, discretized at a sampling rate of 1 hour (which is common in building temperature control) is of the form:
\begin{equation}
\label{eq:bldg_dyn}
x_{k+1} = Ax_{k}+Bu_k+B_dd_k
\end{equation}
Here, $A$, $B$ and $B_d$ matrices are from the hamlab ISE model \cite{VanSchijndel2005}. $x \in \mathbb{R}^4$ is the state of the model, the $4^{th}$ element of which is the zone temperature, the others are auxiliary temperatures corresponding to other physical properties of the zone (walls and facade). 
The input to the system, $u \in \mathbb{R}^1$, is the heating/cooling energy. 
$b_d \in \mathbb{R}^3$ are disturbances (due to occupancy, outside temperature, solar radiation). 
We assume these are known a priori, and without loss of generality, set them to zero.

%i.e. robustness (smooth, when applicable) is only in the objective, to be maximized. This allows for a fair comparison between the three methods. 
%With respect to the general control problem of \eqref{eq:general_ctrl}, the limits on the states are $X=[0,50]^4$ and on the inputs $U=[-1000,2000]$.
In satisfaction mode, BluSTL find trajectories that satisfy the specification (with $0$ robustness), as does Smooth Operator (with a mean robustness of $0.14$).

In robustness maximization mode, BluSTL fails to find satisfying trajectories (negative robustness) for each of the 100 runs.
Specifically the mean robustness was -2.8742 ($\sigma = 0.1982$) whereas Smooth Operator results had a mean robustness of 2.9984 ($\sigma = 0.0028$). 
Note, for the given comfort zone of $[22,28]$ Celsius, the upper bound on robustness is $3$, which would be achieved by keeping the temperature constant at 25C during the interval of interest. Also, the reason why BluSTL only finds negative robustness trajectories in this case might be because MILP solvers in general result in local minima (as does SQP).

The runtimes are presented in Table~\ref{tbl:bldg}. Ongoing work focuses on evaluation over other systems as well as detailed comparison to other heuristics (e.g. Simulated Annealing). % speeding up Smooth Operator by reformulating the optimization problem in a more efficient manner.
%\begin{table}[t]
%\small
%\begin{center}
%\caption{{\small Robustness after robustness maximization for $\Phi$ for SR-SQP and BlueSTL over 100 runs with random initial states.}}
%\vspace{-10pt}
%\label{tbl:bldg_robustness}
%\begin{tabular} {|c|c|}
%	\hline
%	 mean/std $\rho$ BS & mean/std $\rho$ SRSQP \\ \hline
% -2.8742/0.1982 & 2.9984/0.0028\\ \hline	
%\end{tabular}	
%\end{center}
%\vspace{-20pt}
%\end{table}
