{\sf Example1, continued}.
The ATC rules can be formalized in Metric Temporal Logic (MTL) as follows.
E.g., Rule \ref{rule:floor ceiling} can be formalized as follows ($\always$ means `Always', $q$ is an aircraft and $q_z$ is its altitude).
\begin{equation}
\label{eq:rule1mtl}
\always( q \in Zone1 \implies q_z \leq \text{Ceiling1} \land q_z \geq \text{Floor1})
\end{equation}
Rule \ref{rule:holding} can be formalized as follows.
\begin{flalign}
\label{eq:rule3mtl}
\always(Busy \implies\eventually_{[t_1,t_2]} (&q \in \text{Holding-6} \, \lor \,q \in \text{Holding-7}) 
\nonumber \\
&\until_{[0,\text{MaxHolding}]} \neg Busy)
\end{flalign}
This says that Always ($\always$), if airport is Busy, then Eventually $(\eventually)$, sometime between times $t_1$ and $t_2$, the plane goes into a holding area.
It stays there $\until$ntil the airport is not ($\neg$) busy, or the timer expires at time MaxHolding. 

By maximizing the robusness of these MTL specifications, the ATC can automatically find landing patterns that leave room for maneuvering in case of emergencies.
\textit{Any unforeseen disturbance smaller than a known bounded size will not violate the rules, and will not lead to an unsafe situation.} 
%\vspace{-20pt} \begin{flushright} $\blacksquare$ \end{flushright}
