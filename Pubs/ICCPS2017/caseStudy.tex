\section{Case study}
\label{sec:case study}
Two numerical examples. Finite horizon control in a single shot, can be applied in a receding/shrinking horizon approach wherever applicable. 


\subsection{HVAC Control of a building for comfort}
1. Single zone building model from ...

2. Specification for comfort when occupied.

3. 24 hour look ahead, given disturbances and occupancy. Initial guess (with negative robustness) via solving an LP.

4. Can be applied in a receding horizon manner. For the given setting, could very well be solved using a linear program asking for temperature between 22-28C for time steps 10 to 19, but we use our method to illustrate how robustness based control can be used to satisfy a specification. The next example (Autonomous ATC) shows control with a specification cannot be trivially translated to a linear program with Polyhedral constraints.

5. Figure shows room temperature for the 3 methods (other states and disturbances/control in a single figure if necessary)

6. Table shows robustness of obtained trajectory via the 3 methods. Note, Optimal solution would be temperature of 25C  (robustness of 3) for the occupancy period (if dynamics/constraints would allow it).



\subsection{Autonomous Air Traffic Controller for quad-rotors}
1. We take a quad-rotor model with linearized dynamics around hover, similar to those used in \cite{}. The case study involves centralized control of two quad-rotors, with operational objectives given as an MTL specification, and a constrained air-space.

1.b. Give model, constraints, specification. Shrinking horizon (fixing history) approach applicable (cite Vasu paper)

2. With the given specification, standard control approaches involving polyhedral constraints are hard to apply because of the temporal aspect of the eventually operator involved. While the two (if-then) altitude rules in the specification can be coded as polyhedral constraints on the set, it would result in a non-convex constraint set for positions. Similarly, the minimum distance between two quad-rotors can also be moved to the constraints but would result in another non-convex constraint if we choose to do so. In our formulation, the non-convexity remains in the cost-function while the constraints are linear.

3. For simulation purposes, we use obtain 3 initial trajectories (via solving different linear programs) from the given initial state to the terminal (landing) set. These three trajectories, each of which has negative robustness (i.e. does not satisfy the given specification), serve as three different initial solutions to A) Our approach B) SQP using the actual robustness function as the cost, C) Simulated Annealing with the actual robustness function as the cost. This multi-start approach can be used in practice when there is a fast initial trajectory generator available.

3.b. parameters for simulation annealing and citation for it.

4. Fig shows the initial trajectories for both the quadrotors in the given air-space. Fig. shows the three trajectories obtained after applying our control method, with the three initial trajectories as starting points for the optimization, respectively. 

5. Table shows for the three trajectories, initial robustness, robustness for trajectory obtained via the three methods (and the approximate robustness when applicable). In addition, we also tested out simulated annealing with the smooth robustness function in the cost (with the first initial trajectory), resulting in a trajectory with a final cost of blah (approximate robustness of blah).





