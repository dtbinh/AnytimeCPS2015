\subsection{Wavelet approximations of the distance function}
\label{sec:waveletApx}

Wavelets can be viewed as a generalization of the kernel $g_\varepsilon$ of the previous section to a whole family of kernels, obtained by translations and dilations of one function called the `mother wavelet' $\psi$: $\psi_{j,k}(x) = 2^{j/2}\psi(2^jx - k)$  \cite{MallatBook}.
They are used extensively in signal processing, as they have very good approximation properties.
See \cite{MallatBook}.
Wavelets give greater control on the approximation error and allow us to 
In the experiments in the rest of the paper, we use wavelet approximations to the distance function.

Specifically, the wavelet approximation to a function $f:\Re^n \rightarrow \Re$ is 
\[f(x) = \sum_{\mathbf{j,k}}c_{\mathbf{j,k}}\psi_{\mathbf{j,k}}(x)\]
where $\mathbf{j} = (j_1\ldots,j_n), \mathbf{k} = (k_1,\ldots,k_n)$ are multi-indices in $\Ze^n$ 
and the $c_{\mathbf{j,k}}$ coefficients are the result of projecting $f$ onto the wavelet basis
\[c_{\mathbf{j,k}} = \int_{\Re^n}f(x)\psi_{\mathbf{j,k}}^*\]
The multi-dimensional wavelet we use is a tensor product of 1-D Meyer wavelets:
\begin{equation}
\label{eq:multid meyer}
\psi_{\mathbf{j,k}} = \psi_{j_1,k_1}\ldots \psi_{j_n,k_n}, \mathbf{j,k} \in \Ze^n
\end{equation}
In the experiments in the rest of this paper, we used the Meyer wavelets \cite{MallatBook}. 

 
