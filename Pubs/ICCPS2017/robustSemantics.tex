\section{Robustness of MTL formulae}
\label{sec:robust semantics}

Consider a discrete-time dynamical system $\Sys$ given by 
\begin{equation}
\label{eq:xt}
x_{t+1} = f(x_t,u_t)
\end{equation}
where $x \in X \subset \Re^n$ is the state of the system and $u \in U \subset \Re^m$ is its control input.
The system's initial state $x_0$ takes value from some initial set $X_0 \subset \Re^n$.
Given an initial state $x_0$ and a control input sequence $\inpSig = (u_0,u_1,\ldots), u_t \in U$, a trajectory of the system is the unique sequence of states $\sstraj = (x_0,x_1,\ldots)$ s.t. for all $t$, $x_t$ is in $X$ and it obeys \eqref{eq:xt} at every time step.
We will use $\TDom$ to abbreviate the time domain $\{0,1,2,\ldots\}$.
All temporal intervals that appear in this paper are discrete-time, e.g. $[a,b]$ means $[a,b] \cap \TDom$. 
For an interval $I$, we write $t+I = \{t+a \such a \in I\}$.
The set of subsets of a set $S$ is denoted $\Pc(S)$.
The signal space $\SigSpace$ is the set of all signals $\sstraj: \TDom \rightarrow X$.
The max operator is written $\sqcup$ and min is written $\sqcap$.

\subsection{Metric Temporal Logic}
\label{sec:mtl}
The controller of $\Sys$ is designed to make the closed loop system \eqref{eq:xt} satisfy a specification expressed in Metric Temporal Logic (MTL) \cite{Koymans90}.
MTL allows one to formally express complex reactive specifications, beyond stability, trajectory tracking and the like.
See examples \eqref{eq:rule1mtl} and \eqref{eq:rule3mtl}.

Formally, let $AP$ be a set of atomic propositions, which can be thought of as point-wise constraints on the state of the system.
An MTL formula $\formula$ is built recursively from the atomic propositions using the following grammar:
\[\formula \defeq \top|p|\neg \formula | \formula_1 \lor \formula_2 | \formula_1 \land \formula_2 | \formula_1 \until_I \formula_2\]
where $I \subset \Re$ is a time interval.
Here, $\top$ is the Boolean True, $p$ is an atomic proposition, $\neg$ is Boolean negation, $\lor$ and $\land$ are the Boolean OR and AND operators, respectively, and $\until$ is the Until temporal operator.
Informally, $\formula_1 \until_I \formula_2$ means that $\formula_1$ must hold \textit{until} $\formula_2$ holds, and that the hand-over from $\formula_1$  to $\formula_2$ must happen sometime during the interval $I$.
The implication ($\implies$), Always ($\always$) and Eventually ($\eventually$) operators can be derived using the above operators.

Formally, the \textit{semantics} of an MTL formula define what it means for a system trajectory $\sstraj$ to satisfy the formula $\formula$.
Let $\Oc: AP \rightarrow \Pc(X)$ be an \textit{observation} map for the atomic propositions.
The validity (boolean truth value) of a formula $\formula$ w.r.t. the trajectory $\sstraj$ at time $t$ is defined recursively.
\begin{definition}[MTL semantics]
	\label{def:boolean sat}
	\begin{eqnarray*}
		\label{eq:boolean sat}
		(\sstraj,t) \models \top &\Leftrightarrow& \top
		\\
		\forall p \in AP, \; \models_\Oc p &\Leftrightarrow& x_t \in \Oc(p)
		\\
		(\sstraj,t) \models_\Oc \neg \formula&\Leftrightarrow& \neg (\sstraj,t) \models_\Oc \formula
		\\
		(\sstraj,t) \models_\Oc \formula_1 \lor \formula_2 &\Leftrightarrow& (\sstraj,t) \models_\Oc \formula_1 \lor (\sstraj,t) \models_\Oc \formula_2
		\\
		(\sstraj,t) \models_\Oc  \formula_1 \land \formula_2&\Leftrightarrow& (\sstraj,t) \models_\Oc \formula_1 \land (\sstraj,t) \models_\Oc \formula_2
		\\
		(\sstraj,t) \models_\Oc \formula_1 \until_I \formula_2 &\Leftrightarrow& \exists t' \in t+I .  (\sstraj,t') \models_\Oc \formula_2  
		\\
		&& \, \land \forall t'' \in (t,t'), \;\; (\sstraj,t'') \models_\Oc \formula_1 
	\end{eqnarray*}
\end{definition}
As $\Oc$ is fixed in this paper, we just drop it from the notation.
We say $\sstraj$ satisfies $\formula$ if $(\sstraj,0) \models \formula$.
All formulas that appear in this paper have bounded temporal intervals: $ 0\leq \inf I < \sup I < +\infty$.
To evaluate whether such a formula $\formula$ holds on a given trajectory, only a finite-length prefix of that trajectory is needed.
Its length can be upper-bounded by the \textit{horizon} of $\formula$, $hrz(\formula)$.
The horizon is the maximum sum of all right endpoints of intervals appearing in $\formula$ \cite{Dokhanchi14_OnlineMonitoring}.
Thus, we need only consider trajectories and input sequences of finite length.

\subsection{Robust semantics of MTL}
\label{sec:rob sem}
Designing a controller that satisfies the MTL formula $\formula$\footnote{Strictly speaking, a controller such that the closed-loop satisfies the formula.} is not always enough.
In a dynamic environment, where the system must react to new unforeseen events, it is useful to have a margin of maneuverability.
That is, it is useful to control the system such that we \textit{maximize} our degree of satisfaction of the formula.
When unforeseen events occur, the system can react to them without violating the formula.
This degree of satisfaction can be formally defined and computed using the robust semantics of MTL.
Given a point $x \in X$ and a set $A \subset X$, $d_A(x) \defeq \inf_{a \in \cl{A}}|x-a|_2$ is the distance from $x$  to the closure $\cl{A}$ of $A$.
\begin{definition}[Robustness\cite{FainekosP09tcs}]
	\label{def:robustness estimate}
	The \emph{robustness} of $\varphi$ relative to $\sstraj$ at time $t$ is recursively defined as 
	\begin{eqnarray*}
		\label{eq:robustness estimate}
		\rob_{\top} (\sstraj,t) &=& +\infty
		\\
		\forall p \in AP, \;  \rob_{p} (\sstraj,t) &=& \left \lbrace \begin{matrix}
			d_{\stSet \setminus \Oc(p)}(\stPt_t), \text{ if } \stPt_t \in \Oc(p)
			\\
			-d_{ \Oc(p)}(\stPt_t), \text{ if } \stPt_t \notin \Oc(p)			
		\end{matrix} \right.
		\\
		\rob_{\lnot \formula} (\sstraj,t) &=& - \rob_{\formula} (\sstraj)
		\\
		\rob_{ \formula_1 \lor \formula_2} (\sstraj,t) &=& \rob_{\formula_1} (\sstraj) \sqcup \rob_{\formula_2} (\sstraj) 
		\\
		\rob_{ \formula_1 \land \formula_2} (\sstraj,t) &=& \rob_{\formula_1} (\sstraj) \sqcap \rob_{\formula_2} (\sstraj) 
		\\
		\rob_{ \formula_1 \until_I \formula_2} (\sstraj,t) &=& \sqcup_{t' \in t+_{\TDom}I} \left(\rob_{\formula_2} (\sstraj,t') \bigsqcap \right.
		\\
		&& \left. \sqcap_{t'' \in [t,t')}   \rob_{\formula_1} (\sstraj,t'') \right) 
	\end{eqnarray*}
	When $t=0$, we write $\robf(\sstraj)$ instead of $\robf(\sstraj,0)$.
\end{definition}
The robustness is a real-valued function of $\sstraj$ with the following important property.
\begin{theorem} \cite{FainekosP09tcs}
	\label{thm:rob objective}
	For any $\sstraj \in \SigSpace$ and MTL formula $\formula$, 
	if $\robf(\sstraj,t) <0$ then $\sstraj$ falsifies the spec $\formula$ at time $t$, and if $\robf(\sstraj) > 0$ then $\sstraj$ satisfies $\formula$ at $t$. 
	The case $\robf(\sstraj,t) =0$ is inconclusive.
\end{theorem} 

Thus, we can compute control inputs by maximizing the robustness over the set of finite input sequences of a certain length.
The obtained sequence $\inpSig^*$ is valid if $\robf(\sstraj,t)$ is positive, where $\sstraj$ and $\inpSig^*$ obey \eqref{eq:xt}.
The larger the magnitude $|\robf(\sstraj,t)|$, the more robust is the behavior of the system: intuitively, $\sstraj$ can be disturbed and $\robf$ might decrease but not go negative.
