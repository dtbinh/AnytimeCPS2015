\subsection{Smooth max and min}
\label{max min smoothing}
We use the following standard smooth approximations of $m$-ary max and min.
Let $k \geq 1$.
\begin{eqnarray}
	\label{eq:soft max min}
	\smax_k(a_1,\ldots,a_m) &\defeq& \frac{1}{k} \ln(e^{ka_1}+\ldots+e^{ka_m})
	\\
	\smin_k(a_1,\ldots,a_m) &\defeq& -\smax(-a_1,\ldots,-a_m)
\end{eqnarray}
Suppose $k=1$ and that $a_1$ is the largest number.
Then $e^{a_1}$ is even larger than the other $e^{a_i}$'s, and dominates the sum. 
Thus $\smax_1(\mathbf{a}) \approxeq \ln e^{a_1} = a_1 = \max(\mathbf{a})$.
If $a_1$ is not significantly larger than the rest, the sum is not well-approximated by $e^{a_1}$ alone.
To counter this, the scaling factor $k$ is used: it amplifies the differences between the numbers.
It holds that for any set of $m$ reals,
\begin{eqnarray}
\label{eq:smooth max error}
0 \leq \smax_k(a_1,\ldots,a_m) -\max(a_1,\ldots,a_m) \leq \ln(m)/k
\\
0 \leq \min(a_1,\ldots,a_m) -\smin_k(a_1,\ldots,a_m) \leq \ln(m)/k
\end{eqnarray}
Indeed, the error of smooth max can be bounded as follows.
Assume $a_1$ is the largest number, then 
\begin{eqnarray*}
\varepsilon_M &\defeq& \smax_k(\mathbf{a}) - a_1 =  \frac{\ln(\sum_ie^{ka_i})-ka_1}{k}
\\
&=& k^{-1}\ln\left(\frac{\sum_ie^{ka_i}}{e^{ka_1}}\right) \leq k^{-1}\ln \left(\frac{me^{ka_1}}{e^{ka_1}}\right)
\\
&=&\frac{\ln m}{k}
\end{eqnarray*}
It is also clear from what preceded that $\varepsilon_M \geq 0$.
The maximum error is achieved when all the $a_i$'s are equal.