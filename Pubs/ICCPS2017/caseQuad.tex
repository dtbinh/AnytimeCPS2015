%CaseQuad

Air Traffic Control (ATC) offers many opportunities for automation to allow safer and more efficient landing patterns. 
The constraints of ATC are complex and contain many safety rules \cite{}???. 
In this example we formalize a subset of these rules for an autonomous ATC in MTL.
We demonstrate how the smoothed robustness is used to generate control strategies for safely and robustly landing two quad-rotors in an enclosed airspace with an obstacle. 

\textbf{The specification}.
The specification for the autonomous ATC with two quadrotors is:
{\small
\begin{subequations}
\begin{align}
\formula &= \eventually_{[0,N]}(q_1 \in \text{Terminal}) \land \eventually_{[0,N]}(q_2 \in \text{Terminal}) \land   \nonumber \\
& \always_{[0,N]} (q_1 \in \text{Zone}_1 \implies z_1 \in [1,5]) \, \land \nonumber \\
& \always_{[0,N]} (q_2 \in \text{Zone}_1 \implies z_2 \in [1,5]) \, \land \nonumber \\
& \always_{[0,N]} (q_1 \in \text{Zone}_2 \implies z_1 \in [0,3]) \, \land \nonumber \\
& \always_{[0,N]} (q_2 \in \text{Zone}_2 \implies z_2 \in [0,3]) \, \land \nonumber \\
& \always_{[0,N]} (\neg (q_1 \in \text{Unsafe})) \land \always_{[0,N]} (\neg (q_2 \in \text{Unsafe})) \, \land  \nonumber \\
& \always_{[0,N]} (||q_1-q_2||_2^2 \geq d_{min}^2)
\end{align}
\end{subequations}
}

Here $q_1$ and $q_2$ refer to the position of the two quad-rotors in $(x,y,z)$-space, and $z_1$ and $z_2$ refer to their altitude. 
The specification says that, within a horizon of $N$ steps,  both quad-rotors 
should: 
a) Eventually land (reach the terminal zone), 
b) Follow altitude rules in two zones, $\text{Zone}_1$ and $\text{Zone}_2$ which have different altitude floors and ceilings,
c) Avoid the $\text{Unsafe}$ set, and d) always maintain a safe distance between each other ($d_{min}$). 

\textit{Note that turning the specification into constraints for the control problem is no longer simple.}
This is due to the $\eventually$ operator, which would require a MILP formulation to be accounted for. 
In addition, the minimum separation and altitude rules for the two zones cannot be turned into convex constraints for the optimization. As will be seen below, our approach allows us to keep the non-convexity in the cost function, and have convex (linear) constraints on the optimization problem.

\textbf{System dynamics.}
See Fig.~\ref{fig:quad_init}.
The airspace is a hyper-rectangle in $\mathbb{R}^3$, $[-5,5] \times [-5,5] \times [0,5]$.
The two quadrotors must start in $\text{Zone}_1 = [-5,0] \times [-5,5] \times [0,5]$, which has a ceiling of $5$m and an enforced floor of $1$m. 
They must eventually reach the terminal set, which is given by $\text{Terminal} = [3,4] \times [3,4] \times [0,1]$.
The terminal set resides in $\text{Zone}_2 = [0,5] \times [-5,5] \times [0,5]$, which has a ceiling of $3$m and a floor of $0$m.
Finally, the $\text{Unsafe}$ set is a hyper-rectangle $[-1,1] \times [-1,1] \times [0,5]$ in the middle of the airspace. In simulation, $d_{min}$ is set to $0.2$ m and the formula horizon $N$ is set to $20$.
\todo[inline]{we might remove some of these dimensions to reduce space, since thee figure shows the sets...}

%This arrangement of the air-space, where the specifications for altitude ceiling and floors for either zone are in a $A \implies B$ format, is common in ATC (e.g. If holding in progress, go to holding zone and follow holding zone rules) and also allows us to combine the $\text{Airspace}$ with velocity-limits ($[-5,5]^3$) into the set $X$, which is a convex (Polyhedron) set constraint on the state of both quad-rotors, as will be explained below.
%\todo[inline]{unnecessary paragraph}

The quad-rotor dynamics are obtained via linearization around hover, and discretization at $5$-Hz. Similar models have been used for control of real quad-rotors with success (\cite{PantAMNDM15_Anytime}). For simulation, we set the mass of either quad-rotor to be $0.5$ kg. Acceleration due to gravity is, as usual, $9.8\,ms^{-2}$.
The corresponding linearized and discretized quad-rotor dynamics are given as:

{\tiny
\begin{equation}
\label{eq:quad_dyn}
\begin{bmatrix} \dot{x}_{k+1} \\ \dot{y}_{k+1} \\ \dot{z}_{k+1} \\ x_{k+1} \\ y_{k+1} \\ z_{k+1} \end{bmatrix}= \begin{bmatrix} 1&0&0&0&0&0 \\0&1&0&0&0&0 \\0&0&1&0&0&0 \\0.2&0&0&1&0&0 \\0&0.2&0&0&1&0 \\0&0&0.2&0&0&1\end{bmatrix} \begin{bmatrix} \dot{x}_{k} \\ \dot{y}_{k} \\ \dot{z}_{k} \\ x_{k} \\ y_{k} \\ z_{k} \end{bmatrix} + \begin{bmatrix} 1.96&0&0 \\ 0&-1.96&0 \\0&0&0.4 \\0.196&0&0 \\0&-0.196&0\\0&0&0.04 \end{bmatrix} \begin{bmatrix} \theta_k \\ \phi_k \\ \text{T}_k \end{bmatrix}
\end{equation}
}

Here, the state consists of the velocities and positions in the $x,y,z$ co-ordinates. 
The inputs to the system are the desired roll angle $\theta$, pitch angle $\phi$ and thrust $\text{T}$. 
In a real system, this thrust is in addition to the hover thrust, $mg$, and a high-frequency low-level controller (not simulated) is responsible for generating rotor speeds to match the desired roll, pitch and thrust (with yaw set to be stabilized at $0$). 

\textbf{The control problem.}
We solve the following control problem.
\begin{subequations}
\label{eq:atc_ctrl}
\begin{align}
\text{max } & \srob_{\formula}(\sstraj) \\ %- \gamma \sum_{k=0}^{N-1} l(x_{k+1},u_{k}) \\
\text{s.t. } & x_{k+1} = f(x_k,u_k), \, \forall k=0,\dotsc,N-1 \\
 & x_k \in X, \, \forall k=0,\dotsc,N \\
 & u_k \in U, \, \forall k=0,\dotsc,N-1 
% & \delta \srob_{\formula}(\sstraj) \geq 0
\end{align}
\end{subequations}

Here, with some abuse of notation, $x$ represents the concatenated state of the two quadrotors, and $u$ the concatenated inputs to the system. 
$X$ and $U$ represent the bounds on the states and inputs respectively. 
$f$ represents the linearized dynamics of \eqref{eq:quad_dyn}. The initial state for the first quad-rotor is $[2\,2\,2\,0\,0\,0]'$ and for the second, $[2, -2 , 2 ,0 ,0 ,0]'$
%The control problem is to maximize the smooth robustness of the specification $\srob_{\formula}$, subject to the dynamics of two identical quad-rotors, and input limits of $0.5236$ radians (or $30$ degrees) on both roll and pitch, and $\text{T}\in[-1.5,1.5]$, giving us the set $U$. The set $X$ for state limits on either quadrotor is, as defined earlier, the Cartesian product of velocity limits $[-5,5]^3$ and the $\text{Airspace}$ set, i.e. $[-5,5]^3$. Also with respect to the general control problem, \eqref{eq:general_ctrl}, we set $\gamma=0$ and $\delta=0$. This keeps the smooth robustness only in the cost (to be maximized) and allows a fair comparison of the optimization of robustness with other methods. In general, our formulation has linear constraints (dynamics and $X$, $U$) for both quad-rotors, allowing us to easily apply off the shelf optimization solvers.
%\todo[inline]{repetitive. get to it. remove this paragraph and replace it with the control problem mathematically, as done for toy example.}
 %, Simulated Annealing and gradient descent, via SQP, on the robustness function.

%For simulation purposes, 
%\todo[inline]{what other purposes are there?}
%we obtain 3 initial trajectories (via solving different linear programs) from the given initial state to the terminal (landing) set. These three trajectories, each of which has negative robustness (i.e. does not satisfy the given specification), serve as three different initial solutions to A) Our approach B) Gradient-descent via SQP, using the actual robustness function as the cost, C) Simulated Annealing with the actual robustness function as the cost. This multi-start approach can be used in practice when there is a fast initial trajectory generator available.
%\todo[inline]{re-word in a much more direct style, as follows:}

\textbf{Results.}
For each approach, we ran three optimizations, starting from three different trajectories to initialize the optimization. 
These \textit{initial trajectories} can be obtained in practice by a fast trajectory generator. 
The three initial trajectories all have negative robustness, i.e. they violate $\formula$.
\begin{figure}[t]
\centering
\includegraphics[width=0.49\textwidth]{figures/QuadInitTrajs_scissored}
\caption{The airspace with the corresponding sets, and initial trajectories for the two quad-rotors. Note, all 3 initial trajectories violate the specification. Here, $q0_{i}^j$ refers to the positions of the $i^{th}$ initial trajectory for the $j^{th}$ quadrotor.} 
\label{fig:quad_init}
\end{figure}

Fig.\ref{fig:quad_init} shows the initial trajectories for both the quadrotors in the given air-space, neither of which satisfy the specification $\formula$. 
Fig.\ref{fig:quad_ssqp} shows the three trajectories obtained after applying SR-SQP,
with the three initial trajectories as initial guesses for the optimizations. 
%\todo[inline]{use the names of the methods, here, SR-SQP} 
%\todo[inline]{it's either a trajectory or a point. the reader can't peer into your head and see that you're thinking about them in the ssame way. Stick to one nomenclature, "initial trajectory". moreover, we've already explained above what an initial rajectory is, so we don't have to keep explaining that it initializes the optim.}
%for the optimization, respectively. 
All three trajectories obtained by SR-SQP satisfy the specification $\formula$. 
To avoid visual clutter, we do not show the trajectories obtained from the other two methods on the figure.
Instead, we summarize the results in Table \ref{tbl:opt_performance} which shows the true robustness of the three initial trajectories, and the true robustness for the trajectories obtained via the three methods, SR-SQP, SA, and R-SQP.
%Our method, SQP with Smooth Robustness (SR-SQP)
%\todo[inline]{define once, use forever: SR-SQP}, 
%results in trajectories with the highest robustness.
%SA, with an upper limit of 30,000 function evaluations, results in only one trajectory that satisfies $\formula$. 
%SQP on robustness 
%\todo[inline]{chicken on rice. R-SQP}
%results in trajectories that satisfy $\formula$ from all 3 initial trajectories. 
%R-SQP consistently returns lower values of robustness than SR-SQP, but also from three very different initial trajectories, 
%\todo[inline]{"very" initial?}
%\todo[inline]{re-organize: SR-SQP and R-SQP satisfies the spec every time, SA only once.
%	\\ SR-SQP gives highest robustness values, higher than R-SQP.
%	\\ observations below about getting stuck at local minima due to non-diff. }
%results in trajectories with the same robustness value $0.1798$. 

It is seen that SR-SQP and R-SQP satisfy $\formula$ for all instances, while SA satisfies it only once.

Note that in all three cases, R-SQP results in trajectories with the same robustness value. 
We conjecture that this is because R-SQP is getting stuck at local minima at points of non-differentiability of the objective, as illustrated in Example \ref{ex:cluster nondiff}.
On further investigation, we also noticed that the robustness value achieved is due to the segment of the $\formula$ corresponding to $\eventually_{[0,N]}(q_2 \in \text{Terminal})$. R-SQP does not drive the trajectory (for quad-rotor 2) deeper inside the set $\text{Terminal}$, unlike the proposed approach, SR-SQP, even though the minimum separation property is far from being violated. This lends credence to our hypothesis of SQP terminating on a local minima (which is indeed the flag MATLAB's optimization gives).

%\todo[inline]{this last discussion is important. break down your sentences, clarify it, take it easy. you've done the work, don't rush the explanation.}


\begin{figure}[t]
\centering
\includegraphics[width=0.49\textwidth]{figures/QuadTrajs_scissored}
\caption{Trajectories obtained via SQP on smooth robustness, with three different initial trajectories acting as initial solutions for the SQP. Note, all 3 trajectories satisfy $\formula$. Here, $q_{i}^j$ refers to the positions of the $i^{th}$ initial trajectory for the $j^{th}$ quadrotor. Consider $q_{1}^1$ and $q_{2}^1$, the first trajectory for the two quad-rotors. The second quad-rotor swerves around the obstacle and reaches $\text{Terminal}$, while the first quad-rotor swerves towards $\text{Unsafe}$ (but without violating $\formula$) in order to maintain a safe distance with the other quad-rotor while reaching $\text{Terminal}$. This results in a the two quad-rotors satisfying $\formula$. Similar behavior is seen with the other two trajectories as well.}
\label{fig:quad_ssqp}
\end{figure}


%
{\small
\begin{table}[htb]
\begin{center}
\caption{Robustness of final trajectory $\sstraj^*$ for three optimization runs with different initial trajectories}
\label{tbl:opt_performance}
\begin{tabular} {|c|c|c|c|c|}
	\hline
	\textbf{Run} & $\rob(\sstraj_0) $ &SR-SQP $\rob(\sstraj^*) /\srob$ & SA: $\rob(\sstraj^*)$ & R-SQP: $\rob(\sstraj^*)$\\ \hline
	1 & -0.8803 & \textbf{0.2985} /\,0.2460 & -0.2424 & 0.1798 \\ \hline
	2 & -0.7832 & \textbf{0.3255} /\,0.3103 & -0.5861 & 0.1798 \\ \hline
	3 & -0.0399 & \textbf{0.2967} /\,0.2652 & 0.0854 & 0.1798 \\ \hline
\end{tabular}	
\end{center}
\end{table}
}

