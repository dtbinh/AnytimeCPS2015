\subsection{Overall approximation}
\label{sec:overall apx}
Putting the pieces together, we obtain the approximation error for the robustness of any MTL formula.
\begin{theorem}
	\label{thm:total apx error}
	Consider an MTL formula $\formula$ and reals $\varepsilon > 0$ and $k \geq 1$. 
	Given a set $U$, let $\sdist_\varepsilon(\cdot,U)$ be an $\varepsilon$-approximation of $\dist(\cdot,U)$, i.e. for all $x$ in their common domain, $|\dist(x,U) - \sdist_\varepsilon(x,U)| \leq \varepsilon$.	
	
	Define the \textit{smooth robustness} $\srobf$, obtained by substituting $\sdist_{\varepsilon}$ for $\dist$, $\smax_k$ for $\max$, and $\smin_k$ for $\min$, in Def. \ref{def:robustness estimate}.
	Then for any length-$N$ trajectory $\sstraj$, it holds that
	\[|\robf-\srobf| \leq \delta_\formula\]
	where $\delta_\formula$ is
	(a) independent of $N$,
	(b) independent of the evaluation time $t$, and 
	(c) goes to 0 as $\varepsilon \rightarrow 0$ and $k \rightarrow \infty$.
%	\[|\robf-\srobf| \leq |\formula| \cdot \max(\varepsilon,\ln(m)/k)\]
%	where $m$ is the length of the longest temporal interval appearing in $\formula$ and $|\formula|$ is the size of the formula, i.e. the number of operators that appear in it.
\end{theorem}
\begin{proof}
	We will prove a stronger result that implies the theorem.	
%	Define $\delta_\formula \in \Re_+$ inductively: 
%	$\delta_\top = 0, \delta_p = \varepsilon$, $\delta_{\neg \formula} = \delta_\formula$, 
%	$\delta_{\formula_1 \lor \formula_2} = \delta_{\formula_1 \land \formula_2} = \delta_{\formula_1} \sqcup \delta_{\formula_2} + \ln(2)/k$,
%	$\delta_{\formula_1 \until_{[a,b]} \formula_2}  = \delta_{\formula_1} \sqcup \delta_{\formula_2} + \ln(b-a)/k$.
%	Noting that $\delta_\formula \leq |\formula| \cdot \max(\varepsilon,\ln(m)/k)$, it suffices to show that the error is upper-bounded by $\delta_\formula$.
	When $\sstraj$ or $t$ are clear from the context, we will drop them from the notation.
	
	The proof is by structural induction on $\formula$, and works by carefully characterizing the approximation error.
	
\underline{Case $\formula = p \in AP$.}
$\robf(\sstraj,t)$ is given by either $\dist{x_t}{\Oc(p)}$ or $-\dist{x_t}{\Oc(p)}$, and 
$\srobf(\sstraj,t)$ is given by either $\sdist_\varepsilon(x_t,\Oc(p))$ or $-\sdist_\varepsilon(x_t,\Oc(p))$, respectively.
Either way, $|\srobf(\sstraj,t) - \robf(\sstraj,t)| \leq \varepsilon$.
Indeed, $\varepsilon$ satisfies the conditions on $\delta_\formula$.

\underline{Case $\formula = \neg \formula_1$} 
$|\rob_{\neg \formula_1}(\sstraj,t)-\srob_{\neg \formula_1}(\sstraj,t)| = |-\rob_{\formula_1}(\sstraj,t) + \rob_{ \formula_1}(\sstraj,t)|  \leq \delta_{\formula_1}$, and $\delta_{\formula_1}$ satisfies (a)-(c) by induction hypothesis.

\underline{Case $\formula = \formula_1 \lor \formula_2$}.
If the same sub-formula $\formula_i$ achieves the max for both $\rob_{\formula_1}(\sstraj,t) \sqcup \rob_{\formula_2}(\sstraj,t)$ and $\srob_{\formula_1}(\sstraj,t) \sqcup \srob_{\formula_2}(\sstraj,t)$, then by induction hypothesis we immediately obtain 
$|\robf(\sstraj,t)-\srob(\sstraj,t)|  \leq \delta_{\formula_i}$.

Otherwise if, say, $\robf = \rob_{\formula_1}$ and $\srobf = \srob_{\formula_2}$ then
\[\robfa -\delta_{\formula_1} \leq \srob_{\formula_1} \leq \srob_{\formula_2} \implies \robfa-\srob_{\formula_2} \leq \delta_{\formula_1}\]
Also 
\[\srob_{\formula_2} \leq \robfb+\delta_{\formula_2} \leq \robfa + \delta_{\formula_2} \implies -\delta_{\formula_2} \leq \robfa - \srob_{\formula_2}\]
Therefore
\begin{equation*}
\label{eq:inter}
-(\delta_{\formula_1} \sqcup \delta_{\formula_2})\leq \robfa - \srob_{\formula_2}\leq \delta_{\formula_1} \sqcup \delta_{\formula_2} \Leftrightarrow |\robfa - \srob_{\formula_2}|\leq \delta_{\formula_1} \sqcup \delta_{\formula_2}
\end{equation*}
Similarly, if $\robf = \rob_{\formula_2}$ and $\srobf = \srob_{\formula_1}$, we have $|\robfb - \srob_{\formula_1}|\leq  \delta_{\formula_1} \sqcup \delta_{\formula_2}$.
So in all cases, 
\[|\robfa \sqcup \robfb - \srob_{\formula_1} \sqcup \srob_{\formula_2}|\leq \delta_{\formula_1} \sqcup \delta_{\formula_2}\]
Therefore by the triangle inequality and \eqref{eq:smooth max error}
\[|\robfa \sqcup \robfb - \smax_k(\srob_{\formula_1} , \srob_{\formula_2})|\leq  \delta_{\formula_1} \sqcup \delta_{\formula_2} + \ln(2)/k  = \delta_\formula\]
Clearly, $\delta_\formula$ satisfied (a)-(c).

The case $\formula_1 \land \formula_2$ is treated similarly.

\underline{$\formula = \formula_1 \until_I \formula_2$.} 
Before proving this case, we will need the following lemma, which is provable by induction on $n$:
\begin{lemma}
	\label{lemma:n-ary apx}
	If $\formula = \formula_1\land \ldots\land \formula_n$ or $\formula = \formula_1\lor \ldots\lor \formula_n$, $n\geq 2$, then 
	$|\robf - \srobf| \leq \sqcup_{1\leq i \leq n}\delta_{\formula_i} + \ln(n)/k$.
\end{lemma}

We now proceed with the proof of the last case.
Recall that $\rob_{ \formula_1 \until_I \formula_2} (\sstraj,t) = \sqcup_{t' \in t+_{\TDom}I} \left(\rob_{\formula_2} (\sstraj,t') \bigsqcap \right.
\\
\left. \sqcap_{t'' \in [t,t')}   \rob_{\formula_1} (\sstraj,t'') \right)$.
Starting with the innermost sub-expression $\rob_\psi \defeq \sqcap_{t'' \in [t,t')}   \rob_{\formula_1} (\sstraj,t'')$, we have, by Lemma \ref{lemma:n-ary apx}
\begin{equation}
\label{eq:robpsi}
|\rob_{\psi}-\srob_\psi| \leq \sqcup_{t'' \in [t,t')}\delta_{\formula_1}^{t''} + \ln(t'-t)/k
\end{equation} 
where $\delta_{\formula_1}^{t''} $ is the bound for approximating $\rob_{\formula_1}(\sstraj,t'')$.
But $\delta_\formula$ does not depend on the time at which the formula is evaluated. 
Therefore the bound in \eqref{eq:robpsi} becomes
\begin{equation}
\label{eq:t'-t}
|\rob_{\psi}-\srob_\psi| \leq \delta_{\formula_1} + \ln(t'-t)/k
\end{equation} 
%
To avoid introducing a dependence on time, we further upper-bound by 
\begin{equation*}
|\rob_{\psi}-\srob_\psi| \leq \delta_{\formula_1} + \ln(hrz(\formula))/k \defeq \delta_\psi
\end{equation*} 
where, recall, $hrz(\formula)$ is the horizon of $\formula$ (see Section \ref{sec:mtl}).

Continuing with the sub-expression $\rob_\alpha = \rob_{\formula_2} (\sstraj,t') \bigsqcap \rob_\psi$, by the induction hypothesis it holds that 
$|\rob_\alpha - \srob_\alpha| \leq \delta_{\formula_2} \sqcup \delta_\psi + \ln(2)/k \defeq \delta_\alpha$.
%
Finally, the top-most max operator introduces the total error 
\begin{eqnarray}
\label{eq:deltaphi until}
|\robf-\srobf| &\leq& \delta_\alpha + \ln(|I|)/k 
\nonumber
\\
&=& \delta_{\formula_2} \sqcup \delta_\psi + \ln(2)/k + \ln(|I|)/k 
\nonumber 
\\
&=& \delta_{\formula_2} \sqcup ( \delta_{\formula_1} + \ln(hrz(\formula))/k )+ \ln(2|I|)/k 
\nonumber
\\
&=&\delta_\formula
\end{eqnarray}
The first inequality obtains from the fact that $\delta_\alpha$ is independent of evaluation time and Lemma \ref{lemma:n-ary apx}.
The bound $\delta_\formula$ obeys (a)-(c).
This concludes the proof.
%if $\formula_1$ is a sub-formula of $\formula$, then $\delta_{\formula_1} \leq \delta_\formula$.
	\end{proof}
	
\begin{rem}
	The proof suggests in \eqref{eq:deltaphi until} that every Until operator increases the error by $+\ln(hrz(\formula))/k$, which may be a large quantity for long-horizon formulas.
	However, a more careful analysis of the accumulation of interval widths $t'-t$, which we upper bounded above by $hrz(\formula)$, reveals that that is not the case.
	Indeed, consider the formula
	\[\formula = \underbrace{(\formula_1 \until_{[a,b]} \formula_2) }_{\psi}\until_{[c,d]} \formula_3\]
	where $\formula_i$ does not contain temporal operators and let $t_\psi$ be the satisfaction time of $\psi$. 
	I.e. it is the time in $[a,b]$ when $\formula_2$ becomes true and `takes over' from $\formula_1$.
	Then, using \eqref{eq:t'-t}, $\delta_{\formula}$ is equal to
	\begin{eqnarray*}
	 && \delta_{\formula_3} \sqcup ( \delta_{\psi} + \ln(t_{\formula_3} - t_\psi)/k )+ \ln(2[d-c])/k 
	\\
	&\leq& \delta_{\formula_3} \sqcup ( \delta_{\psi} + \ln(t_\psi + d - t_\psi)/k )+ \ln(2[d-c])/k 
	\\
	&\leq& \delta_{\formula_3} \sqcup ( \delta_{\psi} + \ln(d)/k )+ \ln(2[d-c])/k 
	\\
	&=& \delta_{\formula_3} \sqcup \left( \underbrace{\delta_{\formula_2} \sqcup ( \delta_{\formula_1} + \ln(b)/k )+ \ln(2[b-a])/k }_{\delta_\psi} + \ln(d)/k \right)
	\\
	&& \quad + \ln(2[d-c])/k 
	\end{eqnarray*}
	The sum of interval widths, such as $\ln(2[b-a])+\ln(2[d-c])$ in this example, gives a \textit{total} $\sum_j\ln(2|I_j|) = \#\until\ln(2) +  \sum\ln(|I_j|)$, where $\#\until$ is the number of times $\until$ appears in $\formula$ and th e$I_j$ are all the temporal intervals. 
	This quantity, in most cases, will actually be smaller than the horizon.
	The sum of interval right endpoints, such as $\ln(b)/k + \ln(d)/k$ in this example, yields $\sum_j \ln(\sup I_j)/k$ \textit{in total}, which is smaller than the horizon.
\end{rem}