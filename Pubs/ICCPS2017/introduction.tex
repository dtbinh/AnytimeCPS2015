\section{Controlling for robustness}
\label{sec:intro}
Cyber-Physical Systems (CPS) must withstand a wide range of problems, from bugs in their software to attacks on and failures of their physical sensors.
Under certain error models, like a bounded disturbance on a sensor reading 
%or a wrong transition in the logic
, a CPS can be designed to be robust to that source of error.
In general, however, unforeseen issues can occur. 
To deal with unforeseen problems, at design time, the system must be verified to be \textit{robust} : i.e., not only does it satisfy its design specifications under the known error models, it must satisfy robustly.
Similarly, at runtime, the system's controller must make decisions that maximize this satisfaction margin, or \textit{robustness}.
This can give a margin of maneuvarability to the system during which it addresses the unforeseen problem.
Since these problems are, by definition, unforeseen and unmodeled and only detected by their effect on the output, the notion of robustness must be computable using only the output behavior of the system.

The \textit{robustness of Metric Temporal Logic specifications} \cite{Fainekos2006_TLVerifSimu,Donze2010} is such a notion that has been successfully for the verification of automotive systems \cite{Fainekos12_Automotive,Dreossi15_RRTFalsification}, medical devices \cite{SankaranarayananF2012cmsb}, and general CPS.
Given a specification $\formula$ written in Metric Temporal Logic (MTL) and a system execution $\sstraj$, the robustness $\robf(\sstraj)$ of the spec relative to $\sstraj$ measures two things:
its sign tells whether $\sstraj$ satisfies the spec ($\robf(\sstraj) > 0$) or falsifies it (i.e., violates, $\robf(\sstraj) <0$).
Its magnitude $|\robf(\sstraj)|$ measures how \textit{robustly} the spec is satisfied or falsified.
Namely, any perturbation to $\sstraj$ of size less than $|\robf(\sstraj)|$ will not cause its truth value to change.
Thus, the control algorithm can \textit{maximize} the robustness over all possible control actions to determine the next control input.

Unfortunately, the robustness function $\robf$ is hard to work with.
It is recursively defined (does not have a closed form), non-convex, non-differentiable and does not obey an optimality principle.
This makes its optimization a challenge - indeed, most existing approaches treat it as a black box and apply heuristics to its optimization (see Related Work below).
These heuristics provide little to no guarantees, have too many user-set parameters, and don't have rigorous termination criteria.
On the other hand, \textit{gradient descent optimization} algorithms typically offer convergence guarantees to the function's (local) minima, have known convergence rates for certain function classes, and have been optimized so they outperform heuristics that don't have access to the objective's gradient information.
The existence of a gradient also allows us to do \textit{local} search for falsifying trajectories, which is necessary for corner case bugs.
Moreover, such optimization algorithms usually have a fewer number of parameters to be set by the user, and important issues like step-size selection are rigorously addressed.

\textbf{Contributions}. In this paper, we present smooth (infinitely differentiable) approximations to the robustness function of arbitrary MTL formulas.
This allows us to run powerful and rigorous off-the-shelf gradient descent optimizers.
We show that the smooth approximation is always within a user-defined error of the true robustness, and illustrate the result experimentally.
%\yhl{We demonstrate that optimizing the smooth approximation yields near-minima of the original robustness}.
We demonstrate that the resulting falsification and control algorithms, which use gradient descent on the smoothed robustness, perform better than a heuristic like Simulated Annealing optimizing the original, non-differentiable robustness.
We demonstrate the results on a control case study for an autonomous airport traffic controller.

\textit{Related work.}
Current approaches to optimizing the robustness fall into four categories: 
the use of heuristics like Simulated Annealing and RRTs~\cite{NghiemSFIGP10hscc,AbbasF_HybridSA12,SankaranarayananF2012hscc,Dreossi15_RRTFalsification,zutshi_trajectory_2013,Deshmukh15_IterativeApproaches}, 
nonsmooth optimization \cite{AbbasF13acc}, 
Mixed Integer Linear Programming (MILP) \cite{Raman14_MPCSTL}, 
and iterative approximations \cite{AbbasATVA11_LinFalsification,Abbas14_MTLDescent}.
Black-box heuristics are the most commonly used approach: for example, Simulated Annealing~\cite{NghiemSFIGP10hscc}, cross-entropy \cite{SankaranarayananF2012hscc} and RRTs~\cite{Dreossi15_RRTFalsification}.
The clear advantage of these methods is that they do not require any special form of the objective function: they simply need to evaluate it at various points of the search space, and use its value as feedback to decide on the next point to try.
A significant shortcoming is that, unlike gradient descent optimization, they offer little to no guarantees of convergence to local minima, and their convergence rates are often not known. 
They also use many `magic' parameters that are heurstically set and may affect the results significantly, thus requiring more user interaction than desired.
%The same is true of deterministic heuristics \cite{zutshi_trajectory_2013}, albeit they tend to be more transparent to the user and can be better tuned.
%\todo[inline]{look up latest convergence of cross entropy for continuous spaces, genetic algos}
Because the robustness is non-smooth, the work in \cite{AbbasF13acc} developed an algorithm that decreases the objective function along its sub-gradient. 
This involved a series of conservative approximations, and was restricted to the case of safety formulas.
In \cite{Raman14_MPCSTL}, the authors encoded the MTL formula as a set of linear and boolean constraints, and used an MILP solver to solve them.
%The resulting MILP and used a Mixed Integer Linear Program (MILP) has $O(N\cdot|P|)$ binary variables (where $N$ is the number of samples in the trajectory over which we optimize and $|P|$ is the number of predicates, and $O(N\cdot |\formula|)$ continuous variables.
MILPs are NP-hard, and the sophisticated heuristics used to mitigate this make it hard to characterize their runtimes, which is important in control - see examples in \cite{Raman14_MPCSTL}.
The work closest to ours is \cite{AbbasATVA11_LinFalsification,Abbas14_MTLDescent}.
There, the authors considered safety formulas, for which the robustness reduces to the minimum distance between $\sstraj$ and the unsafe set $U$.
By sub-optimally focusing on one point on the trajectory $\sstraj$, they replaced the objective by a differentiable indicator function for $U$ and solved the resulting problem with gradient descent.


