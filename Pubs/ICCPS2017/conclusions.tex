\section{Conclusions}
%\todo[inline]{use either formulas or formulae consistently}
%\todo[inline]{use {\small } for figure and table captions}
%\todo[inline]{thre's some funny spacing sometimes..}\todo[inline]{use commas in vectors, [2,2,-1] to distinguish the entries}
%\todo[inline]{we present 3 diff control problems, one with and one without control cost. need to make this a bit smoother, right now it's something of a back and forth between the three. (i know, they're special cases of each other, but it reads bad)}
We present a method to obtain smooth (infinity differentiable) approximations to the robustness of MTL formulae, without bounded and asymptotically decaying approximation error. Empirically, we show that the approximation is indeed small for a variety of commonly used MTL formulae. Through multiple examples, we show how we leverage the smoothness property of the approximation for solving a control (or falsification) problem by maximizing (or minimizing) the robustness using an off the shelf gradient based optimization technique, SQP. We compare our technique (SR-SQP) to two other approaches for robustness maximization (SA, R-SQP) for control of two dynamic systems, with state and input constraints, and show how our approach consistently outperforms the other two and can be used for control of dynamical systems to satisfy an MTL specification.

%\begin{figure}[t]
%\centering
%\includegraphics[width=0.49\textwidth]{figures/Habbas}
%\caption{{\small Habbas, pictured here, is the author of this paper, half the papers cited in this paper, half the papers submitted to CPS Week and many others. He can be contacted at habbasATseas.upenn.edu}}
%\label{fig:quad_ssqp}
%\end{figure}
