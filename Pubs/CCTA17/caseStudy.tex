\section{Case studies}
\label{sec:case study}

To evaluate our approach, we solve a robustness maximization problem for control of two systems, using four methods (all of which are implemented in MATLAB)

\begin{itemize}
	\vspace{-5pt}
	\item SR-SQP, which uses SQP to optimize the smooth approximation to robustness, $\srobf$. We run this for both $boolean$ and $robust$ modes of operation as defined in the previous section.
	\vspace{-5pt}
	\item BluSTL. Also run for both $boolean$ and $robust$ modes of operation.
	\item R-SQP, which uses SQP to optimize the \textit{true} robustness, $\robf$.
	\vspace{-5pt}
	\item SA, which uses Simulated Annealing to optimize $\robf$.%the true robustness
	\vspace{-5pt}
\end{itemize}

%A) Temperature control of a 4-state, single-zone model of a room, with an occupancy dependent comfort specification (similar to \cite{Raman14_MPCSTL}) and B) An Autonomous Centralized Air Traffic Control system for control of two quad-rotors in a constrained airspace with an obstacle, the specification in which corresponds to safely landing the two quad-rotors while following the rules of the airspace. We compare out approach, which consists of gradient descent on the smooth robustness of the specification (via SQP) to A) Simulated Annealing (\cite{}) for robustness maximization and B) Gradient descent on the robustness function (via SQP). 
%\todo[inline]{rather than this intro, just start giving each case study subsection. so lose above paragraph, maybe just one sentence that says "we evaluated the control performance using smooth robustness on a alnging bla and a building bla." Also don't use A and B twice, every time for different things. }

%All methods are implemented in MATLAB, and 
%solving the control problem is preceded by an offline phase which consists of finding the wavelet parameters for the signed distance function approximation for all atomic propositions involving polyhedrons. 
%\todo[inline]{sentence is long and awkward and mixes 2 very different pieces of information. can simply say "we compute the wavelet approimation of the distance function to the sets $\Oc(p)$ off-line"}

For both examples considered here, first, we compute the wavelet approximation of the distance function to the sets $\Oc(p)$ off-line. Next, we solve the control problem \eqref{eq:general_ctrl} as a single shot, finite horizon constrained optimization. 
%In general, for a real online implementation, the problem can be solved in a receding horizon manner or in a manner where the state and actions of the past are stored an added as constraints at each time step while the look-ahead horizon of the optimization shrinks (similar to \cite{Raman14_MPCSTL}). The problem of which method is suitable for what kinds of specifications is left for future work.
%\todo[inline]{too soon, the control problem hasn't been formulated yet. first forumlate it (or, better, refer to section 4.2), \textit{solve it}, then critique it. Discussion of related work like Vasu's should be after you present what you did, not before}
%\todo[inline]{introduce two numerical examples. Finite horizon control in a single shot, can be applied in a receding/shrinking horizon approach wherever applicable.}


\subsection{HVAC Control of a building for comfort}
%\todo[inline]{skipping for now. i suspect many of my comments for ATC are applicable here, so please apply them as needed}
%caseBldg

In order to evaluate the control performance of our method on a system with more meaningful dynamics than the point-mass system, we test it on temperature control of a 4-state model of a single zone room in a building (\cite{HAMLAB}). Such a model is commonly used in literature for evaluation of predictive control algorithms \cite{AchinACC}. The control problem we solve is similar to the example used in \cite{Raman14_MPCSTL}, where the objective is to bring the room temperature to a comfortable range when the building is occupied (given predictions on the building occupancy). The specification is gives as:
\begin{equation}
\label{eq:BldgSpec}
\Psi = \always_I(\text{RoomTemp} \in \text{Comfort})
\end{equation}
Here, $I$ is the interval where the room is occupied, and $\text{Comfort}$ is the range of temperatures (in Celsius) deemed comfortable ($[22,28]$). For the control horizon, we consider a 24 hour period, in which the building is occupied from time steps $10$ to $19$ (i.e. $I=[10,19]$), corresponding to a 10-hour workday. The single-zone model, discretized at a sampling rate of 1 hour (which is common in building temperature control) is of the form:

\begin{equation}
x_{k+1} = Ax_{k}+Bu_k+B_dd_k
\end{equation}
Here, $A$, $B$ and $B_d$ matrices are from the HAMLAB model. $x \in \mathbb{R}^4$ is the state of the model, the $4^{th}$ element of which is the room temperature, the others are auxiliary temperatures corresponding to other physical properties of the zone. The input to the system, $u \in \mathbb{R}^1$ is the heating/cooling energy to the system. $b_d \in \mathbb{R}^3$ are disturbances (due to occupancy, outside temperature, solar radiation), which we assume perfect predictions of. Data for the disturbances is obtained for $1^{st}$ April 2000, which is our 24 hours of interest, from the ??? data set. The control problem we solve is of the form in \eqref{eq:general_ctrl}, with $\gamma$ and $\delta$ both set to zero, i.e. robustness is only in the objective, to be maximized. This allows for a fair comparison between our method (SQP on smooth robustness), gradient descent (SQP) on robustness, and Simulated Annealing (SA) for robustness maximization. With respect to the general control problem of \eqref{eq:general_ctrl}, the limits on the states are $X=[0,50]^4$ and on the inputs $U=[-1000,2000]$.

To initialize the optimzation for all three methods, we generate an initial trajectory, starting from $x_0=[21 \, 21 \, 21 \, 21]'$, which does not satisfy $\formula$. The trajectories after optimization from the three methods are shown in Fig.\ref{fig:ZoneTemp}. Our method and SA both result in trajectories that satisfy $\formula$, with a robustness of $2.9994$ and $2.8862$ respectively. On the other hand, SQP for gradient descent on the robustness function results in a trajectory that does not satisfy $\formula$ ($\rob_{\formula} = -0.1492$), and terminates on a local maxima. This is possibly due to the lack of existence of a gradient along certain directions. 

In the particular problem, with the given comfort range, the maximum robustness achievable in the unconstrained case would $3$, achieved by setting the room temperature at $25$C for the interval $I$. Our method results in a robustness which is just $0.02\%$ less than the unconstrained optimal value, while SA gets to a value $3.8\%$ less than the (unconstrained) optima. Note, for this particular problem, since we assume perfect knowledge of occupancy and disturbances, the problem of satisfying the formula, or indeed of even maximizing robustness, can be solved simply with a quadratic program with linear constraints. We use this example to illustrate the applicability of our method, as well as its performance, while adding a word of caution against the naive use of gradient descent for robustness maximization, even while the gradient of robustness exists \textit{almost everywhere}. In the following example, we take a specification which cannot be trivially turned into a quadratic-program without adding tighter constraints than the specification asks for (or binary variables).


\begin{figure}[t]
\centering
\includegraphics[width=0.49\textwidth]{figures/ZoneTemp_scissored}
\caption{Zone temperatures. The green rectangle shows the comfortable temperature limit of 22-28 C, applicable during time steps 10-19 (when the building is occupied).}
\label{fig:ZoneTemp}
\end{figure}

%1. Single zone building model from ...

%2. Specification for comfort when occupied.

%3. 24 hour look ahead, given disturbances and occupancy. Initial guess (with negative robustness) via solving an LP.

%4. Can be applied in a receding horizon manner. For the given setting, could very well be solved using a linear program asking for temperature between 22-28C for time steps 10 to 19, but we use our method to illustrate how robustness based control can be used to satisfy a specification. The next example (Autonomous ATC) shows control with a specification cannot be trivially translated to a linear program with Polyhedral constraints.

%5. Figure shows room temperature for the 3 methods (other states and disturbances/control in a single figure if necessary)

%6. Table shows robustness of obtained trajectory via the 3 methods. Note, Optimal solution would be temperature of 25C  (robustness of 3) for the occupancy period (if dynamics/constraints would allow it).


%\vspace{-10pt}
\subsection{Autonomous ATC for quad-rotors}
\label{sec:ATCquad}
%\todo[inline]{title exceeds margin}
%CaseQuad
\todo[inline]{Start by motivating the example as something more than a dumb system. E.g. "Air traffic control offers many opportunities for automation to allow a more efficient and safer landing patterns. The constraints of air traffic control are complex and contain many safety rules. In this example we express these rules in MTL and demonstrate how the smoothed robustness is used to generate control strategies for safely and robustly landing two quadrotors. The proposed approach outperforms Simulated Annealing and gradient descent on the non-smooth robustness."
	
	Then you give the details below}
1. We take a quad-rotor model with linearized dynamics around hover, similar to those used in \cite{}. The case study involves centralized control of two quad-rotors, with operational objectives given as an MTL specification, and a constrained air-space.

1.b. Give model, constraints, specification. Shrinking horizon (fixing history) approach applicable (cite Vasu paper)

2. With the given specification, standard control approaches involving polyhedral constraints are hard to apply because of the temporal aspect of the eventually operator involved. While the two (if-then) altitude rules in the specification can be coded as polyhedral constraints on the set, it would result in a non-convex constraint set for positions. Similarly, the minimum distance between two quad-rotors can also be moved to the constraints but would result in another non-convex constraint if we choose to do so. In our formulation, the non-convexity remains in the cost-function while the constraints are linear.

3. For simulation purposes, we use obtain 3 initial trajectories (via solving different linear programs) from the given initial state to the terminal (landing) set. These three trajectories, each of which has negative robustness (i.e. does not satisfy the given specification), serve as three different initial solutions to A) Our approach B) SQP using the actual robustness function as the cost, C) Simulated Annealing with the actual robustness function as the cost. This multi-start approach can be used in practice when there is a fast initial trajectory generator available.

3.b. parameters for simulation annealing and citation for it.

4. Fig shows the initial trajectories for both the quadrotors in the given air-space. Fig. shows the three trajectories obtained after applying our control method, with the three initial trajectories as starting points for the optimization, respectively. 

5. Table shows for the three trajectories, initial robustness, robustness for trajectory obtained via the three methods (and the approximate robustness when applicable). In addition, we also tested out simulated annealing with the smooth robustness function in the cost (with the first initial trajectory), resulting in a trajectory with a final cost of blah (approximate robustness of blah).



\begin{figure}[t]
\centering
\includegraphics[width=0.49\textwidth]{figures/QuadInitTrajs_scissored}
\caption{The airspace with the corresponding sets, and initial trajectories for the two quad-rotors. Note, all 3 initial trajectories violate the specification. Here, $p0_{i}^j$ refers to the positions of the $i^{th}$ initial trajectory for the $j^{th}$ quadrotor.}
\label{fig:quad_init}
\end{figure}

\begin{figure}[t]
\centering
\includegraphics[width=0.49\textwidth]{figures/QuadTrajs_scissored}
\caption{ Trajectories obtained via SQP on smooth robustness, with three different initial trajectories acting as initial solutions for the SQP. Note, all 3 trajectories satisfy $\Psi$. Here, $p_{i}^j$ refers to the positions of the $i^{th}$ initial trajectory for the $j^{th}$ quadrotor.}
\label{fig:quad_init}
\end{figure}


\subsection{Discussion}
%\todo[inline]{skipping fr now}
With two case studies on dynamic systems, we show the applicability and consistently good performance of our method, SR-SQP, which outperforms both SA and R-SQP. For every instance we covered, SR-SQP finds trajectories that satisfy the specification, while the other two methods do not always do so.

%Also, while in the first case study, R-SQP, in the second case study, simulated annealing cannot find trajectories that satisfy the specification from two of the three initial trajectories (used as initial solutions for optimization). Our method on the other hand, successfully finds a trajectory that satisfies the specification, while resulting in the best robustness value achieved across all examples considered. 

While we solve the control problem in a single-shot, finite horizon manner, in general, for a real-time implementation, the problem can be solved in a receding horizon manner (similar to \cite{PantAMNDM15_Anytime}, \cite{Jain2016}). Or, it can be solved in a manner where the state and actions of the past are stored an added as constraints at each time step while the look-ahead horizon of the optimization shrinks (similar to \cite{Raman14_MPCSTL}). This will be explored further in future work. We have shown previously \cite{PantAMNDM15_Anytime} that control of an actual quad-rotor with the dynamics in \eqref{eq:quad_dyn} is possible on a low computation power platform. The control algorithm there involved solving multiple quadratic programs at even higher sampling rates ($20Hz$), in a receding horizon manner. Future work will include a C implementation of SR-SQP, which will allow us to experiment on real platforms, like the aforementioned quad-rotors.
%Since all implementations were done in MATLAB, the focus was not real-time applicability of the proposed method. For the first case study, where the dynamics are slow, our method should still be applicable ($\sim 20s$) of execution time (compared to $\sim 5 \text{ mins}$ for SA). For the second case study a MATLAB implementation is infeasible due to the very fast dynamics and sampling times involved. We have shown previously in \cite{PantAMNDM15_Anytime} that control of a real quad-rotor with the dynamics in \eqref{eq:quad_dyn} is possible on a low computation power platform while solving multiple quadratic programs at even higher sampling rates ($20Hz$). With this in mind, we expect a C/C++ implementation of SQP (and the smooth robustness function) should allow us to implement our method on such a system. Ongoing work focuses on a general interpreter for formulae to generate corresponding smooth robustness functions, as well as their derivatives.