\subsubsection{A Non-linear example: Unicycle}
\label{sec:nl_unicycle}
Since SQP can handle non-linear (twice differentiable) constraints, our method can also deal with non-linear dynamics whereas the MILP-based methods have to linearize dynamics to solve the system. 
The following example shows SOP applied in a one-shot manner to the unicycle dynamics ($\dot{x}_t=v_t \cos (\theta_t)\, ,\dot{y}_t=v_t \sin (\theta_t)\, ,\dot{\theta}_t= u_t$) discretized at 10Hz.
%\begin{subequations}
%\label{eq:unicycle}
%\begin{align}
%\dot{x}_t &=v_t \cos (\theta_t) \nonumber \\ 
%\dot{y}_t &=v_t \sin (\theta_t) \nonumber \\
%\dot{\theta}_t &= u_t
%\end{align}
%\end{subequations}
For the specification of Ex. \ref{ex:toyproblem}, the resulting trajectory of length 20 steps obtained by SOP (R) (in Robust mode) is shown in fig.\ref{fig:toy control}, starting from an initial state of $[-2,-2,0]$. The resulting robustness is $0.248$, which is close to the global optimum of $0.25$. This shows that SOP can indeed handle non-linear dynamics without the need for explicit linearization as is done in MILP-based methods.
