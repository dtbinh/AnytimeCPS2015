\subsection{The need for temporal logic requirements}
\label{sec:morari}
\todo[inline]{Morari defense. develop}
The \textit{robustness of Metric Temporal Logic specifications} \cite{Fainekos2006_TLVerifSimu,Donze2010} is a rigorous notion that has been used successfully for the verification of automotive systems \cite{Fainekos12_Automotive,Dreossi15_RRTFalsification}, medical devices \cite{SankaranarayananF2012cmsb}, and general CPS.
In details, MTL is a formal language for expressing complex reactive requirements with time constraints, such as those of the ATC~\cite{Koymans90}.
Given a specification $\formula$ written in Metric Temporal Logic (MTL) and a system execution $\sstraj$, the robustness $\robf(\sstraj)$ of the spec relative to $\sstraj$ measures two things:
its sign tells whether $\sstraj$ satisfies the spec ($\robf(\sstraj) > 0$) or falsifies it (i.e., violates it, $\robf(\sstraj) <0$).
Its magnitude $|\robf(\sstraj)|$ measures how \textit{robustly} the spec is satisfied or falsified.
Namely, any perturbation to $\sstraj$ of size less than $|\robf(\sstraj)|$ will not cause its truth value to change.
Thus, the control algorithm can \textit{maximize} the robustness over all possible control actions to determine the next control input.