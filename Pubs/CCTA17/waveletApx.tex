%\underline{Wavelet approximations of the distance function}

\textbf{Wavelet approximations of the distance function}. Wavelets can be viewed as a generalization of the kernel $g_\varepsilon$ of the previous section to a whole family of kernels, obtained by translations and dilations of one function called the `mother wavelet' $\psi$: $\psi_{j,k}(x) = 2^{j/2}\psi(2^jx - k)$  \cite{MallatBook}.
They are used extensively in signal processing, because their approximation properties can be tailored to the class of functions being approximated - e.g., hyperbolic wavelets are suitable for approximating functions of mixed smoothness, such as the distance function we are interested in~\cite{Heping04_HyperbolicWav}.
%In our case, since we know the singularity lines of $\dist$, we can use a directional wavelet basis, and orient it locally such that it has slow variation along the singularity line, and fast variation accross it.

In the experiments of this paper, we use a wavelet approximation to the distance function.
Specifically, we use
\[\sdist_\varepsilon(x,U) =  \sum_{\mathbf{(j,k)} \in D}c_{\mathbf{j,k}}\psi_{\mathbf{j,k}}(x)\]
which is a partial expansion of $\dist(\cdot,U)$ in the wavelet basis $\{\psi_{\mathbf{j,k}}, \mathbf{j} = (j_1\ldots,j_n) \in \Ze^n, \mathbf{k} = (k_1,\ldots,k_n) \in \Ze^n\}$.
The $c_{\mathbf{j,k}}$ coefficients are the result of projecting $\dist$ onto the wavelet basis
\[ c_{\mathbf{j,k}} = \int_{\Re^n}\dist(x,U)\psi_{\mathbf{j,k}}dx\]
By increasing the number of terms $|D|$ in the partial expansion, we improve the accuracy of the approximation.
The multi-dimensional wavelet we use is a tensor product of 1-D Meyer wavelets $\psi^{meyer}$ \cite{MallatBook}:
\begin{equation}
\label{eq:multid meyer}
\psi_{\mathbf{j,k}} = \psi_{j_1,k_1}^{meyer}\ldots \psi_{j_n,k_n}^{meyer}, \mathbf{j,k} \in \Ze^n
\end{equation}

 
