\section{Conclusions and future work}

From the profiling of the vanishing point algorithm running on the NVIDIA Jetson Tk1 with different resource allocations and clock frequencies of the CPU and GPU, we see that there is no trivial fixed operating point to run at if the computation power is taken into account. 
From preliminary experiments as highlighted in the previous sections, running all tasks on the GPU does not offer the best update rate (or fastest performance) from the vanishing point algorithm. This can partly be explained by profiling the code for the vanishing point (e.g., we used gprof \cite{Graham:1982:GCG:800230.806987}), which offers the insight that conversions of the OpenCV Mat type (which stores the image and intermediate results) and the OpenCV Gpu:Mat are a time consuming process. 
We also hypothesize that the OpenCV GPU implementations of the functions may not be optimized. 
Looking at Fig. \ref{fig:dfsa}, it is clear that the most benefit of the GPU can be obtained by scheduling only the Hough transform to run on it (configuration CCG). 
While Fig. \ref{fig:dfsa_pow} shows that this allocation is also power-efficient among nearly all frequencies of the CPU and GPU, its power consumption is close enough to the other modes that different weights on the power consumption could lead to a different operating point (CPU-GPU allocation, frequencies) being selected. 
Further experiments will be conducted with more knobs to study this in more detail.

Given the profiling information, on-going work is also focused on developing a supervisory algorithm that decides at run-time what mode the perception algorithm should operate in based on the closed loop performance of the system and the computation power consumption. Since in our system, a separate battery pack is used for the Jetson and the sensors (and a separate pack for the motors), this computation power aware supervisory algorithm would possibly lead to improved system lifetime. We are also developing a model predictive controller that can leverage the time/power/quality trade-off while guaranteeing safe operation of the closed loop system (e.g., respecting constraints on position and velocity). A more experimental approach also being looked at is switching between multiple PID controllers (each tuned to one mode of the estimation algorithm) based on the error signal from the vanishing point algorithm, which is the only feedback in the simple corridor navigation system which was considered in this work.
