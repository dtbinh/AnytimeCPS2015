
The profiling results indicate that the knobs $\sigma, F_c$ and $F_g$ allow us to trade-off throughput for power, as expected.
At runtime, we must decide which knob setting to choose at every time step. 
This is done by maximizing the following objective function at every time step $t$:

\begin{equation}
\max_{\sigma,F_{c},F_{g}} \alpha(x_m,x_v)\mathbf{\bar{T}}(\sigma,F_{c},F_{g}) + \frac{1-\alpha(x_m,x_v)}{\mathbf{E[\bar{P}]}(\sigma,F_{c},F_{g})}
\label{eq:cost_runtime}
\end{equation}

Recall that $\mathbf{\bar{T}}(\sigma,F_{c},F_{g})$ is the normalized throughput of VP, and $\mathbf{E[\bar{P}]}(\sigma,F_{c},F_{g})$ is the mean normalized power consumed by the computation platform.
The parameter $\alpha(x_m,x_v) \in [0,1]$ determines how much to weigh throughput versus performance at every time step. 
It is computed as a function of the abscissa $x_m,x_v$: recall that non-zero values of either indicates deviation from the center line of the corridor (Section \ref{sec:vp}). 
Since $x_m,x_v$ are time-varying, $\alpha$ is also time-varying.
As $x_m$ or $x_v$ deviates further away from 0, $\alpha$ increases to more heavily weigh throughput, thus skewing the optimization towards larger throughput and better control performance.
This translates as different CPU-GPU schedules and different frequencies.
In this paper, we use three different functions for $\alpha$, where $d >0$:

{\footnotesize{
		\begin{equation}
		\alpha= f_1(x_v(t)) = \\
		\begin{cases}
		0.001,&\text{if }x_v(t)\in[-d,d]\\
		x_v(t)+d,&\text{if }x_v(t)<-d\\
		x_v(t)-d,&\text{if }x_v(t)>d
		\end{cases}
		\label{eq:f1}
		\end{equation}
	}}
	
	{\footnotesize{
			\begin{equation}
			\alpha = f_2(x_m(t)) = \\
			\begin{cases}
			0.001,&\text{if }x_m(t)\in[-d,d]\\
			x_m(t)+d,&\text{if }x_m(t)<-d\\
			x_m(t)-d,&\text{if }x_m(t)>d
			\end{cases}
			\label{eq:f2}
			\end{equation}
		}}
		
{\footnotesize{
		\begin{equation}
		\alpha = f_3(x_m(t),x_v(t)) \\
		\begin{cases}
		0.001,\text{if }|x_m(t)|+|x_v(t)|<d\\
		|x_m(t)|+|x_v(t)|-d,\text{otherwise}\\
		\end{cases}
		\label{eq:f3}
		\end{equation}
	}}