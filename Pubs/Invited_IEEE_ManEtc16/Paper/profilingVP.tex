

For the perception algorithm, the first stage of our method is profiling the performance (timing and, if available, quality) and power consumption of the computation. With the vanishing point algorithm, we can execute the Blur, the Edge detection and the Hough transform on either the CPU or the GPU (Fig. \ref{fig:vanishing}). RANSAC runs fast enough to not have a significant impact on the total execution time, so we do not consider running it on the GPU. Execution on the GPU results, in general, in a speed-up over the CPU but at the cost of higher power draw from the Jetson. Additionally, on the Jetson, we can control the performance of the CPU and GPU by changing the clock frequencies at which they operate. This gives us multi-dimensional knobs on the hardware level that we can control to trade-off computation speed and power consumption.

To do this profiling offline, we first navigate the robot manually in corridors and log video from the on-board camera at a high frame-rate (60Hz). 
We run Vanishing point algorithm on this video offline and profile it with different scheduling of the three components (Blur, Canny and Hough) on the CPU and GPU, and at different frequencies of both processors.

We wrote a custom C-code library to log power measurements from a Tektronix PWS4205 Programmable DC power supply at 100Hz. 
For this we communicate with the power supply over USB using the USB Test and Measurement Class (USB-TMC) communication protocol. 
 
Since for an algorithm like Vanishing point there is no well-defined notion of ground truth, we do not have a measure of accuracy of the algorithm. 
Instead, Vanishing point's update rate (which is the inverse of the computation time) is used as a performance measure, since with faster updates the controller has less delay, resulting in better control performance. 

\subsubsection{Experimental results for profiling}
%profilingresults
Figure \ref{fig:sfda} shows the profiling results for the throughput of 
VP at different CPU-GPU allocations of the 3 tasks and different frequencies of the CPU and the GPU. 
Note, the CPU can be clocked upto 2.32 GHz (on all 4 cores), while the GPU can be clocked upto 0.852 GHz. 
We select 6 operating frequencies evenly spaced between the minimum and maximum CPU and GPU frequencies. 
\begin{figure}[htbp]
	\centering
	\includegraphics[width=0.46\textwidth]{Figs/surf_Rate.pdf}
	\caption{VP throughput. Each surface corresponds to a particular schedule (see legend). For a given schedule, different CPU and GPU frequencies yields a different throughput. (Color in online version).}
	\label{fig:sfda}%same freq diff assignment}
\end{figure}

%\begin{figure}[hbtp]
%\centering
%\includegraphics[scale=0.3]{Figs/RateHist.pdf}
%\caption{Control update rate for different frequencies and a given CPU-GPU assignment. For clarity we only consider 3 CPU and GPU frequencies for this %figure, ranging from the minimum to the maximum of frequencies of CPU and GPU. (Color in online version) }
%\label{fig:dfsa} %diff freq same assignment}
%\end{figure}

Figure \ref{fig:sfda_pow} shows the profiling of average power consumed by VP over all frames in the video for the same combinations of the knobs.


\begin{figure}[t]
	\centering
	\includegraphics[width=0.46\textwidth]{Figs/surf_Power.pdf}
	\caption{Mean power consumed by the Jetson. Each surface corresponds to a particular schedule (see legend). For a given schedule, different CPU and GPU frequencies yields a different power. (Color in online version).}
	\label{fig:sfda_pow}%same freq diff assignment}
\end{figure}

%\begin{figure}[htbp]
%\centering
%\includegraphics[width=0.46\textwidth]{Figs/PowerHist.pdf}
%\caption{Mean power consumed by the Jetson for different frequencies and a given CPU-GPU assignment.  For clarity we only consider 3 CPU and GPU frequencies for this figure, ranging from the minimum and maximum of both the CPU and the GPU. (Color in online version)}
%\label{fig:dfsa_pow} %diff freq same assignment}
%\end{figure}
