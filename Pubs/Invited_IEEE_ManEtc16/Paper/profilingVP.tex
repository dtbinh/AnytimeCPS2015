For the Vanishing Point (VP) algorithm, the first stage of our method is profiling the timing and power consumption of the computation.
Namely, we vary the schedule and frequency knobs described in Section \ref{sec:knobs}.
For each value of the knobs, we run VP on a video sequence previously acquired by the robot while navigating a corridor. 
We wrote a custom C-code library to log power measurements from a Tektronix PWS4205 Programmable DC power supply at 100Hz. 
For this we communicate with the power supply over USB using the USB Test and Measurement Class (USB-TMC) communication protocol. 
 
Since for an algorithm like VP there is no well-defined notion of output quality, we use the update rate as a performance measure, since faster updates mean that the car controller has less delay, resulting in better control performance. 

\begin{figure}[t]
	\centering
	\includegraphics[width=0.46\textwidth]{Figs/bigFig.pdf}
	\caption{Two stage approach.}
	\label{fig:juicyj}%same freq diff assignment}
\end{figure} 

%\subsubsection{Experimental results for profiling}
%profilingresults
Figure \ref{fig:sfda} shows the profiling results for the throughput of 
VP at different CPU-GPU allocations of the 3 tasks and different frequencies of the CPU and the GPU. 
Note, the CPU can be clocked upto 2.32 GHz (on all 4 cores), while the GPU can be clocked upto 0.852 GHz. 
We select 6 operating frequencies evenly spaced between the minimum and maximum CPU and GPU frequencies. 
\begin{figure}[htbp]
	\centering
	\includegraphics[width=0.46\textwidth]{Figs/surf_Rate.pdf}
	\caption{VP throughput. Each surface corresponds to a particular schedule (see legend). For a given schedule, different CPU and GPU frequencies yields a different throughput. (Color in online version).}
	\label{fig:sfda}%same freq diff assignment}
\end{figure}

%\begin{figure}[hbtp]
%\centering
%\includegraphics[scale=0.3]{Figs/RateHist.pdf}
%\caption{Control update rate for different frequencies and a given CPU-GPU assignment. For clarity we only consider 3 CPU and GPU frequencies for this %figure, ranging from the minimum to the maximum of frequencies of CPU and GPU. (Color in online version) }
%\label{fig:dfsa} %diff freq same assignment}
%\end{figure}

Figure \ref{fig:sfda_pow} shows the profiling of average power consumed by VP over all frames in the video for the same combinations of the knobs.


\begin{figure}[t]
	\centering
	\includegraphics[width=0.46\textwidth]{Figs/surf_Power.pdf}
	\caption{Mean power consumed by the Jetson. Each surface corresponds to a particular schedule (see legend). For a given schedule, different CPU and GPU frequencies yields a different power. (Color in online version).}
	\label{fig:sfda_pow}%same freq diff assignment}
\end{figure}

%\begin{figure}[htbp]
%\centering
%\includegraphics[width=0.46\textwidth]{Figs/PowerHist.pdf}
%\caption{Mean power consumed by the Jetson for different frequencies and a given CPU-GPU assignment.  For clarity we only consider 3 CPU and GPU frequencies for this figure, ranging from the minimum and maximum of both the CPU and the GPU. (Color in online version)}
%\label{fig:dfsa_pow} %diff freq same assignment}
%\end{figure}
