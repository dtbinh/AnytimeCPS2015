%\subsection{Experimental Setup}
%
%For our experiments, we converted a Radio controlled Traxxas Rally car into an autonomous robot shown in Fig. \ref{fig:traxxas}, similar to the ones used in \cite{racecar_mit}. The computation platform is a NVIDIA Jetson TK1. The Jetson has a quad-core ARM Cortex-A15 CPU and a NVIDIA Kepler GPU. 
%This setup allows us to schedule tasks on the CPU or GPU while observing the effect of this on power consumption and timing of the algorithm. 
%The Jetson runs Ubuntu for Tegra as the operating system, and the algorithms are implemented using ROS \cite{ros}. The control signals for the drive and steer motor on the platform are generated by a Teensy 3.1 microcontroller running a ROS node which converts the continuous output of the control software to a PWM signal that acts as an input to the motor controller on the Traxxas. Finally, the vanishing point algorithm implemented in OpenCV \cite{opencv} gets images from a front facing Point Grey Firefly MV camera capable of recording color images at upto a resolution of 752x480 pixels and upto a frame rate of 60 FPS. 

The online control procedure performs the following calculations at ever time step $t$ (see Fig. \ref{fig:juicyj}):
\begin{enumerate}
	\item Obtain $x_m,x_v$ from VP, and provide it to both Supervisor and Controller.
	\label{begin}
	\item The Supervisor:
	\begin{enumerate}
		\item Computes $\alpha(x_m,x_v)$
		\item For each value of $(\sigma,F_c,F_g)$, computes the objective value \eqref{eq:cost_runtime}. $\bar{P}$ and $\mathbf{E[\bar{P}]}$ are obtained from the offline profiling stage.
		\item Selects the value of $(\sigma^*,F_c^*,F_g^*)$ that maximizes the objective. This is provided to VP.
	\end{enumerate}
	\item The controller computes the input value as described in the next subsection.
	\item VP executes with the CPU/GPU schedule $\sigma^*$ and frequencies $F_c^*,F_g^*$. Goto \ref{begin}.
\end{enumerate}







