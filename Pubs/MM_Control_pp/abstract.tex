\begin{abstract}
Modern control systems, like controllers for swarms of quadrotors, must satisfy complex control objectives while withstanding a wide range of disturbances, from bugs in their software to attacks on their sensors and changes in their environments.
These requirements go beyond stability and tracking, and involve temporal and sequencing constraints on system response to various events, and can be expressed succinctly and formally in Metric Temporal Logic (MTL).
%Previous work formalized the requirements as formulas in Metric Temporal Logic (MTL), and designed a controller for linear systems that solves a Mixed Integer Linear Program (MILP) to compute control inputs.
%This approach has high computational cost and its runtimes are hard to predict.
This work develops an efficient, one-shot, gradient-descent algorithm to satisfy a given bounded-horizon MTL requirement.
If the requirement's horizon exceeds the computational capabilities of the controller, a receding horizon controller is developed that is recursively feasible.
When the requirement is composed of several sub-formulas of different importance, we are developing an algorithm for run-time trade-off of requirements.


% and designs a controller that maximizes the \textit{robustness} of the MTL formula.
%Formally, if the system satisfies the formula with robustness $r$, then any disturbance %of size less than $r$ cannot cause it to violate the formula.
%Because robustness is not differentiable, this work provides arbitrarily precise, %infinitely differentiable, approximations of it, thus enabling the use of powerful gradient descent optimizers.
%Experiments on a temperature control example and a two-quadrotor system demonstrate that this approach to controller design outperforms existing approaches to robustness maximization based on Mixed Integer Linear Programming and stochastic heuristics.
%Moreover, it is not constrained to linear systems.
\end{abstract}
