%%%%%%%%%%%%%%%%%%%%%%%%%%%%%%%%%%%%%%%%%%%%%%%%%%%%%%%%%%%%%%%%%%%%%%%%%%%%%%%%
%2345678901234567890123456789012345678901234567890123456789012345678901234567890
%        1         2         3         4         5         6         7         8

%\documentclass[final,letterpaper, 10 pt, conference]{ieeeconf}  % Comment this line out if you need a4paper

%\documentclass[10pt, conference, compsocconf]{IEEEtran}
\documentclass[a4paper, 10pt, conference]{ieeeconf}      % Use this line for a4 paper

\IEEEoverridecommandlockouts                              % This command is only needed if
                                                          % you want to use the \thanks command

\newcommand{\eu}{\textrm{e}}
\newcommand{\figref}[1]{Fig.~\ref{fig:#1}}
\overrideIEEEmargins
\usepackage[obeyFinal]{todonotes}
\RequirePackage{graphicx}
\usepackage{subfigure}
\usepackage{bm}
%\usepackage{subequation}
\graphicspath{ {figures/} }
\usepackage{url}
%\usepackage[tight,footnotesize]{subfigure}
\usepackage{fainekos-macros}
\newcommand{\hatodo}[1]{\todo{HA: #1}}
\newcommand{\hatodoin}[1]{\todo[inline]{HA: #1}}
\newcommand{\sAccu}{\epsilon}
\newcommand{\sDelay}{\delta}
\newcommand{\de}{(\sDelay,\sAccu)}
\newcommand{\dek}[1]{(\sDelay_{#1},\sAccu_{#1})}
\newcommand{\hWc}{\widehat{\Wc}}
\newcommand{\sDelayV}{\underline{\sDelay}}
\newcommand{\sAccuV}{\underline{\sAccu}}
\newcommand{\ESet}{\mathcal{E}}
%\newcommand{\bm}{\hat{B}}
\newcommand{\MPCProb}[1]{\mathbb{P_{#1}}}
\newcommand{\RAMPCProb}[2]{\mathbb{P}_{#1}(\hat{\stPt}_{#2},\sDelay_{#2},\sAccu_{#2},\inpPt_{#2 -1})}
\global\long\def\ZSet{\Zc}
\global\long\def\Cc{\mathcal{C}}
\global\long\def\Nom#1{\overline{#1}}
\newcommand{\bla}[1]{\overbar{#1}}
\newcommand{\bli}[1]{\overline{BLI #1}}
%\newboolean{TECH_REPORT}
%\setboolean{TECH_REPORT}{FALSE}
\newcommand{\Given}{|}

\usepackage{array}
\usepackage{url}

%\newcommand\Mark[1]{\textsuperscript#1}


% See the \addtolength command later in the file to balance the column lengths
% on the last page of the document

% The following packages can be found on http:\\www.ctan.org
%\usepackage{graphics} % for pdf, bitmapped graphics files
%\usepackage{epsfig} % for postscript graphics files
%\usepackage{mathptmx} % assumes new font selection scheme installed
%\usepackage{times} % assumes new font selection scheme installed
\usepackage{amsmath} % assumes amsmath package installed
\usepackage{amssymb}  % assumes amsmath package installed
%\begin{document}
\title{\LARGE \bf
Poster Abstract: Co-Design of Anytime Computation and Robust Control
}
%
\author{ Nischal K.N., Paritosh Kelkar, Dhurva Kumar, Yash Vardhan Pant, Houssam Abbas, Joseph Devietti, Rahul Mangharam% <-this % stops a space
\thanks{*This work was supported by STARnet a Semiconductor Research
Corporation program sponsored by MARCO and DARPA, NSF MRI-0923518 and the US Department of Transportation University Transportation Center Program}% <-this % stops a space
\thanks{The Departments of Electrical and Systems Engineering and Computer and Information Sciences, University of Pennsylvania, Philadelphia, U.S.A.
        {\small
        \{yashpant,habbas,kmohta,nghiem,rahulm\}@seas.upenn.edu, devietti@cis.upenn.edu}}%
}


%\author{\IEEEauthorblockN{Yash Vardhan Pant, Houssam Abbas, Kartik Mohta, Rahul Mangharam}
%\IEEEauthorblockA{The Department of Electrical  \\ and Systems Engineering, \\
%University of Pennsylvania, \\
%Philadelphia, U.S.A\\
%yashpant@seas.upenn.edu, kmohta@seas.upenn.edu, \\habbas@seas.upenn.edu, rahulm@seas.upenn.edu}
%\and
%\IEEEauthorblockN{Joseph Devietti}
%\IEEEauthorblockA{The Department of Computer\\  and Information Sciences, \\
%University of Pennsylvania, \\
%Philadelphia, U.S.A\\
%devietti@cis.upenn.edu}
%}




\begin{document}

\maketitle
\thispagestyle{empty}
\pagestyle{empty}

\bibliographystyle{IEEEtran}%abbrv}
	\bibliography{IEEEabrv,rtss2015,cdc14,anytime_ref}

	%\section*{Appendix}
\label{appendix}

In this appendix we give the detailed mathematical derivation of the results of Section \ctrlProbSecRef.
The controller is designed using a Robust Model Predictive
Control (RMPC) approach via constraint restriction \cite{richardsetal05rmp, chiscietal01swp}, 
and augments it by an adaptation to the error-delay curve of the estimator.
In order to ensure robust safety and feasibility, the key idea of
the RMPC approach is to tighten the constraint sets iteratively to account
for possible effect of the disturbances. 
As time progresses, this ``robustness
margin'' is used in the MPC optimization with the nominal dynamics,
i.e., the original dynamics where the disturbances are either removed
or replaced by nominal disturbances.
%An advantage of this approach is that, 
Because only the nominal dynamics are used, the complexity of the optimization is the same as for the nominal problem.

Since the controller only has access to the estimated state $\hat{x}$, we need
to rewrite the plant's dynamics with respect to $\hat{x}$. 
The error
between $ $$x_{k}$ and $\hat{x}_{k}$ is $e_{k}=x_{k}-\hat{x}_{k}$.
At time step $k+1$ we have
\begin{align*}
\hat{x}_{k+1} & =x_{k+1}-e_{k+1}\\
 & =Ax_{k}+B_{1}(\sDelay_k)u_{k-1}+B_{2}(\sDelay[k])u_{k}+w_{k}-e_{k+1}\text{,}
\end{align*}
 then, by writing $x_{k}=\hat{x}_{k}+e_{k}$, we obtain the dynamics
\begin{equation}
\hat{x}_{k+1}=A\hat{x}_{k}+B_{1}(\sDelay[k])u_{k-1}+B_{2}(\sDelay[k])u_{k}+\hat{w}_{k}\label{eq:estimator-dynamics}
\end{equation}
 where $\hat{w}_{k}=w_{k}+Ae_{k}-e_{k+1}$.
The set of possible values of $\hat{w}_{k}$
depends on the estimation accuracy at steps $k$ and $k+1$ and is
denoted by $\hWc(\sAccu[k],\sAccu[k+1])$, i.e.,
$\hWc(\sAccu,\sAccu')\defeq \left\{ w+Ae-e'\sut w\in\Wc,e\in\ESet(\sAccu),e'\in\ESet(\sAccu')\right\}$.
Note that %we assume
$\hWc(\sAccu[k],\sAccu[k+1])$ is independent
of the time step $k$. %
It can be computed as $\hWc(\sAccu,\sAccu')=\Wc\oplus A\ESet(\sAccu)\oplus\left(-\ESet(\sAccu')\right)$
where the symbol $\oplus$ denotes the Minkowski sum of two sets.

The dynamics in \eqref{eq:estimator-dynamics} has a non-standard form
where it depends on both the current and the previous control inputs.
However we can expand the state variable to store the previous control
input as
\[
\hat{z}_{k}=\begin{bmatrix}\hat{x}_{k}\\
u_{k-1}
\end{bmatrix}\in\Re^{n+m}
\]
and rewrite the dynamics as, for all $k\geq0$,
\begin{equation}
\hat{z}_{k+1}=\hat{A}(\sDelay_k)\hat{z}_{k}+\hat{B}(\sDelay_k)u_{k}+\hat{F}\hat{w}_{k}\text{.}\label{eq:estimator-std-dynamics}
\end{equation}
Here, the system matrices are
\begin{equation}
\begin{gathered}
\hat{A}(\sDelay_k)=\begin{bmatrix}A & B_{1}(\sDelay_k)\\
\bm{0}_{m\times n} & \bm{0}_{m\times m}
\end{bmatrix},\\
\hat{B}(\sDelay_k)=\begin{bmatrix}B_{2}(\sDelay_k)\\
\IdentityMatrix_{m}
\end{bmatrix},\quad\hat{F}=\begin{bmatrix}\IdentityMatrix_{n}\\
\bm{0}_{m\times n}
\end{bmatrix}\text{.}
\end{gathered}
\label{eq:lifted-matrices}
\end{equation}

Let the actual expanded state be $z_{k}=\left[x_{k}^{T},u_{k-1}^{T}\right]^{T}$.
Because the expanded state consists of both the plant's state and
the previous control input, the state constraint $x_{k}\in\stSet$
and the control constraint $u_{k}\in\inpSet$ are equivalent to the
joint constraint $z_{k}\in\stSet\times\inpSet$. We can now describe
the RAMPC algorithm for the dynamics in \eqref{eq:estimator-std-dynamics}.

\subsection{Tractable RAMPC Algorithm}

Let $N\geq1$ be the horizon length of the RMPC optimization. 
Because the system
matrices in the state equation~(\ref{eq:estimator-std-dynamics})
depend nonlinearly on the variables $\sDelay_k$, the RMPC optimization
is generally a mixed-integer nonlinear program, which is very hard
to solve. To simplify the RMPC optimization to make it tractable, we fix the estimation mode for the entire RMPC horizon.

Let $\RAMPCProb{\de}{k}$
denote the RMPC optimization problem at step $k\geq0$ where the current
state estimate is $\hat{x}_{k}$, the current estimation mode is $(\sDelay_k,\sAccu_k)\in\Delta$,
the previous control input is $u_{k-1}$, and the estimation mode
for the entire horizon (after step $k$) is fixed at $(\sDelay,\sAccu)\in\Delta$.
Since the system matrices become constant now, if the stage cost $\ell(\cdot)$
is linear or positive semidefinite quadratic, each optimization problem
$\RAMPCProb{\de}{\cdot}$ is tractable and can be solved
efficiently as we will show later. 
The RAMPC algorithm with Anytime Estimation is stated in Alg. \algoref.

\subsection{RMPC Formulation}

We formulate the RMPC optimization $\RAMPCProb{\de}{k}$
with respect to the nominal dynamics, which is the original dynamics
in \eqref{estimator-std-dynamics} but the disturbances are either
removed or replaced by nominal disturbances. 
To ensure robust feasibility
and safety, the state constraint set is tightened after each step
using a candidate stabilizing state feedback control, and a terminal
constraint is derived. 
In this RMPC formulation, we extend the approach
in \cite{richardsetal05rmp, chiscietal01swp}. 
At time step $k$, given
$(\hat{x}_{k},\sDelay_k,\sAccu_k,u_{k-1})$ and for a fixed $(\sDelay,\sAccu)$,
we solve the following optimization 

\begin{subequations}
	\label{eq:RMPC1}
 \begin{equation} J_{\sDelay,\sAccu}^{*} \left(\hat{x}_{k},\sDelay_k,\sAccu_k,u_{k-1}\right) = \min_{\boldsymbol{u},\boldsymbol{x}}\sum_{j=0}^{N}\ell(\Nom x_{k+j\Given k},u_{k+j\Given k})
 \end{equation}
 \begin{equation}
  \text{subject to, }\forall j\in\left\{ 0,\dots,N\right\} \nonumber 
 \end{equation}
 \begin{equation}
  \Nom z_{k+j+1\Given k}=\hat{A}(\sDelay_{k+j\Given k})\Nom z_{k+j\Given k}+\hat{B}(\sDelay_{k+j\Given k})u_{k+j\Given k}\label{eq:RMPC1-dyn}
 \end{equation}
 \begin{equation}
  ( \sDelay_{k+j+1\Given k},\sAccu_{k+j+1\Given k} ) \!=\! (\sDelay,\sAccu ) \nonumber
 \end{equation}
 \begin{equation}
  (\sDelay_{k\Given k},\sAccu_{k\Given k}) \!=\! (\sDelay_k,\sAccu_k)  \label{eq:RMPC1-delay}
 \end{equation}
 \begin{equation}
  \Nom x_{k+j\Given k}=\begin{bmatrix}\IdentityMatrix_{n} \quad \bm{0}_{n\times m}\end{bmatrix}\Nom z_{k+j\Given k}\label{eq:RMPC1-z2x}
 \end{equation}
 \begin{equation}
  \Nom z_{k\Given k}=\left[\hat{x}_{k}^{T},u_{k-1}^{T}\right]^{T} \label{eq:RMPC1-z0}
 \end{equation}
 \begin{equation}
  \Nom z_{k+j\Given k}\in\ZSet_{j}\left(\sAccu_k,\sAccu\right)\label{eq:RMPC1-zset}
 \end{equation}
 \begin{equation}
  \Nom z_{k+N+1\Given k}\in\ZSet_{f}\left(\sAccu_k,\sAccu\right) \label{eq:RMPC1-zfinalset}
  \end{equation}
\end{subequations} 

in which $\Nom z$ and $\Nom x$
are the variables of the nominal dynamics. The constraints of the
optimization are explained below.
\begin{itemize}
\item \eqref{eq:RMPC1-dyn} is the nominal dynamics.
\item \eqref{eq:RMPC1-delay} states that the estimation mode is fixed at $\left(\sDelay,\sAccu\right)$
except for the first time step when it is $\left(\sDelay_k,\sAccu_k\right)$.
\item \eqref{eq:RMPC1-z2x} extracts the nominal state $\Nom x$ of the plant
from the nominal expanded state $\Nom z$.
\item \eqref{eq:RMPC1-z0} initializes the nominal expanded state at time step
$k$ by stacking the current state estimate and the previous control
input.
\item \eqref{eq:RMPC1-zset} tightens the admissible set of the nominal expanded
states by a sequence of shrinking sets.
\item \eqref{eq:RMPC1-zfinalset} constrains the terminal expanded state to
the terminal constraint set $\ZSet_{f}$.
\end{itemize}

\noindent\textit{The state constraint $\ZSet_{j}$:}
%
The tightened state constraint sets $\ZSet_{j}\left(\sAccu_k,\sAccu\right)$
are parameterized with two parameters $\sAccu_k$ and $\sAccu$.
They are defined as follows, for all $j\in\left\{ 0,\dots,N\right\} $
\begin{eqnarray}
\ZSet_{0}(\sAccu_k,\sAccu)=\ZSet\ominus\hat{F} \ESet(\sAccu_k)\label{eq:RMPC1-Z0}
\\
\ZSet_{j+1}(\sAccu_k,\sAccu)=\ZSet_{j}(\sAccu,\sAccu)\ominus L_{j}\hat{F}\hWc(\sAccu_k,\sAccu)\label{eq:RMPC1-Zj}
\label{eq:RMPC1-Z}
\end{eqnarray} 
in which the symbol $\ominus$
denotes the Pontryagin difference between two sets. The set $\ZSet$
combines the constraints for both the plant's state and the control
input: $\ZSet=\stSet\times\inpSet$. The matrix $L_{j}$ is the state
transition matrix for the nominal dynamics in \eqref{eq:RMPC1-dyn} under
a candidate state feedback gain $K_{j}(\sDelay)$, for $j\in\left\{ 0,\dots,N\right\}$
\begin{eqnarray}
\label{eq:RMPC1-L}
L_{0}=\IdentityMatrix\label{eq:RMPC1-L0}\\
L_{j+1}=(\hat{A}(\sDelay)+\hat{B}(\sDelay)K_{j}(\sDelay))L_{j}\label{eq:RMPC1-Lj}
\end{eqnarray}
Note that the possibly time-varying sequence $K_{j}(\sDelay)$ is designed for each choice of $\sDelay$ (i.e., the system matrices $\hat{A}(\sDelay)$ and $\hat{B}(\sDelay)$), hence $L_{j}$ depends on $\sDelay$; however we write $L_{j}$ for brevity. The candidate control $K_{j}(\sDelay)$ is designed to stabilize the nominal system (\ref{eq:RMPC1-dyn}), desirably as fast as possible so that the sets $\ZSet_{j}$ are shrunk as little as possible. In particular, if $K_{j}(\sDelay)$ renders the nominal system nilpotent after $M<N$ steps then $L_{j}=\bm{0}$ for all $j\geq M$, therefore $\ZSet_{j}\left(\sAccu_k,\sAccu\right)=\ZSet_{M}\left(\sAccu_k,\sAccu\right)$ for all $j>M$.


\noindent\textit{The terminal constraint $\ZSet_{f}$:}
%
$\ZSet_{f}$ is given by %the Pontryagin difference
\begin{equation}
\label{eq:RMPC1-Zf}
\ZSet_{f}\left(\sAccu_k,\sAccu\right)=\Cc(\sDelay,\sAccu)\ominus L_{N}\hat{F}\hWc(\sAccu_k,\sAccu)
\end{equation}
where $\Cc(\sDelay,\sAccu)$ is a robust control invariant admissible
set for $\sDelay$ \cite{kerrigan00rcs}, i.e., there exists a feedback control law $u=\kappa(z)$
such that $\forall z\in\Cc(\sDelay,\sAccu)$ and $\forall w\in\hWc(\sAccu,\sAccu)$
\begin{eqnarray}
\label{eq:RMPC1-Zf-invariant}
& \hat{A}(\sDelay)z \!+\! \hat{B}(\sDelay)\kappa(z) \!+\! L_{N}\hat{F}w\in\Cc(\sDelay,\sAccu) \label{eq:RMPC1-Zf-invariant-dyn}\\
& z\in\ZSet_{N}\left(\sAccu,\sAccu\right)\label{eq:RMPC1-Zf-invariant-z}
\end{eqnarray}
We remark that $\Cc(\sDelay,\sAccu)$ does not depend on $\left(\sDelay_k,\sAccu_k\right)$, therefore it can be computed offline for each mode $\left(\sDelay,\sAccu\right)$.

\subsection{Proofs of Feasibility}
The RMPC formulation of the previous section, with a fixed estimation mode
$\left(\sDelay,\sAccu\right)\in\Delta$, is designed to ensure that the control problem is robustly feasible, as stated in the following theorem.
\begin{thm}
[Robust Feasibility of RAMPC]\label{thm:robust-feasible-RMPC} For
any estimation mode $\left(\sDelay,\sAccu\right)$, if $\RAMPCProb{\de}{k}$
is feasible then the system (\ref{eq:disc-dynamics}) controlled by
the RAMPC and subjected to disturbances constrained by $w_k \in \Wc$
robustly satisfies the state constraint $\stPt_k \in \stSet$
and the control input constraint $\inpPt_k \in \inpSet$, and
all subsequent optimizations $\MPCProb{\sDelay,\sAccu}(\hat{x}_{k},\sDelay[k],\sAccu[k],u_{k-1})$,
$\forall k>k_{0}$, are feasible.
\end{thm}
\begin{proof}
%
We will prove the theorem by recursion. We will show that if at any
time step $k$ the RMPC problem $\MPCProb{\sDelay,\sAccu}(\hat{x}_{k},\sDelay[k],\sAccu[k],u_{k-1})$
is feasible and feasible control input $u_{k}=u_{k\Given k}^{\star}$
is applied with estimation mode $\left(\sDelay[k+1],\sAccu[k+1]\right)=\left(\sDelay,\sAccu\right)$
then $u_{k}$ is admissible and at the next time step $k+1$, the
actual plant's state $x_{k+1}$ is inside $\stSet$ and the optimization
$\MPCProb{\sDelay,\sAccu}(\hat{x}_{k+1},\sDelay[k+1],\sAccu[k+1],u_{k})$
is feasible for all disturbances. Then we can conclude the theorem
because, by recursion, feasibility at time step $k_{0}$ implies robust
constraint satisfaction and feasibility at time step $k_{0}+1$, and
so on at all subsequent time steps.

Suppose $\MPCProb{\sDelay,\sAccu}(\hat{x}_{k},\sDelay[k],\sAccu[k],u_{k-1})$
is feasible. Then it has a feasible solution $\left(\{ \overline{z}_{k+j\Given k}^{\star}\} _{j=0}^{N+1},\{ u_{k+j\Given k}^{\star}\} _{j=0}^{N}\right)$
that satisfies all the constraints in \eqref{eq:RMPC1}. Now we will
construct a feasible candidate solution for $\MPCProb{\sDelay,\sAccu}(\hat{x}_{k+1},\sDelay[k+1],\sAccu[k+1],u_{k})$
at the next time step by shifting the above solution by one step.
Consider the following candidate solution for $\MPCProb{\sDelay,\sAccu}(\hat{x}_{k+1},\sDelay[k+1],\sAccu[k+1],u_{k})$:
\begin{subequations}
\label{eq:proofs:candidate-solution}
\begin{align}
\Nom z_{k+j+1\Given k+1} & =\Nom z_{k+j+1\Given k}^{\star}+L_{j}\hat{F}\hat{w}_{k}\label{eq:proofs:candidate-solution:zj}\\
\Nom z_{k+N+2\Given k+1} & =\hat{A}\left(\sDelay\right)\Nom z_{k+N+1\Given k+1}+\hat{B}\left(\sDelay\right)u_{k+N+1\Given k+1}\label{eq:proofs:candidate-solution:zN}\\
u_{k+i+1\Given k+1} & =u_{k+i+1\Given k}^{\star}+K_{i}\left(\sDelay\right)L_{i}\hat{F}\hat{w}_{k}\label{eq:proofs:candidate-solution:uj}\\
u_{k+N+1\Given k+1} & =\kappa\left(\Nom z_{k+N+1\Given k+1}\right)\label{eq:proofs:candidate-solution:uN}
\end{align}
\end{subequations} in which
$j\in\left\{ 0,\dots,N\right\} $, $i\in\left\{ 0,\dots,N-1\right\} $,
and $\kappa\left(\cdot\right)$ is the feedback control law for the
invariant set $\Cc(\sDelay,\sAccu)$ that is used in the terminal
set. We first show that the input and
state constraints are satisfied for $u_{k}$ and $x_{k+1}$, then
we will prove the feasibility of the above candidate solution for
$\MPCProb{\sDelay,\sAccu}(\hat{x}_{k+1},\sDelay[k+1],\sAccu[k+1],u_{k})$.

\noindent\textit{Validity of the applied input and the next state:}
%
The next plant's state is 
\begin{align*}
x_{k+1} & =Ax_{k}+B_{1}\left(\sDelay[k]\right)u_{k-1}+B_{2}\left(\sDelay[k]\right)u_{k}+w_{k}\\
 & =A\left(\hat{x}_{k}+e_{k}\right)+B_{1}\left(\sDelay[k]\right)u_{k-1}+B_{2}\left(\sDelay[k]\right)u_{k\Given k}^{\star}+w_{k}\\
 & =\begin{bmatrix}A & B_{1}\left(\sDelay[k]\right)\end{bmatrix}\begin{bmatrix}\hat{x}_{k}\\
u_{k-1}
\end{bmatrix}+B_{2}\left(\sDelay[k]\right)u_{k\Given k}^{\star} \\
&\qquad\qquad + e_{k+1}+\left(w_{k}+Ae_{k}-e_{k+1}\right)
\end{align*}
in which $e_{k+1}\in\ESet\left(\sAccu\right)$ and $\left(w_{k}+Ae_{k}-e_{k+1}\right)\in\hWc\left(\sAccu[k],\sAccu\right)$.
Note that $\Nom z_{k\Given k}^{\star}=\left[\hat{x}_{k}^{T},u_{k-1}^{T}\right]^{T}$.
Hence we have
\begin{align*}
\begin{bmatrix}x_{k+1}\\
u_{k}
\end{bmatrix} & =\hat{A}(\sDelay[k])\Nom z_{k\Given k}^{\star}+\hat{B}(\sDelay[k])u_{k\Given k}^{\star}\\
&\qquad\qquad +\hat{F}e_{k+1}+\hat{F}\left(w_{k}+Ae_{k}-e_{k+1}\right)\\
 & =\Nom z_{k+1\Given k}^{\star}+\hat{F}e_{k+1}+\hat{F}\left(w_{k}+Ae_{k}-e_{k+1}\right)
\end{align*}
where we use the dynamics in \eqref{eq:RMPC1-dyn}. From \eqref{eq:RMPC1-zset}
and \eqref{eq:RMPC1-Z}, $\Nom z_{k+1\Given k}^{\star}$ satisfies $\Nom z_{k+1\Given k}^{\star}\in\ZSet_{1}\left(\sAccu[k],\sAccu\right)=\ZSet\ominus\hat{F}\ESet\left(\sAccu\right)\ominus\hat{F}\hWc\left(\sAccu[k],\sAccu\right)$.
It follows that
\(
\left[ x_{k+1}^{T}, u_{k}^{T} \right]^{T} \in \ZSet = \stSet\times\inpSet\text{,}
\)
% which allows us to conclude that
therefore  $x_{k+1}\in\stSet$ and $u_{k}\in\inpSet$.


\noindent\textit{Initial condition:}
%
We have from \eqref{eq:estimator-std-dynamics} that $\hat{z}_{k+1}=\hat{A}(\sDelay[k])\hat{z}_{k}+\hat{B}(\sDelay[k])u_{k}+\hat{F}\hat{w}_{k}$.
On the other hand, by \eqref{eq:proofs:candidate-solution:zj},
\begin{align*}
\Nom z_{k+1\Given k+1} & =\Nom z_{k+1\Given k}^{\star}+L_{0}\hat{F}\hat{w}_{k}\\
 & =\hat{A}(\sDelay[k])\Nom z_{k\Given k}^{\star}+\hat{B}(\sDelay[k])u_{k\Given k}^{\star}+L_{0}\hat{F}\hat{w}_{k}\text{.}
\end{align*}
Note that $\Nom z_{k\Given k}^{\star}=\hat{z}_{k}$, $u_{k}=u_{k\Given k}^{\star}$,
and $L_{0}=\IdentityMatrix$. Therefore $\Nom z_{k+1\Given k+1}=\hat{z}_{k+1}$,
hence the initial condition is satisfied.


\noindent\textit{Dynamics:}
%
We show that the candidate solution satisfies the dynamics constraint
in \eqref{eq:RMPC1-dyn}. For $0\leq j<N$ we have
\begin{align*}
&\Nom z_{k+j+2\Given k+1} \\
=\, & \Nom z_{k+j+2\Given k}^{\star}+L_{j+1}\hat{F}\hat{w}_{k}\\
=\, &\hat{A}\left(\sDelay\right)\Nom z_{k+j+1\Given k}^{\star}+\hat{B}(\sDelay)u_{k+j+1\Given k}^{\star}+L_{j+1}\hat{F}\hat{w}_{k}\\
=\, &\hat{A}\left(\sDelay\right)\left(\Nom z_{k+j+1\Given k+1}-L_{j}\hat{F}\hat{w}_{k}\right) \\
&+\hat{B}(\sDelay)\left(u_{k+j+1\Given k+1}-K_{j}\left(\sDelay\right)L_{j}\hat{F}\hat{w}_{k}\right) +L_{j+1}\hat{F}\hat{w}_{k} \\
=\, &\hat{A}\left(\sDelay\right)\Nom z_{k+j+1\Given k+1}+\hat{B}(\sDelay)u_{k+j+1\Given k+1} \\
&-\left(\hat{A}\left(\sDelay\right) + \hat{B}(\sDelay)K_{j}\left(\sDelay\right)\right)L_{j}\hat{F}\hat{w}_{k}+L_{j+1}\hat{F}\hat{w}_{k}\\
=\, &\hat{A}\left(\sDelay\right)\Nom z_{k+j+1\Given k+1}+\hat{B}(\sDelay)u_{k+j+1\Given k+1}
\end{align*}
where the equality in \eqref{eq:RMPC1-Lj} is used to derive the last
equality. % from the previous one.
Therefore the dynamics constraint
is satisfied for all $0\leq j<N$. For $j=N$, the constraint is satisfied
by construction by \eqref{eq:proofs:candidate-solution:zN}.


\noindent\textit{State constraints:}
%
We need to show that $\Nom z_{(k+1)+j\Given k+1}\in\ZSet_{j}\text{\ensuremath{\left(\sAccu,\sAccu\right)}}$
for all $j\in\left\{ 0,\dots,N\right\} $. Consider any $0\leq j<N$.
\eqref{eq:RMPC1-Zj} states that $\ZSet_{j+1}\left(\sAccu[k],\sAccu\right)=\ZSet_{j}\left(\sAccu,\sAccu\right)\ominus L_{j}\hat{F}\hWc\left(\sAccu[k],\sAccu\right)$.
From the construction of the candidate solution we have $\Nom z_{k+j+1\Given k+1}=\Nom z_{k+j+1\Given k}^{\star}+L_{j}\hat{F}\hat{w}_{k}$,
where $\hat{w}_{k}\in\hWc\left(\sAccu[k],\sAccu\right)$ and $\Nom z_{k+j+1\Given k}^{\star}\in\ZSet_{j+1}\left(\sAccu[k],\sAccu\right)$.
By definition of the Pontryagin difference, we conclude that $\Nom z_{k+j+1\Given k+1}\in\ZSet_{j}\left(\sAccu,\sAccu\right)$
for all $j\in\left\{ 0,\dots,N-1\right\} $.

At $j=N$ the candidate solution in \eqref{eq:proofs:candidate-solution:zj}
gives us $\Nom z_{(k+1)+N\Given k+1}=\Nom z_{k+N+1\Given k}^{\star}+L_{N}\hat{F}\hat{w}_{k}$.
Because $\Nom z_{k+N+1\Given k}^{\star}\in\ZSet_{f}\left(\sAccu[k],\sAccu\right)=\Cc\left(\sDelay,\sAccu\right)\ominus L_{N}\hat{F}\hWc\left(\sAccu[k],\sAccu\right)$
and $\hat{w}_{k}\in\hWc\left(\sAccu[k],\sAccu\right)$, we have
$\Nom z_{(k+1)+N\Given k+1}\in\Cc\left(\sDelay,\sAccu\right)$. The
definition of $\Cc\left(\sDelay,\sAccu\right)$ in \eqref{eq:RMPC1-Zf-invariant}
implies $\Cc\left(\sDelay,\sAccu\right)\subseteq\ZSet_{N}\left(\sAccu,\sAccu\right)$.
Therefore $\Nom z_{(k+1)+N\Given k+1}\in\ZSet_{N}\left(\sAccu,\sAccu\right)$.


\noindent\textit{Terminal constraint:}
%
We need to show that $\Nom z_{k+N+2\Given k+1}\in\ZSet_{f}\left(\sAccu,\sAccu\right)=\Cc\left(\sDelay,\sAccu\right)\ominus L_{N}\hat{F}\hWc\left(\sAccu,\sAccu\right)$.
Add $L_{N}\hat{F}\hat{w}$, for any $w\in\hWc\left(\sAccu,\sAccu\right)$,
to both sides of \eqref{eq:proofs:candidate-solution:zN} and note that
$u_{k+N+1\Given k+1}=\kappa\left(\Nom z_{k+N+1\Given k+1}\right)$,
we have 
\begin{multline*}
  \Nom z_{k+N+2\Given
    k+1}+L_{N}\hat{F}\hat{w}=\hat{A}\left(\sDelay\right)\Nom
  z_{k+N+1\Given k+1} \\
  +\hat{B}\left(\sDelay\right)\kappa\left(\Nom
    z_{k+N+1\Given k+1}\right)+L_{N}\hat{F}\hat{w}\text{.}
\end{multline*}


 It follows from $\Nom z_{k+N+1\Given k+1}\in\Cc\left(\sDelay,\sAccu\right)$
and from the definition of the invariant control invariant admissible
set $\Cc\left(\sDelay,\sAccu\right)$ 
that $\Nom z_{k+N+2\Given k+1}+L_{N}\hat{F}\hat{w}\in\Cc\left(\sDelay,\sAccu\right)$
for all $w\in\hWc\left(\sAccu,\sAccu\right)$. Then by definition
of the Pontryagin difference, we conclude that $\Nom z_{k+N+2\Given k+1}\in\Cc\left(\sDelay,\sAccu\right)\ominus L_{N}\hat{F}\hWc\left(\sAccu,\sAccu\right)=\ZSet_{f}\left(\sAccu,\sAccu\right)$.


%%% Local Variables: 
%%% mode: latex
%%% TeX-master: "CDC14_Anytime_Main"
%%% End: 

\end{proof}
The control algorithm in Alg.~\ref{algo:RMPC-algo}, in each time step $k$, solves $\RAMPCProb{\de}{k}$ for each estimation mode $\de \in\Delta$ and selects the control input $u_{k}$ and the next estimation mode $\dek{k+1}$
corresponding to the best total cost $J_{\de}$.
Therefore, during the course of control, the algorithm may switch between the estimation modes in $\Delta$ depending on the system's state. Thm.~\ref{thm:robust-feasible-anytime-RMPC} states that if the control algorithm Alg.~\ref{algo:RMPC-algo} is feasible in its first time step then it will be robustly feasible and the state and control input constraints are also robustly satisfied.
\begin{thm}%[Robust Feasibility of RMPC with Anytime Estimation]
\label{thm:robust-feasible-anytime-RMPC}
If at the initial time step there exists $\left(\sDelay,\sAccu\right)\in\Delta$
such that $\RAMPCProb{\de}{0}$
is feasible then the system Eq.~\ref{eq:estimator-dynamics} controlled by
Alg.~\ref{algo:RMPC-algo} and subjected to disturbances constrained
s.t. $w_k\in \Wc,\forall k\geq0$ robustly satisfies the state constraint
$x_k\in\stSet,\forall k\geq0$ and the control input constraint $u_k\in\inpSet,\forall k\geq0$,
and all subsequent iterations of the algorithm are feasible.
\end{thm}
\begin{proof}
The Theorem can be proved by recursively applying Thm.~\ref{thm:robust-feasible-RMPC}.
Indeed, suppose at time step $k$ the algorithm
is feasible and results in control input $u_{k}$ and next estimation
mode $\dek{k+1}$, then $\RAMPCProb{\dek{k+1}}{k}$
is feasible. By Thm.~\ref{thm:robust-feasible-RMPC}, $u_{k}\in\inpSet$ and
at the next time step $k+1$, $\stPt_{k+1}\in\stSet$ and $\RAMPCProb{\dek{k+1}}{k+1}$
is also feasible, hence the algorithm is feasible.
Therefore, the Theorem holds by induction.
\end{proof}


%%% Local Variables: 
%%% mode: latex
%%% TeX-master: "CDC14_Anytime_Main"
%%% End: 


 %not in camera ready 

	In recent years, autonomous cars have become more and more prolific and increasingly capable \cite{}. For the perception and control software of these systems, there are stringent real-time requirements that have to be met for the safe performance of the system. In order for these requirements to be met, autonomous systems (both their hardware and software) are overengineered, resulting in a power consumption for the computation tasks that is not negligible. Also, while the focus of reducing power consumption in the design of such systems has been mostly on the power consumed by the actuators, computational platforms have been consuming more and more power. It is worth noting that over the past two decades, the power consumption of processors has increased by more than double, while battery energy density has only improved by about a quarter $\%$ \cite{}. 

In this poster, we present a method to obtain more energy efficient computations for the perception and estimation algorithms used in autonomous systems by exploiting hardware and algorithm level knobs. These knobs allow us to leverage a tradeoff between power consumption and a measure of quality of the perception and estimation algorithms. On-going work focuses also on coming up with a supervisory algorithm and control algorithms that are aware of this trade-off and choose an optimal operating point (or knobs settings) such that the performance of the overall closed-loop system is safe as well as energy efficient.

Our method consists of a two-phase approach. The first involves extensively profiling the perception and estimation algorithm offline with all knob settings for timing, quality of output (if available) and power consumption. It also involves developing a control rule which chooses the best operating point for the perception and estimation algorithm while ideally providing mathematical guarantees on the safe performance of the system, e.g. as in \cite{} . The second, online stage, involves picking the best operating mode and control setting and run-time monitoring of the system performance. 

We have developed an $1/10^{th}$ scale autonomous car testbed on which we evaluate our methods. In this particular work, we focus on a vanishing point algorithm based autonomous corridor navigation. We look at the algorithm level knobs, which are scheduling the sequential tasks of the vanishing point based algorithm on either the CPU or GPU; and the hardware level knobs which are frequencies of the CPU and GPU. Scheduling the tasks on different resources along with the different hardware settings result in a wide variety of timing and power distributions obtained.    
%
	


\end{document}
