Efficient computation of the constraint sets necessary for the RMPC formulation can be achieved for some special cases of disturbances and estimation errors as explained below.

\subsection{Compute State Constraint Sets}
\label{sec:ComputePontryagin}

If the error set $\ESet$ and the disturbance set $\WSet$ are defined
by vector norms, while the other sets are polytopes in H-representation
form, then it is possible to compute the Pontryagin difference in
the above equations efficiently \cite{setcomp}.%. See excerpts in \secref{Excerpts}.
\begin{itemize}
\item If $\ESet(\sAccu)\definedas\left\{ e\in\RR^{n}\SuchThat\norm{e}_{p}\leq\sAccu\right\} $
for some norm $p\in\left\{ 1,2,\infty\right\} $ then its support function
is $h_{\sAccu}(\eta)=\sAccu\norm{\eta}_{q}$ with $p^{-1}+q^{-1}=1$.
Similarly we also have $h_{w}(\eta)$.
\item Suppose $\ZSet=\left\{ z\SuchThat H_{\ZSet}z\leq b_{\ZSet}\right\} $
is a polytope in H-representation form (\ie intersection of a finite number
of half-planes). The rows of $H_{\ZSet}$ are $H_{\ZSet,i}^{T}$.
% \end{itemize}
% Then we can compute the constraint sets of the RMPC optimization efficiently.
%For example, to compute the shrinking state constraint sets in \eqref{RMPC1-Z}:
%\noindent\textit{Using \eqref{RMPC1-Z}:}
%\begin{itemize}
\item The most demanding computation in computing the constraint sets of the RMPC optimization is the Pontryagin difference.  If a set $V$ has support function $h_{V}$ then $\ZSet \ominus V$ can be computed as:
\begin{equation*}
      \ZSet \ominus V
      =\left\{ z\SuchThat H_{\ZSet}z\leq b_{\ZSet} -
       \begin{bmatrix}h_{V}\left(H_{\ZSet,1}\right)\\
         \vdots\\
         h_{V}\left(H_{\ZSet,m}\right)
       \end{bmatrix}\right\} 
 \end{equation*}
%
\item The support function $h_{\sAccu[k],\sAccu}$ for $\WhSet\left(\sAccu[k],\sAccu\right)$ is
\begin{equation*}
h_{\sAccu[k],\sAccu}(\eta)=h_{w}(\eta)+h_{\sAccu[k]}\left(A^{T}\eta\right)+h_{\sAccu}\left(-\eta\right)
\end{equation*}
%
% \item Compute 
%   \begin{equation*}
%     \begin{split}
%      \ZSet_{0}\left(\sAccu[k],\sAccu\right) 
%      &=\ZSet\ominus\hat{F}\ESet\left(\sAccu[k]\right) \\
%      &=\left\{ z\SuchThat H_{\ZSet}z\leq b_{\ZSet} -
%        \begin{bmatrix}h_{\sAccu[k]}\left(H_{\ZSet,1}\right)\\
%          \vdots\\
%          h_{\sAccu[k]}\left(H_{\ZSet,m}\right)
%        \end{bmatrix}\right\} 
%    \end{split}
%  \end{equation*}
%
% \item It is straightforward to show that
% \begin{equation*}
% \ZSet_{j}\left(\sAccu,\sAccu\right)= z\SuchThat H_{\ZSet}z\leq b_{\ZSet}-\begin{bmatrix}h_{\sAccu}\left(H_{\ZSet,1}\right)\\
% \vdots\\
% h_{\sAccu}\left(H_{\ZSet,m}\right)
% \end{bmatrix} \\
% -\sum_{i=0}^{j-1}\begin{bmatrix}h_{\sAccu,\sAccu}\left(\left(L_{j}\hat{F}\right)^{T}H_{\ZSet,1}\right)\\
% \vdots\\
% h_{\sAccu,\sAccu}\left(\left(L_{j}\hat{F}\right)^{T}H_{\ZSet,m}\right)
% \end{bmatrix}
% \end{equation*}
%
% \item Then we can compute
% %\begin{align}
% {\footnotesize
% \begin{equation}
% \begin{split}
% &\ZSet_{j+1}\left(\sAccu[k],\sAccu\right) =\ZSet_{j}\left(\sAccu,\sAccu\right)\ominus L_{j}\hat{F}\WhSet\left(\sAccu[k],\sAccu\right)\\
% &= z\SuchThat H_{\ZSet}z\leq b_{\ZSet}-\begin{bmatrix}h_{\sAccu}\left(H_{\ZSet,1}\right)\\
% \vdots\\
% h_{\sAccu}\left(H_{\ZSet,m}\right)
% \end{bmatrix} \\
% &-\sum_{i=0}^{j-1}\begin{bmatrix}h_{\sAccu,\sAccu}\left(\left(L_{i}\hat{F}\right)^{T}H_{\ZSet,1}\right)\\
% \vdots\\
% h_{\sAccu,\sAccu}\left(\left(L_{i}\hat{F}\right)^{T}H_{\ZSet,m}\right)
% \end{bmatrix}-\begin{bmatrix}h_{\sAccu[k],\sAccu}\left(\left(L_{j}\hat{F}\right)^{T}H_{\ZSet,1}\right)\\
% \vdots\\
% h_{\sAccu[k],\sAccu}\left(\left(L_{j}\hat{F}\right)^{T}H_{\ZSet,m}\right)
% \end{bmatrix}%\right\} 
% %\end{align}
% \end{split}
% \end{equation}
% }
\end{itemize}

\begin{comment}
\noindent\textit{Using \eqref{RMPC2-Yj-tight}:}
\begin{itemize}
\item We first derive the support function $h_{\YYY,j}$ of $\YYY_{j}\left(\sAccu[k],\sAccu\right)$:
%\[
{\footnotesize
\begin{equation}
h_{\YYY,j}\left(\eta\right)=\begin{cases}
h_{\sAccu[k]}\left(\begin{bmatrix}\IdentityMatrix_{n} & \bm{0}_{m}\end{bmatrix}\eta\right) & \text{\text{\text{if \ensuremath{j=0}}}}\\
\sum_{i=1}^{j}h_{w}\left(\left(\Phi^{j-i}\hat{F}\right)^{T}\eta\right)\\
+h_{\sAccu[k]}\left(\left(\Phi^{j-1}\hat{F}A\right)^{T}\eta\right)\\ +\sum_{i=2}^{j}h_{\sAccu}\left(\left(-\Phi^{j-i}\hat{B}K_{x}\right)^{T}\eta\right) & \text{if \ensuremath{j\geq1}}
\end{cases}
\end{equation}
}
%\]

\item And $Z_{j}$ can be computed as:
\[
\ZSet_{j}\left(\sAccu[k],\sAccu\right)=\left\{ z\SuchThat H_{\ZSet}z\leq b_{\ZSet}-\begin{bmatrix}h_{\YYY,j}\left(H_{\ZSet,1}\right)\\
\vdots\\
h_{\YYY,j}\left(H_{\ZSet,m}\right)
\end{bmatrix}\right\} 
\]

\end{itemize}
\end{comment}

The computation of $Z_{j}$ therefore involves only simple linear
algebra: matrix and vector multiplications, additions and substractions,
as well as vector norms. It is not difficult to see that we can write
\begin{equation}
\ZSet_{j}\left(\sAccu[k],\sAccu\right)=\left\{ z\SuchThat H_{\ZSet}z\leq b_{\ZSet}-\sAccu d_{j}-\sAccu[k]g_{j}\right\} \label{eq:impl-Zj-linear}
\end{equation}
where $d_{j}$ and $g_{j}$ are constant vectors, % whose elements are non-negative.
%Note that $d_{j}$ and $g_{j}$ 
which depend only on $\sDelay$. %
%the estimation mode $\left(\sDelay,\sAccu\right)$.
Therefore we can improve %the performance of
these computations further by pre-computing offline
these vectors; then once $\sAccu[k]$ is given, we can compute $Z_{j}$
in real time very fast.
%
The terminal set $Z_{f}$ can be computed in the same way.


\subsection{Compute Robust Control Invariant Set}

To compute the set $\CCC\left(\sDelay,\sAccu\right)$ in \eqref{RMPC1-Zf-invariant}:
\begin{enumerate}
\item Compute $\Omega=\ZSet_{N}\left(\sAccu,\sAccu\right)$ and
$\WhSet\left(\sAccu,\sAccu\right)$ (see above).
\item Call the Matlab Invariant Set Toolbox \cite{invset} to compute the maximal robust
control invariant set $\CCC\left(\sDelay,\sAccu\right)$ inside $\Omega$ subject to the disturbance
set $\WhSet\left(\sAccu,\sAccu\right)$. 
% We note that the control
% input is implicitly constrained via the constraint on the expanded
% state $z$, therefore we do not need to specify the input constraint
% explicitly (though we can use $\USet$ if we want).
\end{enumerate}


%%% Local Variables: 
%%% mode: latex
%%% TeX-master: "CDC14_Anytime_Main"
%%% End: 
