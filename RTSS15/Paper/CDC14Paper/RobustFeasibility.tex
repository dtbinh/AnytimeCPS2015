The RMPC formulation in \eqref{RMPC1}, with a fixed estimation mode
$\left(\sDelay,\sAccu\right)\in\Delta$, is designed to ensure that
it is robustly feasible, as stated in \thmref{robust-feasible-RMPC}.
\begin{theorem}
[Robust Feasibility of RMPC]\label{thm:robust-feasible-RMPC} For
any estimation mode $\left(\sDelay,\sAccu\right)$, if $\MPCProb{\sDelay,\sAccu}(\hat{x}_{k_{0}},\sDelay[k_{0}],\sAccu[k_{0}],u_{k_{0}-1})$
is feasible then the system (\ref{eq:disc-dynamics}) controlled by
the RMPC and subjected to disturbances constrained by \eqref{disturb-constraint}
robustly satisfies the state constraint (\ref{eq:state-constraint})
and the control input constraint (\ref{eq:input-constraint}), and
all subsequent optimizations $\MPCProb{\sDelay,\sAccu}(\hat{x}_{k},\sDelay[k],\sAccu[k],u_{k-1})$,
$\forall k>k_{0}$, are feasible.
\end{theorem}
\begin{proof}
See the Appendix.
\end{proof}
The control algorithm in \algoref{RMPC-algo}, in each time step $k$, solves $\MPCProb{\sDelay,\sAccu}(\hat{x}_{k},\sDelay[k],\sAccu[k],u_{k-1})$ for each estimation mode $\left(\sDelay,\sAccu\right)\in\Delta$ and selects the control input $u_{k}$ and the next estimation mode $\left(\sDelay[k+1],\sAccu[k+1]\right)$
corresponding to the best total cost $J_{\sDelay,\sAccu}^{\mathrm{total}}$.
Therefore, during the course of control, the algorithm may switch between the estimation modes in $\Delta$ depending on the system's state. \thmref{robust-feasible-anytime-RMPC} states that if the control algorithm \algoref{RMPC-algo} is feasible in its first time step then it will be robustly feasible and the state and control input constraints are also robustly satisfied.
\begin{theorem}%[Robust Feasibility of RMPC with Anytime Estimation]
\label{thm:robust-feasible-anytime-RMPC}
If at the initial time step there exists $\left(\sDelay,\sAccu\right)\in\Delta$
such that $\MPCProb{\sDelay,\sAccu}(\hat{x}_{0},\sDelay[0],\sAccu[0],u_{-1})$
is feasible then the system (\ref{eq:disc-dynamics}) controlled by
\algoref{RMPC-algo} and subjected to disturbances constrained
by \eqref{disturb-constraint} robustly satisfies the state constraint
(\ref{eq:state-constraint}) and the control input constraint (\ref{eq:input-constraint}),
and all subsequent iterations of the algorithm are feasible.
\end{theorem}
\begin{proof}
The Theorem can be proved by recursively applying \thmref{robust-feasible-RMPC}.
Indeed, suppose at time step $k$ the algorithm
is feasible and results in control input $u_{k}$ and next estimation
mode $\left(\sDelay[k+1],\sAccu[k+1]\right)$, then $\MPCProb{\sDelay[k+1],\sAccu[k+1]}(\hat{x}_{k},\sDelay[k],\sAccu[k],u_{k-1})$
is feasible. By \thmref{robust-feasible-RMPC}, $u_{k}\in\USet$ and
at the next time step $k+1$, $x_{k+1}\in\XSet$ and $\MPCProb{\sDelay[k+1],\sAccu[k+1]}(\hat{x}_{k+1},\sDelay[k+1],\sAccu[k+1],u_{k})$
is also feasible, hence the algorithm is feasible.
Therefore, the Theorem holds by induction.
\end{proof}


%%% Local Variables: 
%%% mode: latex
%%% TeX-master: "CDC14_Anytime_Main"
%%% End: 
