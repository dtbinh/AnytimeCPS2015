%\section{RMPC Formulation}

We formulate the RMPC optimization $\MPCProb{\sDelay,\sAccu}(\hat{x}_{k},\sDelay[k],\sAccu[k],u_{k-1})$
with respect to the nominal dynamics, which is the original dynamics
in \eqref{estimator-std-dynamics} but the disturbances are either
removed or replaced by nominal disturbances. To ensure robust feasibility
and safety, the state constraint set is tightened after each step
using a candidate stabilizing state feedback control, and a terminal
constraint is derived. In this RMPC formulation, we extend the approach
in \cite{richardsetal05rmp, chiscietal01swp}. At time step $k$, given
$(\hat{x}_{k},\sDelay[k],\sAccu[k],u_{k-1})$ and for a fixed $(\sDelay,\sAccu)$,
we solve the following optimization $\MPCProb{\sDelay,\sAccu}(\hat{x}_{k},\sDelay[k],\sAccu[k],u_{k-1})$:
\begin{subequations}
\label{eq:RMPC1}
\begin{align}
 & J_{\sDelay,\sAccu}^{\star}\left(\hat{x}_{k},\sDelay[k],\sAccu[k],u_{k-1}\right)=\min_{\boldsymbol{u},\boldsymbol{x}}\sum_{j=0}^{N}\ell(\Nom x_{k+j\Given k},u_{k+j\Given k})\\
 & \text{subject to, }\forall j\in\left\{ 0,\dots,N\right\} \nonumber \\
 & \Nom z_{k+j+1\Given k}=\hat{A}(\sDelay[k+j\Given k])\Nom z_{k+j\Given k}+\hat{B}(\sDelay[k+j\Given k])u_{k+j\Given k}\label{eq:RMPC1-dyn}\\
 & ( \sDelay[k+j+1\Given k],\sAccu[k+j+1\Given k] ) \!=\! (\sDelay,\sAccu ), (\sDelay[k\Given k],\sAccu[k\Given k]) \!=\! (\sDelay[k],\sAccu[k])  \label{eq:RMPC1-delay}\\
 & \Nom x_{k+j\Given k}=\begin{bmatrix}\IdentityMatrix_{n} & \bm{0}_{n\times m}\end{bmatrix}\Nom z_{k+j\Given k}\label{eq:RMPC1-z2x}\\
 & \Nom z_{k\Given k}=\left[\hat{x}_{k}^{T},u_{k-1}^{T}\right]^{T} \label{eq:RMPC1-z0}\\
 & \Nom z_{k+j\Given k}\in\ZSet_{j}\left(\sAccu[k],\sAccu\right)\label{eq:RMPC1-zset}\\
 & \Nom z_{k+N+1\Given k}\in\ZSet_{f}\left(\sAccu[k],\sAccu\right)\label{eq:RMPC1-zfinalset}
\end{align}
\end{subequations} in which $\Nom z$ and $\Nom x$
are the variables of the nominal dynamics. The constraints of the
optimization are explained below.
\begin{itemize}
\item \eqref{RMPC1-dyn} is the nominal dynamics.
\item \eqref{RMPC1-delay} states that the estimation mode is fixed at $\left(\sDelay,\sAccu\right)$
except for the first time step when it is $\left(\sDelay[k],\sAccu[k]\right)$.
\item \eqref{RMPC1-z2x} extracts the nominal state $\Nom x$ of the plant
from the nominal expanded state $\Nom z$.
\item \eqref{RMPC1-z0} initializes the nominal expanded state at time step
$k$ by stacking the current state estimate and the previous control
input.
\item \eqref{RMPC1-zset} tightens the admissible set of the nominal expanded
states by a sequence of shrinking sets.
\item \eqref{RMPC1-zfinalset} constrains the terminal expanded state to
the terminal constraint set $\ZSet_{f}$.
\end{itemize}

\noindent\textit{The state constraint $\ZSet_{j}$:}
%
The tightened state constraint sets $\ZSet_{j}\left(\sAccu[k],\sAccu\right)$
are parameterized with two parameters $\sAccu[k]$ and $\sAccu$.
They are defined as follows, for all $j\in\left\{ 0,\dots,N\right\} $\begin{subequations}
\begin{gather}
\ZSet_{0}(\sAccu[k],\sAccu)=\ZSet\ominus\hat{F}\ESet(\sAccu[k])\label{eq:RMPC1-Z0}\\
\ZSet_{j+1}(\sAccu[k],\sAccu)=\ZSet_{j}(\sAccu,\sAccu)\ominus L_{j}\hat{F}\WhSet(\sAccu[k],\sAccu)\label{eq:RMPC1-Zj}
\end{gather}
\label{eq:RMPC1-Z}\end{subequations} in which the symbol $\ominus$
denotes the Pontryagin difference between two sets. The set $\ZSet$
combines the constraints for both the plant's state and the control
input: $\ZSet=\XSet\times\USet$. The matrix $L_{j}$ is the state
transition matrix for the nominal dynamics in \eqref{RMPC1-dyn} under
a candidate state feedback gain $K_{j}(\sDelay)$, for $j\in\left\{ 0,\dots,N\right\}$
\begin{subequations}
\label{eq:RMPC1-L}
\begin{gather}
L_{0}=\IdentityMatrix\label{eq:RMPC1-L0}\\
L_{j+1}=(\hat{A}(\sDelay)+\hat{B}(\sDelay)K_{j}(\sDelay))L_{j}\label{eq:RMPC1-Lj}
\end{gather} 
\end{subequations}
Note that the possibly time-varying sequence $K_{j}(\sDelay)$ is designed for each choice of $\sDelay$ (\ie the system matrices $\hat{A}(\sDelay)$ and $\hat{B}(\sDelay)$), hence $L_{j}$ depends on $\sDelay$; however we write $L_{j}$ for brevity. The candidate control $K_{j}(\sDelay)$ is designed to stabilize the nominal system (\ref{eq:RMPC1-dyn}), desirably as fast as possible so that the sets $\ZSet_{j}$ are shrunk as little as possible. In particular, if $K_{j}(\sDelay)$ renders the nominal system nilpotent after $M<N$ steps then $L_{j}=\bm{0}$ for all $j\geq M$, therefore $\ZSet_{j}\left(\sAccu[k],\sAccu\right)=\ZSet_{M}\left(\sAccu[k],\sAccu\right)$ for all $j>M$.


\noindent\textit{The terminal constraint $\ZSet_{f}$:}
%
$\ZSet_{f}$ is given by %the Pontryagin difference
\begin{equation}
\label{eq:RMPC1-Zf}
\ZSet_{f}\left(\sAccu[k],\sAccu\right)=\CCC(\sDelay,\sAccu)\ominus L_{N}\hat{F}\WhSet(\sAccu[k],\sAccu)
\end{equation}
where $\CCC(\sDelay,\sAccu)$ is a robust control invariant admissible
set for $\sDelay$ \cite{kerrigan00rcs}, \ie there exists a feedback control law $u=\kappa(z)$
such that $\forall z\in\CCC(\sDelay,\sAccu)$
\begin{subequations}
\label{eq:RMPC1-Zf-invariant}
\begin{align}
& \hat{A}(\sDelay)z \!+\! \hat{B}(\sDelay)\kappa(z) \!+\! L_{N}\hat{F}w\in\CCC(\sDelay,\sAccu), \forall w\in\WhSet(\sAccu,\sAccu)\label{eq:RMPC1-Zf-invariant-dyn}\\
& z\in\ZSet_{N}\left(\sAccu,\sAccu\right)\label{eq:RMPC1-Zf-invariant-z}
\end{align}
\end{subequations}
We remark that $\CCC(\sDelay,\sAccu)$ does not depend on $\left(\sDelay[k],\sAccu[k]\right)$, therefore it can be computed offline for each mode $\left(\sDelay,\sAccu\right)$.


\subsection{Robust Feasibility}
\subsection{Proofs of Feasibility}
The RMPC formulation of the previous section, with a fixed estimation mode
$\left(\sDelay,\sAccu\right)\in\Delta$, is designed to ensure that the control problem is robustly feasible, as stated in the following theorem.
\begin{thm}
[Robust Feasibility of RAMPC]\label{thm:robust-feasible-RMPC} For
any estimation mode $\left(\sDelay,\sAccu\right)$, if $\RAMPCProb{\de}{k}$
is feasible then the system (\ref{eq:disc-dynamics}) controlled by
the RAMPC and subjected to disturbances constrained by $w_k \in \Wc$
robustly satisfies the state constraint $\stPt_k \in \stSet$
and the control input constraint $\inpPt_k \in \inpSet$, and
all subsequent optimizations $\MPCProb{\sDelay,\sAccu}(\hat{x}_{k},\sDelay[k],\sAccu[k],u_{k-1})$,
$\forall k>k_{0}$, are feasible.
\end{thm}
\begin{proof}
%
We will prove the theorem by recursion. We will show that if at any
time step $k$ the RMPC problem $\MPCProb{\sDelay,\sAccu}(\hat{x}_{k},\sDelay[k],\sAccu[k],u_{k-1})$
is feasible and feasible control input $u_{k}=u_{k\Given k}^{\star}$
is applied with estimation mode $\left(\sDelay[k+1],\sAccu[k+1]\right)=\left(\sDelay,\sAccu\right)$
then $u_{k}$ is admissible and at the next time step $k+1$, the
actual plant's state $x_{k+1}$ is inside $\stSet$ and the optimization
$\MPCProb{\sDelay,\sAccu}(\hat{x}_{k+1},\sDelay[k+1],\sAccu[k+1],u_{k})$
is feasible for all disturbances. Then we can conclude the theorem
because, by recursion, feasibility at time step $k_{0}$ implies robust
constraint satisfaction and feasibility at time step $k_{0}+1$, and
so on at all subsequent time steps.

Suppose $\MPCProb{\sDelay,\sAccu}(\hat{x}_{k},\sDelay[k],\sAccu[k],u_{k-1})$
is feasible. Then it has a feasible solution $\left(\{ \overline{z}_{k+j\Given k}^{\star}\} _{j=0}^{N+1},\{ u_{k+j\Given k}^{\star}\} _{j=0}^{N}\right)$
that satisfies all the constraints in \eqref{eq:RMPC1}. Now we will
construct a feasible candidate solution for $\MPCProb{\sDelay,\sAccu}(\hat{x}_{k+1},\sDelay[k+1],\sAccu[k+1],u_{k})$
at the next time step by shifting the above solution by one step.
Consider the following candidate solution for $\MPCProb{\sDelay,\sAccu}(\hat{x}_{k+1},\sDelay[k+1],\sAccu[k+1],u_{k})$:
\begin{subequations}
\label{eq:proofs:candidate-solution}
\begin{align}
\Nom z_{k+j+1\Given k+1} & =\Nom z_{k+j+1\Given k}^{\star}+L_{j}\hat{F}\hat{w}_{k}\label{eq:proofs:candidate-solution:zj}\\
\Nom z_{k+N+2\Given k+1} & =\hat{A}\left(\sDelay\right)\Nom z_{k+N+1\Given k+1}+\hat{B}\left(\sDelay\right)u_{k+N+1\Given k+1}\label{eq:proofs:candidate-solution:zN}\\
u_{k+i+1\Given k+1} & =u_{k+i+1\Given k}^{\star}+K_{i}\left(\sDelay\right)L_{i}\hat{F}\hat{w}_{k}\label{eq:proofs:candidate-solution:uj}\\
u_{k+N+1\Given k+1} & =\kappa\left(\Nom z_{k+N+1\Given k+1}\right)\label{eq:proofs:candidate-solution:uN}
\end{align}
\end{subequations} in which
$j\in\left\{ 0,\dots,N\right\} $, $i\in\left\{ 0,\dots,N-1\right\} $,
and $\kappa\left(\cdot\right)$ is the feedback control law for the
invariant set $\Cc(\sDelay,\sAccu)$ that is used in the terminal
set. We first show that the input and
state constraints are satisfied for $u_{k}$ and $x_{k+1}$, then
we will prove the feasibility of the above candidate solution for
$\MPCProb{\sDelay,\sAccu}(\hat{x}_{k+1},\sDelay[k+1],\sAccu[k+1],u_{k})$.

\noindent\textit{Validity of the applied input and the next state:}
%
The next plant's state is 
\begin{align*}
x_{k+1} & =Ax_{k}+B_{1}\left(\sDelay[k]\right)u_{k-1}+B_{2}\left(\sDelay[k]\right)u_{k}+w_{k}\\
 & =A\left(\hat{x}_{k}+e_{k}\right)+B_{1}\left(\sDelay[k]\right)u_{k-1}+B_{2}\left(\sDelay[k]\right)u_{k\Given k}^{\star}+w_{k}\\
 & =\begin{bmatrix}A & B_{1}\left(\sDelay[k]\right)\end{bmatrix}\begin{bmatrix}\hat{x}_{k}\\
u_{k-1}
\end{bmatrix}+B_{2}\left(\sDelay[k]\right)u_{k\Given k}^{\star} \\
&\qquad\qquad + e_{k+1}+\left(w_{k}+Ae_{k}-e_{k+1}\right)
\end{align*}
in which $e_{k+1}\in\ESet\left(\sAccu\right)$ and $\left(w_{k}+Ae_{k}-e_{k+1}\right)\in\hWc\left(\sAccu[k],\sAccu\right)$.
Note that $\Nom z_{k\Given k}^{\star}=\left[\hat{x}_{k}^{T},u_{k-1}^{T}\right]^{T}$.
Hence we have
\begin{align*}
\begin{bmatrix}x_{k+1}\\
u_{k}
\end{bmatrix} & =\hat{A}(\sDelay[k])\Nom z_{k\Given k}^{\star}+\hat{B}(\sDelay[k])u_{k\Given k}^{\star}\\
&\qquad\qquad +\hat{F}e_{k+1}+\hat{F}\left(w_{k}+Ae_{k}-e_{k+1}\right)\\
 & =\Nom z_{k+1\Given k}^{\star}+\hat{F}e_{k+1}+\hat{F}\left(w_{k}+Ae_{k}-e_{k+1}\right)
\end{align*}
where we use the dynamics in \eqref{eq:RMPC1-dyn}. From \eqref{eq:RMPC1-zset}
and \eqref{eq:RMPC1-Z}, $\Nom z_{k+1\Given k}^{\star}$ satisfies $\Nom z_{k+1\Given k}^{\star}\in\ZSet_{1}\left(\sAccu[k],\sAccu\right)=\ZSet\ominus\hat{F}\ESet\left(\sAccu\right)\ominus\hat{F}\hWc\left(\sAccu[k],\sAccu\right)$.
It follows that
\(
\left[ x_{k+1}^{T}, u_{k}^{T} \right]^{T} \in \ZSet = \stSet\times\inpSet\text{,}
\)
% which allows us to conclude that
therefore  $x_{k+1}\in\stSet$ and $u_{k}\in\inpSet$.


\noindent\textit{Initial condition:}
%
We have from \eqref{eq:estimator-std-dynamics} that $\hat{z}_{k+1}=\hat{A}(\sDelay[k])\hat{z}_{k}+\hat{B}(\sDelay[k])u_{k}+\hat{F}\hat{w}_{k}$.
On the other hand, by \eqref{eq:proofs:candidate-solution:zj},
\begin{align*}
\Nom z_{k+1\Given k+1} & =\Nom z_{k+1\Given k}^{\star}+L_{0}\hat{F}\hat{w}_{k}\\
 & =\hat{A}(\sDelay[k])\Nom z_{k\Given k}^{\star}+\hat{B}(\sDelay[k])u_{k\Given k}^{\star}+L_{0}\hat{F}\hat{w}_{k}\text{.}
\end{align*}
Note that $\Nom z_{k\Given k}^{\star}=\hat{z}_{k}$, $u_{k}=u_{k\Given k}^{\star}$,
and $L_{0}=\IdentityMatrix$. Therefore $\Nom z_{k+1\Given k+1}=\hat{z}_{k+1}$,
hence the initial condition is satisfied.


\noindent\textit{Dynamics:}
%
We show that the candidate solution satisfies the dynamics constraint
in \eqref{eq:RMPC1-dyn}. For $0\leq j<N$ we have
\begin{align*}
&\Nom z_{k+j+2\Given k+1} \\
=\, & \Nom z_{k+j+2\Given k}^{\star}+L_{j+1}\hat{F}\hat{w}_{k}\\
=\, &\hat{A}\left(\sDelay\right)\Nom z_{k+j+1\Given k}^{\star}+\hat{B}(\sDelay)u_{k+j+1\Given k}^{\star}+L_{j+1}\hat{F}\hat{w}_{k}\\
=\, &\hat{A}\left(\sDelay\right)\left(\Nom z_{k+j+1\Given k+1}-L_{j}\hat{F}\hat{w}_{k}\right) \\
&+\hat{B}(\sDelay)\left(u_{k+j+1\Given k+1}-K_{j}\left(\sDelay\right)L_{j}\hat{F}\hat{w}_{k}\right) +L_{j+1}\hat{F}\hat{w}_{k} \\
=\, &\hat{A}\left(\sDelay\right)\Nom z_{k+j+1\Given k+1}+\hat{B}(\sDelay)u_{k+j+1\Given k+1} \\
&-\left(\hat{A}\left(\sDelay\right) + \hat{B}(\sDelay)K_{j}\left(\sDelay\right)\right)L_{j}\hat{F}\hat{w}_{k}+L_{j+1}\hat{F}\hat{w}_{k}\\
=\, &\hat{A}\left(\sDelay\right)\Nom z_{k+j+1\Given k+1}+\hat{B}(\sDelay)u_{k+j+1\Given k+1}
\end{align*}
where the equality in \eqref{eq:RMPC1-Lj} is used to derive the last
equality. % from the previous one.
Therefore the dynamics constraint
is satisfied for all $0\leq j<N$. For $j=N$, the constraint is satisfied
by construction by \eqref{eq:proofs:candidate-solution:zN}.


\noindent\textit{State constraints:}
%
We need to show that $\Nom z_{(k+1)+j\Given k+1}\in\ZSet_{j}\text{\ensuremath{\left(\sAccu,\sAccu\right)}}$
for all $j\in\left\{ 0,\dots,N\right\} $. Consider any $0\leq j<N$.
\eqref{eq:RMPC1-Zj} states that $\ZSet_{j+1}\left(\sAccu[k],\sAccu\right)=\ZSet_{j}\left(\sAccu,\sAccu\right)\ominus L_{j}\hat{F}\hWc\left(\sAccu[k],\sAccu\right)$.
From the construction of the candidate solution we have $\Nom z_{k+j+1\Given k+1}=\Nom z_{k+j+1\Given k}^{\star}+L_{j}\hat{F}\hat{w}_{k}$,
where $\hat{w}_{k}\in\hWc\left(\sAccu[k],\sAccu\right)$ and $\Nom z_{k+j+1\Given k}^{\star}\in\ZSet_{j+1}\left(\sAccu[k],\sAccu\right)$.
By definition of the Pontryagin difference, we conclude that $\Nom z_{k+j+1\Given k+1}\in\ZSet_{j}\left(\sAccu,\sAccu\right)$
for all $j\in\left\{ 0,\dots,N-1\right\} $.

At $j=N$ the candidate solution in \eqref{eq:proofs:candidate-solution:zj}
gives us $\Nom z_{(k+1)+N\Given k+1}=\Nom z_{k+N+1\Given k}^{\star}+L_{N}\hat{F}\hat{w}_{k}$.
Because $\Nom z_{k+N+1\Given k}^{\star}\in\ZSet_{f}\left(\sAccu[k],\sAccu\right)=\Cc\left(\sDelay,\sAccu\right)\ominus L_{N}\hat{F}\hWc\left(\sAccu[k],\sAccu\right)$
and $\hat{w}_{k}\in\hWc\left(\sAccu[k],\sAccu\right)$, we have
$\Nom z_{(k+1)+N\Given k+1}\in\Cc\left(\sDelay,\sAccu\right)$. The
definition of $\Cc\left(\sDelay,\sAccu\right)$ in \eqref{eq:RMPC1-Zf-invariant}
implies $\Cc\left(\sDelay,\sAccu\right)\subseteq\ZSet_{N}\left(\sAccu,\sAccu\right)$.
Therefore $\Nom z_{(k+1)+N\Given k+1}\in\ZSet_{N}\left(\sAccu,\sAccu\right)$.


\noindent\textit{Terminal constraint:}
%
We need to show that $\Nom z_{k+N+2\Given k+1}\in\ZSet_{f}\left(\sAccu,\sAccu\right)=\Cc\left(\sDelay,\sAccu\right)\ominus L_{N}\hat{F}\hWc\left(\sAccu,\sAccu\right)$.
Add $L_{N}\hat{F}\hat{w}$, for any $w\in\hWc\left(\sAccu,\sAccu\right)$,
to both sides of \eqref{eq:proofs:candidate-solution:zN} and note that
$u_{k+N+1\Given k+1}=\kappa\left(\Nom z_{k+N+1\Given k+1}\right)$,
we have 
\begin{multline*}
  \Nom z_{k+N+2\Given
    k+1}+L_{N}\hat{F}\hat{w}=\hat{A}\left(\sDelay\right)\Nom
  z_{k+N+1\Given k+1} \\
  +\hat{B}\left(\sDelay\right)\kappa\left(\Nom
    z_{k+N+1\Given k+1}\right)+L_{N}\hat{F}\hat{w}\text{.}
\end{multline*}


 It follows from $\Nom z_{k+N+1\Given k+1}\in\Cc\left(\sDelay,\sAccu\right)$
and from the definition of the invariant control invariant admissible
set $\Cc\left(\sDelay,\sAccu\right)$ 
that $\Nom z_{k+N+2\Given k+1}+L_{N}\hat{F}\hat{w}\in\Cc\left(\sDelay,\sAccu\right)$
for all $w\in\hWc\left(\sAccu,\sAccu\right)$. Then by definition
of the Pontryagin difference, we conclude that $\Nom z_{k+N+2\Given k+1}\in\Cc\left(\sDelay,\sAccu\right)\ominus L_{N}\hat{F}\hWc\left(\sAccu,\sAccu\right)=\ZSet_{f}\left(\sAccu,\sAccu\right)$.


%%% Local Variables: 
%%% mode: latex
%%% TeX-master: "CDC14_Anytime_Main"
%%% End: 

\end{proof}
The control algorithm in Alg.~\ref{algo:RMPC-algo}, in each time step $k$, solves $\RAMPCProb{\de}{k}$ for each estimation mode $\de \in\Delta$ and selects the control input $u_{k}$ and the next estimation mode $\dek{k+1}$
corresponding to the best total cost $J_{\de}$.
Therefore, during the course of control, the algorithm may switch between the estimation modes in $\Delta$ depending on the system's state. Thm.~\ref{thm:robust-feasible-anytime-RMPC} states that if the control algorithm Alg.~\ref{algo:RMPC-algo} is feasible in its first time step then it will be robustly feasible and the state and control input constraints are also robustly satisfied.
\begin{thm}%[Robust Feasibility of RMPC with Anytime Estimation]
\label{thm:robust-feasible-anytime-RMPC}
If at the initial time step there exists $\left(\sDelay,\sAccu\right)\in\Delta$
such that $\RAMPCProb{\de}{0}$
is feasible then the system Eq.~\ref{eq:estimator-dynamics} controlled by
Alg.~\ref{algo:RMPC-algo} and subjected to disturbances constrained
s.t. $w_k\in \Wc,\forall k\geq0$ robustly satisfies the state constraint
$x_k\in\stSet,\forall k\geq0$ and the control input constraint $u_k\in\inpSet,\forall k\geq0$,
and all subsequent iterations of the algorithm are feasible.
\end{thm}
\begin{proof}
The Theorem can be proved by recursively applying Thm.~\ref{thm:robust-feasible-RMPC}.
Indeed, suppose at time step $k$ the algorithm
is feasible and results in control input $u_{k}$ and next estimation
mode $\dek{k+1}$, then $\RAMPCProb{\dek{k+1}}{k}$
is feasible. By Thm.~\ref{thm:robust-feasible-RMPC}, $u_{k}\in\inpSet$ and
at the next time step $k+1$, $\stPt_{k+1}\in\stSet$ and $\RAMPCProb{\dek{k+1}}{k+1}$
is also feasible, hence the algorithm is feasible.
Therefore, the Theorem holds by induction.
\end{proof}


%%% Local Variables: 
%%% mode: latex
%%% TeX-master: "CDC14_Anytime_Main"
%%% End: 



% \subsection{RMPC with Time-invariant Candidate Feedback Control}
% In the case when the candidate state feedback control $K_{j}(\sDelay)$
for each $\sDelay$ is time-invariant, \ie $K_{j}(\sDelay)=K(\sDelay)$
for all $0\leq j\leq N$, then the RMPC formulation~(\ref{eq:RMPC1})
can be specialized so that its feasibility set can be enlarged. In
particular, the shrinking state constraint sets $\ZSet_{j}$ in \eqref{RMPC1-Z}
are reformulated to be less tight.

Let $\Phi$ be the state transition matrix: $\Phi=\hat{A}(\sDelay)+\hat{B}(\sDelay)K(\sDelay)$.
Note that $\Phi$ depends on $\sDelay$; however we omit $\sDelay$
for brevity. We split $K(\sDelay)$ column-wise: $K(\sDelay)=\begin{bmatrix}K_{x}(\sDelay) & K_{u}(\sDelay)\end{bmatrix}$
where $K_{x}$ corresponds to the state component in $\Nom z$ and
$K_{u}$ corresponds to the input component in $\Nom z$ (again we
omit $\sDelay$ for brevity). Then the state constraint sets are defined
as follows, for all $j\in\left\{ 0,\dots,N\right\} $: \begin{subequations}
%\begin{gather}
\begin{align}
\ZSet_{j}\left(\sAccu[k],\sAccu\right)&=\ZSet\ominus\YYY_{j}\left(\sAccu[k],\sAccu\right)\label{eq:RMPC2-Zj}\\
\YYY_{j}\left(\sAccu[k],\sAccu\right)&= \hat{F}e_{k+j}+\sum_{i=1}^{j}\Phi^{j-i}\hat{F}\hat{w}_{k+i-1} 
\SuchThat \nonumber \\
e_{k+j}&\in\ESet(\sAccu[k+j\Given k]),\nonumber \\
\hat{w}_{k+i-1}&\in\WhSet(\sAccu[k+i-1\Given k],\sAccu[k+i\Given k])\,\forall i=1\dots j \label{eq:RMPC2-Yj}
\end{align}
%\end{gather}
\label{eq:RMPC2-Z}\end{subequations}

\begin{comment}
\begin{subequations}
\begin{gather}
%\begin{align}
\ZSet_{j}\left(\sAccu[k],\sAccu\right)=\ZSet\ominus\YYY_{j}\left(\sAccu[k],\sAccu\right)\label{eq:RMPC2-Zj}\\
\YYY_{j}\left(\sAccu[k],\sAccu\right)=\left\{ \hat{F}e_{k+j}+\sum_{i=1}^{j}\Phi^{j-i}\hat{F}\hat{w}_{k+i-1} \\
\SuchThat e_{k+j}\in\ESet(\sAccu[k+j\Given k]),\hat{w}_{k+i-1}\in\WhSet(\sAccu[k+i-1\Given k],\sAccu[k+i\Given k])\,\forall i=1\dots j\right\} \label{eq:RMPC2-Yj}
%\end{align}
\end{gather}
\label{eq:RMPC2-Z}\end{subequations}
\end{comment}


The definition in \eqref{RMPC1-Z}
corresponds to the following computation of $\YYY_{j}\left(\sAccu[k],\sAccu\right)$:  \begin{equation}
\begin{split}
\YYY_{j}\left(\sAccu[k],\sAccu\right)&=\left(\ESet(\sAccu[[k+j\Given k])\times\left\{ \bm{0}_{m}\right\} \right) \\
&\oplus\sum_{i=1}^{j}\Phi^{j-i}\hat{F}\WhSet(\sAccu[k+i-1\Given k],\sAccu[k+i\Given k])\label{eq:RMPC2-Yj-loose}
\end{split}
\end{equation}
where the sum operator means the Minkowski sum of sets.

To obtain a tighter approximation of $\YYY_{j}$, we first replace
$\hat{w}_{k+i-1}=w_{k+i-1}+Ae_{k+i-1}-e_{k+i}$ then expand the expression
to get, for $j\geq1$, \begin{equation}
\begin{split}
\hat{F}e_{k+j}+\sum_{i=1}^{j}\Phi^{j-i}\hat{F}\hat{w}_{k+i-1}=\sum_{i=1}^{j}\Phi^{j-i}\hat{F}w_{k+i-1} \\
+\Phi^{j-1}\hat{F}Ae_{k}+\sum_{i=2}^{j}\Phi^{j-i}\left(\hat{F}A-\Phi\hat{F}\right)e_{k+i-1}
\end{split}
\end{equation}
%\]
Then substitute the matrices from (\ref{eq:lifted-matrices}) and
with some straightforward algebraic and linear algebra manipulations,
we have for $j\geq1$, \begin{equation} 
\begin{split}
\hat{F}e_{k+j}+\sum_{i=1}^{j}\Phi^{j-i}\hat{F}\hat{w}_{k+i-1}=\sum_{i=1}^{j}\Phi^{j-i}\hat{F}w_{k+i-1} \\ 
+\Phi^{j-1}\hat{F}Ae_{k}-\sum_{i=2}^{j}\Phi^{j-i}\hat{B}K_{x}e_{k+i-1}
\end{split}
\end{equation}

%\]
 Therefore $\YYY_{j}$ can be computed as
{\small 
\begin{equation}
\YYY_{j}\left(\sAccu[k],\sAccu\right)=\begin{cases}
\ESet(\sAccu[k])\times\left\{ \bm{0}_{m}\right\}  & \text{\text{\text{if \ensuremath{j=0}}}}\\
\sum_{i=1}^{j}\Phi^{j-i}\hat{F}\WSet\oplus\Phi^{j-1}\hat{F}A\ESet(\sAccu[k]) \\ 
\oplus\sum_{i=2}^{j}\left(-\Phi^{j-i}\hat{B}K_{x}\right)\ESet(\sAccu) & \text{if \ensuremath{j\geq1}}
\end{cases}\label{eq:RMPC2-Yj-tight}
\end{equation}}
 in which the sum operator means the Minkowski sum of sets. It is
straightforward to show that the sets $\YYY_{j}\left(\sAccu[k],\sAccu\right)$
computed by \eqref{RMPC2-Yj-tight} are contained in (\ie tighter
than) the corresponding sets computed by \eqref{RMPC2-Yj-loose},
therefore by properties of the Pontryagin difference, the state constraint
sets $\ZSet_{j}\left(\sAccu[k],\sAccu\right)$ computed using \eqref{RMPC2-Yj-tight}
will be at least as large as %, and may be larger than, 
 those computed using \eqref{RMPC2-Yj-loose}.

%%% Local Variables: 
%%% mode: latex
%%% TeX-master: "CDC14_Anytime_Main"
%%% End: 
