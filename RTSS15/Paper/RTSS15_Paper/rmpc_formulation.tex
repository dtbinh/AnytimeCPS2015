\subsection{RMPC Formulation}

We formulate the RMPC optimization $\RAMPCProb{\de}{k}$
with respect to the nominal dynamics, which is the original dynamics
in Eq. \eqref{eq:estimator-std-dynamics} but the disturbances are either
removed or replaced by nominal disturbances. 
To ensure robust feasibility
and safety, the state constraint set is tightened after each step
using a candidate stabilizing state feedback control, and a terminal
constraint is derived. 
In this RMPC formulation, we extend the approach
in \cite{richardsetal05rmp, chiscietal01swp}. 
At time step $k$, given
$(\hat{x}_{k},\sDelay_k,\sAccu_k,u_{k-1})$ and for a fixed $(\sDelay,\sAccu)$,
we solve the following optimization 

\begin{subequations}
	\label{eq:RMPC1}
 \begin{equation} J_{\sDelay,\sAccu}^{*} \left(\hat{x}_{k},\sDelay_k,\sAccu_k,u_{k-1}\right) = \min_{\boldsymbol{u},\boldsymbol{x}}\sum_{j=0}^{N}\ell(\Nom x_{k+j\Given k},u_{k+j\Given k})
 \end{equation}
 \begin{equation}
  \text{subject to, }\forall j\in\left\{ 0,\dots,N\right\} \nonumber 
 \end{equation}
 \begin{equation}
  \Nom z_{k+j+1\Given k}=\hat{A}(\sDelay_{k+j\Given k})\Nom z_{k+j\Given k}+\hat{B}(\sDelay_{k+j\Given k})u_{k+j\Given k}\label{eq:RMPC1-dyn}
 \end{equation}
 \begin{equation}
  ( \sDelay_{k+j+1\Given k},\sAccu_{k+j+1\Given k} ) \!=\! (\sDelay,\sAccu ) \nonumber
 \end{equation}
 \begin{equation}
  (\sDelay_{k\Given k},\sAccu_{k\Given k}) \!=\! (\sDelay_k,\sAccu_k)  \label{eq:RMPC1-delay}
 \end{equation}
 \begin{equation}
  \Nom x_{k+j\Given k}=\begin{bmatrix}\IdentityMatrix_{n} \quad \bm{0}_{n\times m}\end{bmatrix}\Nom z_{k+j\Given k}\label{eq:RMPC1-z2x}
 \end{equation}
 \begin{equation}
  \Nom z_{k\Given k}=\left[\hat{x}_{k}^{T},u_{k-1}^{T}\right]^{T} \label{eq:RMPC1-z0}
 \end{equation}
 \begin{equation}
  \Nom z_{k+j\Given k}\in\ZSet_{j}\left(\sAccu_k,\sAccu\right)\label{eq:RMPC1-zset}
 \end{equation}
 \begin{equation}
  \Nom z_{k+N+1\Given k}\in\ZSet_{f}\left(\sAccu_k,\sAccu\right) \label{eq:RMPC1-zfinalset}
  \end{equation}
\end{subequations} 

in which $\Nom z$ and $\Nom x$
are the variables of the nominal dynamics. The constraints of the
optimization are explained below.
\begin{itemize}
\item \eqref{eq:RMPC1-dyn} is the nominal dynamics.
\item \eqref{eq:RMPC1-delay} states that the estimation mode is fixed at $\left(\sDelay,\sAccu\right)$
except for the first time step when it is $\left(\sDelay_k,\sAccu_k\right)$.
\item \eqref{eq:RMPC1-z2x} extracts the nominal state $\Nom x$ of the plant
from the nominal expanded state $\Nom z$.
\item \eqref{eq:RMPC1-z0} initializes the nominal expanded state at time step
$k$ by stacking the current state estimate and the previous control
input.
\item \eqref{eq:RMPC1-zset} tightens the admissible set of the nominal expanded
states by a sequence of shrinking sets.
\item \eqref{eq:RMPC1-zfinalset} constrains the terminal expanded state to
the terminal constraint set $\ZSet_{f}$.
\end{itemize}

\noindent\textit{The state constraint $\ZSet_{j}$:}
%
The tightened state constraint sets $\ZSet_{j}\left(\sAccu_k,\sAccu\right)$
are parameterized with two parameters $\sAccu_k$ and $\sAccu$.
They are defined as follows, for all $j\in\left\{ 0,\dots,N\right\} $
\begin{eqnarray}
\ZSet_{0}(\sAccu_k,\sAccu)=\ZSet\ominus\hat{F} \ESet(\sAccu_k)\label{eq:RMPC1-Z0}
\\
\ZSet_{j+1}(\sAccu_k,\sAccu)=\ZSet_{j}(\sAccu,\sAccu)\ominus L_{j}\hat{F}\hWc(\sAccu_k,\sAccu)\label{eq:RMPC1-Zj}
\label{eq:RMPC1-Z}
\end{eqnarray} 
in which the symbol $\ominus$
denotes the Pontryagin difference between two sets. The set $\ZSet$
combines the constraints for both the plant's state and the control
input: $\ZSet=\stSet\times\inpSet$. The matrix $L_{j}$ is the state
transition matrix for the nominal dynamics in \eqref{eq:RMPC1-dyn} under
a candidate state feedback gain $K_{j}(\sDelay)$, for $j\in\left\{ 0,\dots,N\right\}$
\begin{eqnarray}
\label{eq:RMPC1-L}
L_{0}=\IdentityMatrix\label{eq:RMPC1-L0}\\
L_{j+1}=(\hat{A}(\sDelay)+\hat{B}(\sDelay)K_{j}(\sDelay))L_{j}\label{eq:RMPC1-Lj}
\end{eqnarray}
Note that the possibly time-varying sequence $K_{j}(\sDelay)$ is designed for each choice of $\sDelay$ (i.e., the system matrices $\hat{A}(\sDelay)$ and $\hat{B}(\sDelay)$), hence $L_{j}$ depends on $\sDelay$; however we write $L_{j}$ for brevity. The candidate control $K_{j}(\sDelay)$ is designed to stabilize the nominal system (\ref{eq:RMPC1-dyn}), desirably as fast as possible so that the sets $\ZSet_{j}$ are shrunk as little as possible. In particular, if $K_{j}(\sDelay)$ renders the nominal system nilpotent after $M<N$ steps then $L_{j}=\bm{0}$ for all $j\geq M$, therefore $\ZSet_{j}\left(\sAccu_k,\sAccu\right)=\ZSet_{M}\left(\sAccu_k,\sAccu\right)$ for all $j>M$.


\noindent\textit{The terminal constraint $\ZSet_{f}$:}
%
$\ZSet_{f}$ is given by %the Pontryagin difference
\begin{equation}
\label{eq:RMPC1-Zf}
\ZSet_{f}\left(\sAccu_k,\sAccu\right)=\Cc(\sDelay,\sAccu)\ominus L_{N}\hat{F}\hWc(\sAccu_k,\sAccu)
\end{equation}
where $\Cc(\sDelay,\sAccu)$ is a robust control invariant admissible
set for $\sDelay$ \cite{kerrigan00rcs}, i.e., there exists a feedback control law $u=\kappa(z)$
such that $\forall z\in\Cc(\sDelay,\sAccu)$ and $\forall w\in\hWc(\sAccu,\sAccu)$
\begin{eqnarray}
\label{eq:RMPC1-Zf-invariant}
& \hat{A}(\sDelay)z \!+\! \hat{B}(\sDelay)\kappa(z) \!+\! L_{N}\hat{F}w\in\Cc(\sDelay,\sAccu) \label{eq:RMPC1-Zf-invariant-dyn}\\
& z\in\ZSet_{N}\left(\sAccu,\sAccu\right)\label{eq:RMPC1-Zf-invariant-z}
\end{eqnarray}
We remark that $\Cc(\sDelay,\sAccu)$ does not depend on $\left(\sDelay_k,\sAccu_k\right)$, therefore it can be computed offline for each mode $\left(\sDelay,\sAccu\right)$.
