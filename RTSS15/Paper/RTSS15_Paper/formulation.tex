\subsection{Problem formulation}
\label{formulation}

Consider a hexarotor, which is an autonomous flying robot with six rotors, shown in Fig.~\ref{fig:hexarotor}.
The state $\stPt$ of the hexarotor is made of its 3D position and 3D linear velocity.
The input $\inpPt$ consists of the desired pitch and roll angles, and the desired thrust.
The dynamics of the hexarotor, relating the time-evolution of its state to the current state and input, can be approximated by the following Linear Time-Invariant (LTI) ODE [??]:
\begin{equation}
\dot{x}(t)  =A_{c}x(t)+B_{c}u(t)+w_{c}(t)  \label{eq:plant-cont-model}
\end{equation}
where $x\in \Re^{n}$ is the state, $\inpPt\in\Re^{m}$ is the control input,
and $w_{c}\in\Re^{n}$ is the process noise. 
LTIs model a wide range of systems, and our results apply to arbitrary LTIs of the form given in \eqref{eq:plant-cont-model}.

The specific hexarotor equations are in [??].
The system is usually constrained to be in some safe set $S$, e.g. $\{Ax \leq b\}$.

For flight, the hexarotor needs to localize itself, i.e. it needs to produce an \emph{estimate} of its current state $x$.
It does so by taking a video through a downward facing camera, detecting and tracking features across frames, and deducing its own position relative to these features.
This produces a state estimate $\hat{x}$, which equals the true state $x + $ random error.
This estimate is fed to the controller which produces a control action $u$ based on it, constrained to be in a set, e.g. $\{l_b \leq u \leq u_b\}$.
This control is input to the linear system dynamics.

Sensing, estimation and actuation happen at discrete time instants, which produces de facto discretized dynamics.

The controller we use is a RMPC that takes into account estimation delays and errors: 
\\
- first, explain MPC
\\
- then augment with Robust feasible sets and delay modeling for full RAMPC formulation







