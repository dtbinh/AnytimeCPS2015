\subsection{System Model}
\label{formulation}

Consider a hexarotor, which is an autonomous flying robot with six rotors, shown in Fig.~\ref{fig:hexarotor}.
The state $\stPt$ of the hexarotor is made of its 3D position and 3D linear velocity.
The input $\inpPt$ consists of the desired pitch and roll angles, and the desired thrust.
The dynamics of the hexarotor, relating the time-evolution of its state to the current state and input, can be approximated by the following Linear Time-Invariant (LTI) ODE [??]:
\begin{equation}
\dot{x}(t) = A_{c}x(t)+B_{c}u(t)+w_{c}(t)  \label{eq:plant-cont-model}
\end{equation}
where $x\in \Re^{n}$ is the state constrained to lie in a set $\stSet \subset \Re^n$, 
$\inpPt\in\Re^{m}$ is the control input constrained to lie in a set $\inpSet \subset \Re^m$,
and $w_{c}\in\Re^{n}$ is the process noise assumed to lie in a set $\Wc \subset \Re^n$. 
LTIs model a wide range of systems, and our results apply to arbitrary LTIs of the form given in \eqref{eq:plant-cont-model} with the constraints $\stSet,\inpSet$ and $\Wc$.

For flight the hexarotor needs to determine its current position and speed, i.e., it needs to produce an \emph{estimate} of its current state $\stPt$.
It does so by taking a video during flight through a downward facing camera, detecting and tracking features across frames, and deducing its own position relative to these features.
This produces a state estimate $\hat{\stPt}$, which differs from the true state $\stPt$ by some (time-varying) error $e$: $\hat{\stPt} = \stPt + e$.
This estimate is fed to the controller which produces a control action $\inpPt$ based on it.
This control is input to the linear system dynamics according to \eqref{eq:plant-cont-model}.

\subsection{Control Performance}
The hexarotor's mission is to fly a pre-defined pattern. 
The goal of the controller is twofold: it needs to ensure that the reference pattern is adhered to as closely as possible, and that the energy consumed to fly this pattern is minimized.
Thus we may define two (stage) cost functions: first, $\ell(\stPt, \inpPt) = (\stPt - \stPt_{ref})^TQ(\stPt - \stPt_{ref})$ defines a weighted error between the actual state and the reference state $\stPt_{ref}$, and $Q$ is a positive semidefinite matrix.
Second, $\pi(\sDelay)$ captures the average power consumed to perform an estimation of duration $\sDelay$. 
This power information is collected offline during the estimator profiling phase.
The total cost function that the controller minimizes is then
\(
J=\sum_{k=0}^{\infty}\left(\ell(\stPt_{k},\inpPt_{k})+\pi(\sDelay_k)\right)
\),
The paper's formulation holds for much more general stage cost functions.
These stage cost functions are chosen by the designer to achieve a desired control performance.

\subsection{Time-Triggered Actuation and Discretized Dynamics}
We consider a discrete time implementation of the controller and the estimator, in which the estimator receives sensor measurements periodically at instants $t_{s,k}=kT$,
where $k\in\Ne$ and $T>0$ is a predefined sampling period.
See Fig.~\ref{fig:senseActuate}.
The sampled measurement is fed to the estimator that computes the state
estimate $\hat{x}_{k}\defeq\hat{x}(t_{s,k})$ with the desired
accuracy $\sAccu_k$ determined by the controller
in the previous time step. 
The controller then uses this state estimate
to compute the control input $u_{k}$ as well as decide on the desired
state estimate's delay and accuracy contract $(\sDelay[k+1],\sAccu[k+1])$ for the next step. 

In our setting, the controller has access to the delay-error curve of the estimator, and makes contract selections \emph{from that curve}. 
This curve is obtained offline as explained in Section \ref{codesign}, and exemplified in Section \ref{delayErrorCurve}.

Furthermore, we assume that the control implementation
is subject to the following assumption.
\begin{ass}
	[Time-triggered actuation]The control actuation is delayed by $\sDelay_k+\tau_k$, where $\tau_k$ is the time it takes to compute the input.
	I.e., the control input computed by the controller is applied exactly
	at the actuation instant $t_{a,k}=t_{s,k}+\sDelay_k+\tau_k$.
\end{ass}

The order of sensing\textendash{}computing\textendash{}actuating and
their timing are illustrated in the diagram in Fig.~\ref{fig:senseActuate}.
We remark that in each step $k\geq0$, the estimation accuracy $\sAccu_k$
and hence the delay $\sDelay_k$ are already decided in the previous
step and known to the controller. 
The previous control input $u_{k-1}$
is still used until $t_{a,k}$ when the new control input $u_{k}$
is computed and applied by the controller. 
The controller also chooses
the next desired accuracy $\sAccu_{k+1}$ and delay $\sDelay_{k+1}$
to be used in the next step $k+1$. 
In the first step $k=0$, the
initial accuracy $\sAccu[0]$, the initial delay $\sDelay[0]$, and
the initial control input $u_{-1}$ are chosen by the designer.







