\subsection{Experimental Results}

For the MPC and RAMPC formulated for the same horizon of 10 time steps, \ref{table:} shows the comparison of average tracking cost measured similar to as in equation \ref{}. To get a more accurate picture of how the controllers really performed, we use the following function 
\begin{equation}
J_{true} = \sum_t (x(t)-x_{ref}(t))'Q(x(t)-x_{ref}(t)) + u(t)'Ru(t)
\end{equation}
Note that since we have access to the true position of the hexrotor with the VICON system, we can obtain the true tracking cost. Table \ref{} also shows the estimated energy consumption based on the time spent in each mode. RAMPC shows better tracking performance than the MPC with either of the 4 fixed modes, showing the improved control performance that can be obtained by dynamically switching between estimation modes in-flight at runtime. Also noticeable in figure \ref{} is how the RAMPC provides better tracking performance, or a lower tracking cost while using lesser energy to do so. This shows that switching between estimation modes improves not only the control performance but also energy efficiency. 

Figure \ref{} shows the degradation in tracking performance and improvement in energy efficiency as the weight for the energy in the cost function, $\alpha$ is increased. This is expected as with more weight to the energy consumption, the RAMPC switches more often to the low energy (and faster) but less accurate mode 0, as is seen in table \ref{} which shows the fraction of time spent in the 4 modes with RAMPC as $\alpha$ changes. 