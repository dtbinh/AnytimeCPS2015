

\section{Co-design of computation and control}

In the traditional design of the perception estimation and control algorithms in closed loop control systems, the controller is unaware of the implementation details of the perception module and the perception module is unaware of the requirements of the controller. In order to improve performance of over-loaded closed loop systems, we propose the co-design of both computation and control. The co-design involves modelling the perception tool-chain as a contract time algorithm which can choose its execution paths adaptively in order to realize a deadline given to it by the control algorithm. In addition, the controller is designed with the knowledge of the resulting operation modes of the perception algorithm, i.e. the performance versus computation time modes which the contract time perception algorithm can realize. This gives the controller the ability to leverage the flexible nature of the perception algorithm to maximize a performance measure while not being concerned about the internal workings of the perception algorithm. This allows a co-design of control and perception/estimation without violating the separability principle which is key to the design of closed loop control systems.

Figure \ref{} shows the closed loop architecture in a system with co-design of the perception and control algorithms. Unlike a closed loop system where the perception module is designed to operate at a fixed operating point, in the co-designed system, the controller can make the estimation algorithm switch to lower time/energy consuming modes based on the controller requirements. 

