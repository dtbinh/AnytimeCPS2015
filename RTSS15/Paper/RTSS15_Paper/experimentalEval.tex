\subsection{Experimental Evaluation}

After profiling the performance of the perception and estimation algorithm and formulating the Robust Adaptive MPC controller for the hexrotor linearized around hover and modelled as an LTI system (Eq.~\ref{eq:plant-cont-model}), we experimentally evaluate the tracking performance and estimated energy consumption based on actual flights around a pre-defined trajectory. 
For comparison, we use a Model Predictive Controller with the same cost function and initial feasible sets as in our Robust MPC formulation. 
The MPC controller is an appropriate baseline against which to measure the benefits of our co-design method, as it is a similar control algorithm that does not leverage co-design and is unaware of the estimator algorithm that gives it a state estimate. 

For the evaluation, we fly in a predefined circular trajectory, repeating the experiment 10 times to gather enough data to conclusively measure the performance of RAMPC for different values of $\alpha$ and MPC with fixed modes of ($\delta,\epsilon$). 
Note that since the controller was a sampled discrete-time controller working with simulated 20Hz camera updates, this realistically restricts us to using modes of estimator operation with delay $\delta$ less than 1/20s, or one sampling period, i.e. modes corresponding to 50, 100, 150 and 200 maximum corners from the FAST detector (see Fig. \ref{fig:svo_error_delay}). 
These modes and their estimated power consumption is in Table \ref{tbl:modes_exp}. 
Note, \#C represents the maximum number of FAST corners requested, $\epsilon$ shows the worst case error bound on the state estimate, $\delta$ is the $90^{th}$ percentile execution time for that mode, and $P$ represents the expected power consumption in that mode as profiled offline. 
This power consumption is the computation power used by a particular mode in excess to the idle power for the Odroid used for profiling, which was 1.5W.

\begin{table}[htb]
\begin{center}
\caption{Estimation modes used in the experiment.}
\label{tbl:modes_exp}
\begin{tabular} {|c|c|c|c|c|}
	\hline
	\textbf{Mode} & \textbf{\#C} & $\pmb{\epsilon}$ & $\pmb{\delta}$ \textbf{(ms)} & $\pmb{P}$\textbf{(W)} \\ \hline
	0 & 50 &  24.88 & 0.028 &  0.778  \\ \hline
 	1 & 100 & 29.82 & 0.0237 &  0.862  \\ \hline
	2 & 150 & 34.66 & 0.0230 & 0.870 \\ \hline
	3 & 200 & 38.01 & 0.0113 & 0.951 \\ \hline
	\end{tabular}	
	\end{center}
\end{table}


