\subsection{Proofs of Feasibility}
The RMPC formulation of the previous section, with a fixed estimation mode
$\left(\sDelay,\sAccu\right)\in\Delta$, is designed to ensure that the control problem is robustly feasible, as stated in the following theorem.
\begin{thm}
[Robust Feasibility of RAMPC]\label{thm:robust-feasible-RMPC} For
any estimation mode $\left(\sDelay,\sAccu\right)$, if $\RAMPCProb{\de}{k}$
is feasible then the system (\ref{eq:disc-dynamics}) controlled by
the RAMPC and subjected to disturbances constrained by $w_k \in \Wc$
robustly satisfies the state constraint $\stPt \in \stSet$
and the control input constraint $\inpPt \in \inpSet$, and
all subsequent optimizations $\MPCProb{\sDelay,\sAccu}(\hat{x}_{k},\sDelay[k],\sAccu[k],u_{k-1})$,
$\forall k>k_{0}$, are feasible.
\end{thm}
\begin{proof}
%
We will prove the theorem by recursion. We will show that if at any
time step $k$ the RMPC problem $\MPCProb{\sDelay,\sAccu}(\hat{x}_{k},\sDelay[k],\sAccu[k],u_{k-1})$
is feasible and feasible control input $u_{k}=u_{k\Given k}^{\star}$
is applied with estimation mode $\left(\sDelay[k+1],\sAccu[k+1]\right)=\left(\sDelay,\sAccu\right)$
then $u_{k}$ is admissible and at the next time step $k+1$, the
actual plant's state $x_{k+1}$ is inside $\stSet$ and the optimization
$\MPCProb{\sDelay,\sAccu}(\hat{x}_{k+1},\sDelay[k+1],\sAccu[k+1],u_{k})$
is feasible for all disturbances. Then we can conclude the theorem
because, by recursion, feasibility at time step $k_{0}$ implies robust
constraint satisfaction and feasibility at time step $k_{0}+1$, and
so on at all subsequent time steps.

Suppose $\MPCProb{\sDelay,\sAccu}(\hat{x}_{k},\sDelay[k],\sAccu[k],u_{k-1})$
is feasible. Then it has a feasible solution $\left(\{ \overline{z}_{k+j\Given k}^{\star}\} _{j=0}^{N+1},\{ u_{k+j\Given k}^{\star}\} _{j=0}^{N}\right)$
that satisfies all the constraints in \eqref{eq:RMPC1}. Now we will
construct a feasible candidate solution for $\MPCProb{\sDelay,\sAccu}(\hat{x}_{k+1},\sDelay[k+1],\sAccu[k+1],u_{k})$
at the next time step by shifting the above solution by one step.
Consider the following candidate solution for $\MPCProb{\sDelay,\sAccu}(\hat{x}_{k+1},\sDelay[k+1],\sAccu[k+1],u_{k})$:
\begin{subequations}
\label{eq:proofs:candidate-solution}
\begin{align}
\Nom z_{k+j+1\Given k+1} & =\Nom z_{k+j+1\Given k}^{\star}+L_{j}\hat{F}\hat{w}_{k}\label{eq:proofs:candidate-solution:zj}\\
\Nom z_{k+N+2\Given k+1} & =\hat{A}\left(\sDelay\right)\Nom z_{k+N+1\Given k+1}+\hat{B}\left(\sDelay\right)u_{k+N+1\Given k+1}\label{eq:proofs:candidate-solution:zN}\\
u_{k+i+1\Given k+1} & =u_{k+i+1\Given k}^{\star}+K_{i}\left(\sDelay\right)L_{i}\hat{F}\hat{w}_{k}\label{eq:proofs:candidate-solution:uj}\\
u_{k+N+1\Given k+1} & =\kappa\left(\Nom z_{k+N+1\Given k+1}\right)\label{eq:proofs:candidate-solution:uN}
\end{align}
\end{subequations} in which
$j\in\left\{ 0,\dots,N\right\} $, $i\in\left\{ 0,\dots,N-1\right\} $,
and $\kappa\left(\cdot\right)$ is the feedback control law for the
invariant set $\Cc(\sDelay,\sAccu)$ that is used in the terminal
set. We first show that the input and
state constraints are satisfied for $u_{k}$ and $x_{k+1}$, then
we will prove the feasibility of the above candidate solution for
$\MPCProb{\sDelay,\sAccu}(\hat{x}_{k+1},\sDelay[k+1],\sAccu[k+1],u_{k})$.

\noindent\textit{Validity of the applied input and the next state:}
%
The next plant's state is 
\begin{align*}
x_{k+1} & =Ax_{k}+B_{1}\left(\sDelay[k]\right)u_{k-1}+B_{2}\left(\sDelay[k]\right)u_{k}+w_{k}\\
 & =A\left(\hat{x}_{k}+e_{k}\right)+B_{1}\left(\sDelay[k]\right)u_{k-1}+B_{2}\left(\sDelay[k]\right)u_{k\Given k}^{\star}+w_{k}\\
 & =\begin{bmatrix}A & B_{1}\left(\sDelay[k]\right)\end{bmatrix}\begin{bmatrix}\hat{x}_{k}\\
u_{k-1}
\end{bmatrix}+B_{2}\left(\sDelay[k]\right)u_{k\Given k}^{\star} \\
&\qquad\qquad + e_{k+1}+\left(w_{k}+Ae_{k}-e_{k+1}\right)
\end{align*}
in which $e_{k+1}\in\ESet\left(\sAccu\right)$ and $\left(w_{k}+Ae_{k}-e_{k+1}\right)\in\hWc\left(\sAccu[k],\sAccu\right)$.
Note that $\Nom z_{k\Given k}^{\star}=\left[\hat{x}_{k}^{T},u_{k-1}^{T}\right]^{T}$.
Hence we have
\begin{align*}
\begin{bmatrix}x_{k+1}\\
u_{k}
\end{bmatrix} & =\hat{A}(\sDelay[k])\Nom z_{k\Given k}^{\star}+\hat{B}(\sDelay[k])u_{k\Given k}^{\star}\\
&\qquad\qquad +\hat{F}e_{k+1}+\hat{F}\left(w_{k}+Ae_{k}-e_{k+1}\right)\\
 & =\Nom z_{k+1\Given k}^{\star}+\hat{F}e_{k+1}+\hat{F}\left(w_{k}+Ae_{k}-e_{k+1}\right)
\end{align*}
where we use the dynamics in \eqref{eq:RMPC1-dyn}. From \eqref{eq:RMPC1-zset}
and \eqref{eq:RMPC1-Z}, $\Nom z_{k+1\Given k}^{\star}$ satisfies $\Nom z_{k+1\Given k}^{\star}\in\ZSet_{1}\left(\sAccu[k],\sAccu\right)=\ZSet\ominus\hat{F}\ESet\left(\sAccu\right)\ominus\hat{F}\hWc\left(\sAccu[k],\sAccu\right)$.
It follows that
\(
\left[ x_{k+1}^{T}, u_{k}^{T} \right]^{T} \in \ZSet = \stSet\times\inpSet\text{,}
\)
% which allows us to conclude that
therefore  $x_{k+1}\in\stSet$ and $u_{k}\in\inpSet$.


\noindent\textit{Initial condition:}
%
We have from \eqref{eq:estimator-std-dynamics} that $\hat{z}_{k+1}=\hat{A}(\sDelay[k])\hat{z}_{k}+\hat{B}(\sDelay[k])u_{k}+\hat{F}\hat{w}_{k}$.
On the other hand, by \eqref{eq:proofs:candidate-solution:zj},
\begin{align*}
\Nom z_{k+1\Given k+1} & =\Nom z_{k+1\Given k}^{\star}+L_{0}\hat{F}\hat{w}_{k}\\
 & =\hat{A}(\sDelay[k])\Nom z_{k\Given k}^{\star}+\hat{B}(\sDelay[k])u_{k\Given k}^{\star}+L_{0}\hat{F}\hat{w}_{k}\text{.}
\end{align*}
Note that $\Nom z_{k\Given k}^{\star}=\hat{z}_{k}$, $u_{k}=u_{k\Given k}^{\star}$,
and $L_{0}=\IdentityMatrix$. Therefore $\Nom z_{k+1\Given k+1}=\hat{z}_{k+1}$,
hence the initial condition is satisfied.


\noindent\textit{Dynamics:}
%
We show that the candidate solution satisfies the dynamics constraint
in \eqref{eq:RMPC1-dyn}. For $0\leq j<N$ we have
\begin{align*}
&\Nom z_{k+j+2\Given k+1} \\
=\, & \Nom z_{k+j+2\Given k}^{\star}+L_{j+1}\hat{F}\hat{w}_{k}\\
=\, &\hat{A}\left(\sDelay\right)\Nom z_{k+j+1\Given k}^{\star}+\hat{B}(\sDelay)u_{k+j+1\Given k}^{\star}+L_{j+1}\hat{F}\hat{w}_{k}\\
=\, &\hat{A}\left(\sDelay\right)\left(\Nom z_{k+j+1\Given k+1}-L_{j}\hat{F}\hat{w}_{k}\right) \\
&+\hat{B}(\sDelay)\left(u_{k+j+1\Given k+1}-K_{j}\left(\sDelay\right)L_{j}\hat{F}\hat{w}_{k}\right) +L_{j+1}\hat{F}\hat{w}_{k} \\
=\, &\hat{A}\left(\sDelay\right)\Nom z_{k+j+1\Given k+1}+\hat{B}(\sDelay)u_{k+j+1\Given k+1} \\
&-\left(\hat{A}\left(\sDelay\right) + \hat{B}(\sDelay)K_{j}\left(\sDelay\right)\right)L_{j}\hat{F}\hat{w}_{k}+L_{j+1}\hat{F}\hat{w}_{k}\\
=\, &\hat{A}\left(\sDelay\right)\Nom z_{k+j+1\Given k+1}+\hat{B}(\sDelay)u_{k+j+1\Given k+1}
\end{align*}
where the equality in \eqref{eq:RMPC1-Lj} is used to derive the last
equality. % from the previous one.
Therefore the dynamics constraint
is satisfied for all $0\leq j<N$. For $j=N$, the constraint is satisfied
by construction by \eqref{eq:proofs:candidate-solution:zN}.


\noindent\textit{State constraints:}
%
We need to show that $\Nom z_{(k+1)+j\Given k+1}\in\ZSet_{j}\text{\ensuremath{\left(\sAccu,\sAccu\right)}}$
for all $j\in\left\{ 0,\dots,N\right\} $. Consider any $0\leq j<N$.
\eqref{eq:RMPC1-Zj} states that $\ZSet_{j+1}\left(\sAccu[k],\sAccu\right)=\ZSet_{j}\left(\sAccu,\sAccu\right)\ominus L_{j}\hat{F}\hWc\left(\sAccu[k],\sAccu\right)$.
From the construction of the candidate solution we have $\Nom z_{k+j+1\Given k+1}=\Nom z_{k+j+1\Given k}^{\star}+L_{j}\hat{F}\hat{w}_{k}$,
where $\hat{w}_{k}\in\hWc\left(\sAccu[k],\sAccu\right)$ and $\Nom z_{k+j+1\Given k}^{\star}\in\ZSet_{j+1}\left(\sAccu[k],\sAccu\right)$.
By definition of the Pontryagin difference, we conclude that $\Nom z_{k+j+1\Given k+1}\in\ZSet_{j}\left(\sAccu,\sAccu\right)$
for all $j\in\left\{ 0,\dots,N-1\right\} $.

At $j=N$ the candidate solution in \eqref{eq:proofs:candidate-solution:zj}
gives us $\Nom z_{(k+1)+N\Given k+1}=\Nom z_{k+N+1\Given k}^{\star}+L_{N}\hat{F}\hat{w}_{k}$.
Because $\Nom z_{k+N+1\Given k}^{\star}\in\ZSet_{f}\left(\sAccu[k],\sAccu\right)=\Cc\left(\sDelay,\sAccu\right)\ominus L_{N}\hat{F}\hWc\left(\sAccu[k],\sAccu\right)$
and $\hat{w}_{k}\in\hWc\left(\sAccu[k],\sAccu\right)$, we have
$\Nom z_{(k+1)+N\Given k+1}\in\Cc\left(\sDelay,\sAccu\right)$. The
definition of $\Cc\left(\sDelay,\sAccu\right)$ in \eqref{eq:RMPC1-Zf-invariant}
implies $\Cc\left(\sDelay,\sAccu\right)\subseteq\ZSet_{N}\left(\sAccu,\sAccu\right)$.
Therefore $\Nom z_{(k+1)+N\Given k+1}\in\ZSet_{N}\left(\sAccu,\sAccu\right)$.


\noindent\textit{Terminal constraint:}
%
We need to show that $\Nom z_{k+N+2\Given k+1}\in\ZSet_{f}\left(\sAccu,\sAccu\right)=\Cc\left(\sDelay,\sAccu\right)\ominus L_{N}\hat{F}\hWc\left(\sAccu,\sAccu\right)$.
Add $L_{N}\hat{F}\hat{w}$, for any $w\in\hWc\left(\sAccu,\sAccu\right)$,
to both sides of \eqref{eq:proofs:candidate-solution:zN} and note that
$u_{k+N+1\Given k+1}=\kappa\left(\Nom z_{k+N+1\Given k+1}\right)$,
we have 
\begin{multline*}
  \Nom z_{k+N+2\Given
    k+1}+L_{N}\hat{F}\hat{w}=\hat{A}\left(\sDelay\right)\Nom
  z_{k+N+1\Given k+1} \\
  +\hat{B}\left(\sDelay\right)\kappa\left(\Nom
    z_{k+N+1\Given k+1}\right)+L_{N}\hat{F}\hat{w}\text{.}
\end{multline*}


 It follows from $\Nom z_{k+N+1\Given k+1}\in\Cc\left(\sDelay,\sAccu\right)$
and from the definition of the invariant control invariant admissible
set $\Cc\left(\sDelay,\sAccu\right)$ 
that $\Nom z_{k+N+2\Given k+1}+L_{N}\hat{F}\hat{w}\in\Cc\left(\sDelay,\sAccu\right)$
for all $w\in\hWc\left(\sAccu,\sAccu\right)$. Then by definition
of the Pontryagin difference, we conclude that $\Nom z_{k+N+2\Given k+1}\in\Cc\left(\sDelay,\sAccu\right)\ominus L_{N}\hat{F}\hWc\left(\sAccu,\sAccu\right)=\ZSet_{f}\left(\sAccu,\sAccu\right)$.


%%% Local Variables: 
%%% mode: latex
%%% TeX-master: "CDC14_Anytime_Main"
%%% End: 

\end{proof}
The control algorithm in Alg.~\ref{RMPC-algo}, in each time step $k$, solves $\MPCProb{\de}(\hat{x}_{k},\sDelay[k],\sAccu[k],u_{k-1})$ for each estimation mode $\left(\sDelay,\sAccu\right)\in\Delta$ and selects the control input $u_{k}$ and the next estimation mode $\left(\sDelay[k+1],\sAccu[k+1]\right)$
corresponding to the best total cost $J_{\sDelay,\sAccu}^{\mathrm{total}}$.
Therefore, during the course of control, the algorithm may switch between the estimation modes in $\Delta$ depending on the system's state. Thm.~\ref{robust-feasible-anytime-RMPC} states that if the control algorithm Alg.~\ref{RMPC-algo} is feasible in its first time step then it will be robustly feasible and the state and control input constraints are also robustly satisfied.
\begin{thm}%[Robust Feasibility of RMPC with Anytime Estimation]
\label{thm:robust-feasible-anytime-RMPC}
If at the initial time step there exists $\left(\sDelay,\sAccu\right)\in\Delta$
such that $\MPCProb{\sDelay,\sAccu}(\hat{x}_{0},\sDelay[0],\sAccu[0],u_{-1})$
is feasible then the system (\ref{eq:disc-dynamics}) controlled by
Alg.~\ref{RMPC-algo} and subjected to disturbances constrained
by \eqref{disturb-constraint} robustly satisfies the state constraint
(\ref{eq:state-constraint}) and the control input constraint (\ref{eq:input-constraint}),
and all subsequent iterations of the algorithm are feasible.
\end{thm}
\begin{proof}
The Theorem can be proved by recursively applying Thm.~\ref{robust-feasible-RMPC}.
Indeed, suppose at time step $k$ the algorithm
is feasible and results in control input $u_{k}$ and next estimation
mode $\left(\sDelay[k+1],\sAccu[k+1]\right)$, then $\MPCProb{\sDelay[k+1],\sAccu[k+1]}(\hat{x}_{k},\sDelay[k],\sAccu[k],u_{k-1})$
is feasible. By Thm.~\ref{robust-feasible-RMPC}, $u_{k}\in\inpSet$ and
at the next time step $k+1$, $x_{k+1}\in\stSet$ and $\MPCProb{\sDelay[k+1],\sAccu[k+1]}(\hat{x}_{k+1},\sDelay[k+1],\sAccu[k+1],u_{k})$
is also feasible, hence the algorithm is feasible.
Therefore, the Theorem holds by induction.
\end{proof}


%%% Local Variables: 
%%% mode: latex
%%% TeX-master: "CDC14_Anytime_Main"
%%% End: 
