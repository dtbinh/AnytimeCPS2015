\section{Robust Model Predictive Control Solution}
\label{robustMPC}

In this section we give an overview of the \emph{Robust Adaptive Model Predictive Controller} (RAMPC) that we use in the contract-based setup of Fig.~\ref{fig:fullcodesignedCE}.
The mathematical details and derivations are available in the online technical report \cite{RTSS15TechRpt324}.
Experiments confirm that the following controller can be run in real-time, and its computation uses a negligible amount of time relative to the estimation delay.

\subsection{Solution overview}
Recall the operation of the contract-based control and estimation framework as presented in Section \ref{sec:codesign} and Fig.~\ref{fig:fullcodesignedCE}.
First, the estimator is profiled offline to obtain its delay-error curve, which we denote by $\Delta$.
The curve $\Delta$ represents a finite number of $\de$ contracts that the estimator can satisfy.
%Let $|\Delta|$ be the number of such contracts on the curve $\Delta$.
At every time step $k$, the controller receives a state estimate $\hat{\stPt}_k$ and uses it to compute two things:
first is the control input $\inpPt_{k}$ to be applied to the physical system at time $t_{a,k}$.
The second is the contract $(\sDelay_{k+1}, \sAccu_{k+1}) \in \Delta$ that will be requested from the estimator at the next step.
At $k+1$, the estimator provides an estimate with error at most $\sAccu_{k+1}$ and within delay $\sDelay_{k+1}$.
Finally, recall that 
$J = \sum_{k=0}^{M}\left(\ell(\stPt_k,\inpPt_k)+ \alpha \pi(\sDelay_k)\right)$ 
combines tracking error and input power in the $\ell$ terms, and estimation power consumption in the $\pi$ terms.
The scalar $\alpha$ quantifies the importance of power consumption to the overall performance of the system.

The contract-based controller's task is to find a sequence of inputs $u_k \in \inpSet$ and of contracts $(\sDelay_k, \sAccu_k) \in \Delta$ such that the cost $J$ is minimized, and the state $\stPt_k$ is always in the set $\stSet$.
The challenge in finding the control inputs is that the controller does not have access to the real state $\stPt_k$, but only to an estimate $\hat{\stPt}_k$.
The norm of the error $e_k = \hat{\stPt}_k - \stPt_k$ is bounded by the contractual $\sAccu_k$, which varies at each time step.

Fix the \emph{prediction horizon} $N \geq 1$.
Assume that the current contract (under which the current estimate $\hat{\stPt}_k$ was obtained) is $\dek{k}$, and that the previously applied input is $\inpPt_{k-1}$.
To compute the new input value $\inpPt_{k}$ and next contract $\dek{k+1}$, the proposed \textbf{Robust Adaptive Model Predictive Control (RAMPC)} seeks to solve the following optimization problem which we denote by $\RAMPCProb{\Delta}{k}$:
\begin{eqnarray}
\label{eq:fullOptim}
J^*[0:N]= \min_{\inpSig,\sttraj,\sDelayV,\sAccuV}\sum_{j=0}^{N}\left(\ell(\stPt_{k+j},\inpPt_{k+j})+\alpha\pi(\sDelay_k)\right)
\end{eqnarray}
Here, RAMPC needs to find the optimal length-$N$ input sequence  $\inpSig^* = (\inpPt_k^*,\ldots,\inpPt_{k+N}^*) \in \inpSet^N$, corresponding state sequence $\sttraj = (\stPt_k,\ldots,\stPt_{k+N}) \in \stSet^N$, delay sequence $\sDelayV = (\sDelay_k,\ldots,\sDelay_{k+N})$ and error sequence $\sAccuV = (\sAccu_k,\ldots,\sAccu_{k+N})$ such that $\dek{k} \in \Delta$, which minimize the $N$-step cost $J[0:N]$.
In the remainder of this section we discuss how to make this problem tractable.
As in regular MPC \cite{camachoetal04mpc}, once a solution $\inpSig^*$ is found, only the \emph{first} input value $u_k^*$ is applied to the physical system, thus yielding the next state $\stPt_{k+1}$ as per \eqref{eq:disc-dynamics}.
At the next time step $k+1$, RAMPC sets up the new optimization $\RAMPCProb{\Delta}{k+1}$ and solves it again.

To make this problem tractable, we first assume that the mode is fixed throughout the $N$-step horizon, i.e. $\dek{k+j} = \de$ for all $1 \leq j \leq N$.
Thus for every value $\de$ in $\Delta$, we can setup a different problem \eqref{eq:fullOptim} and solve it.
Let $J_{\de}^*$ be the corresponding optimum.
The solution with the smallest objective function value yields the input value $\inpPt^*_k$ to be applied and the next contract $(\sDelay^*,\sAccu^*)$.

Because RAMPC only has access to the state estimate, we extend the RMPC approach in \cite{richardsetal05rmp, chiscietal01swp}.
Namely, the problem is solved for the \emph{nominal dynamics} which assume zero process and observation noise ($w_{k+j} = 0)$ and zero estimation error ($\hat{\stPt}_{k+j}=\stPt_{k+j}$) over the prediction horizon.
Let $\Nom \stPt$ be the state of the system under nominal conditions.
To compensate for the use of nominal dynamics, RMPC replaces the constraint $(\Nom\stPt_{k+j},\inpPt_{k-1+j}) \in \stSet \times \inpSet \defeq \ZSet$ 
by $(\Nom \stPt_{k+j},\inpPt_{k+j}) \in \ZSet_{j}(\sAccu_k,\sAccu)$,
where $\ZSet_{j}(\sAccu_k,\sAccu) \subset \ZSet$ is $\ZSet$ `shrunk' by an amount corresponding to $\sAccu$, as explained in the technical report \cite{RTSS15TechRpt324}.
Intuitively, by forcing $(\Nom\stPt_{k+j},\inpPt_{k-1+j})$ to lie in the reduced set $\ZSet_{j}(\sAccu_k,\sAccu)$, the bounded estimation error and process noise are guaranteed not to cause the true state and input to exit the constraint sets $\stSet$ and $\inpSet$.
%
The tractable optimization for a given $\de$, denoted by $\RAMPCProb{\de}{k}$, is then 
\begin{eqnarray}
\label{eq:tractableOptim}
J^*_{\de} &=& \min_{\inpSig,\sttraj}\sum_{j=0}^{N}\left(\ell(\Nom \stPt_{k+j},\inpPt_{k+j})+\alpha\pi(\sDelay_k)\right)
\\
\textrm{s.t.} &&\forall j \in \{0,\ldots,N\}
\nonumber
\\
&&\Nom \stPt_{k+1}=A \Nom \stPt_{k}+B_{1}(\sDelay_k)\inpPt_{k-1}+B_{2}(\sDelay_k)\inpPt_{k}
\nonumber
\\
&& (\Nom \stPt_{k+j},\inpPt_{k+j}) \in \ZSet_{j}(\sAccu_k,\sAccu)
\nonumber
\end{eqnarray}

Algorithm \ref{algo:RMPC-algo} summarizes the RAMPC algorithm.
\begin{algorithm}
	\begin{algorithmic}[1]
		\State $\dek{0}$ and $u_{-1}$ specified by designer
		\State Apply $u_{-1}$
		\For{$k=0,1,\dots,M$}
		\State Estimate $\hat{\stPt}_{k}$ with guarantee $\left(\sDelay_k, \sAccu_k\right)$
		\For{each $\left( \sDelay,\sAccu \right) \in \Delta$}
		\State $J_{\de}^* \leftarrow $ Solve $\RAMPCProb{\de}{k}$		
		\EndFor
		\State $( \sDelay^{*}, \sAccu^{*}, u_{k }^{*} ) \gets \argmin_{\de} J_{\de}^*$
		%\State Wait until $t_{a,k}$
		\State Apply control input $u_{k} = u_{k}^{*}$ and estimation mode $\dek{k+1} = \left( \sDelay^{*}, \sAccu^{*} \right)$
		\EndFor
	\end{algorithmic} 
	
	\caption{Robust Adaptive MPC algorithm with Anytime Estimation.}
	\label{algo:RMPC-algo}
\end{algorithm}

We prove the following result in the technical report \cite{RTSS15TechRpt324}:
\begin{thm}%[Robust Feasibility of RMPC with Anytime Estimation]
	\label{thm:robust-feasible-anytime-RMPC}
	If at the initial time step there exists a contract value $\de \in \Delta$, 
	an initial state estimate $\hat{\stPt}_0 \in \stSet$,
	and an input value $\inpPt_{-1} \in \inpSet$,
	such that $\RAMPCProb{\de}{0}$
	is feasible then the system (\ref{eq:disc-dynamics}) controlled by
	Alg.~\ref{algo:RMPC-algo} and subjected to disturbances constrained
	by $w_k \in \Wc$ robustly satisfies the state constraint $\stPt \in \stSet$
	and the control input constraint $\inpPt \in \inpSet$,
	and all subsequent iterations of the algorithm are feasible.
\end{thm}




