\section{outline}

In autonomous cars, find numbers for power consumption of the perception chain.
(CMU, Shinpei)
\begin{enumerate}
	\item Introduction
	
%	\item Motivating example: 
%	\begin{itemize}
%			\item hexrotor that is unstable/poor-performing with MPC because of actuation delay
%			\item Operating at lowest delay (= worst performance) is not good either 
%			\item How can we reduce the actuation delay while keeping a handle on estimation error? 			
%	\end{itemize}
%	Anytime control formulation explicity accounts for delay/error and optimizes over it.
%	\\\quad A delay-error trade-off curve + control formulation that accounts for it

	\item Architecture of the solution: estimation and control talking to each other
		
	\item Control problem and solution (in a tutorial tone)
	\begin{itemize}
		\item MPC control formulation 
		\item Solution
	\end{itemize}
	
	\item Delay-error curve 
	\begin{enumerate}
		\item Perception and estimation toolchain. Each component has knobs that make it anytime.
		\item Perception+estimation is a task with varying utility and execution time. The controller schedules the task with the best utility/execution time.
		\item this is also a utility/computation power trade-off.
		\item Example: corner detection for visual odometry
		\item Example: Matlab toolchain and decision tree that effectively removes decision points from the curve
	\end{enumerate}
	
	
	\item Experiment
	\begin{itemize}
			\item Show the curve on the quadrotor with ordroid (or maybe move to previous section)
			\item Computation power
			\item Experimental setup
			\item Take system from intro and show results with RAMPC: better performance/ stable.
			\item Degradation experiment: as we decrease control frequency and increase flying speed, performance degrades. How does degradation compare between the two.
			
	\end{itemize}

\end{enumerate}